%!TEX root = ../ILR/revision.tex


  %\resizebox{\textwidth}{!}{%
  \tiny
       \begin{tabular}{p{11.22em}p{12.28em}p{12.28em}p{12.28em}p{14em}p{10em}p{12.28em}}
    \toprule
    & \textbf{Belgium} & \textbf{Germany} & \textbf{Spain} & \textbf{France} & \textbf{The Czech Republic} & \textbf{The United Kingdom} \\
    \midrule
    \textbf{Level of contractual negotiations} & Predominantly at the national, cross-industry level & Industry level between  trade unions and employers' organisations; agreements allow for flexibility at the company level & Predominant role of industry-level but interaction of negotiations at national and province-level, within industries & Peculiarly complex system of industrial relations:  all the levels of collective negotiations (inter-sectoral, industry or Firm-level) are closely intertwined and, in turn, they occur at both national or local level & Uncoordinated wage setting occurring directly between  firms and individuals. Principal Level of Collective Bargaining: company & Wage bargaining is mostly uncoordinated, with most workers bargaining work contracts individually with employers \\
    \midrule
    \textbf{Collective bargaining (\% firms) and Collective bargaining coverage rate (\% employees) } & 66.08\% of companies apply a collective agreement negotiated at higher level than the establishment or the company (in 2009); Collective bargaining coverage rate is 96\% in 2009 (ILO) &  Share of companies covered by forms of collective agreement above firm-level bargaining is around 66.92\% in 2009; Collective bargaining coverage rate is 61.7\% in 2009 (ILO) & Share of firms reporting to negotiate wages outside firms is 66.09\% in 2009; Collective bargaining coverage rate is 80.9\% in 2009 (ILO) & More than 50\% of companies declare to apply centralised bargaining in 2009; Collective bargaining coverage rate is 98\% in 2009 & 80\% of companies  conduct negotiations of wages at the firm or the establishment level; Collective bargaining coverage rate is 35\% in 2009 & 53.4\% of companies sign a firm-level agreement (in 2009); Collective bargaining coverage rate is 32.7\% in 2009 (ILO) \\
    \midrule
    \textbf{Topics covered by collective agreements} & Elements of pay and work conditions including national minimum wage, job creation measures, training and childcare provision set at the national level; industry and company bargaining mostly address non-pay issues & Wide range of issues such as pay, shift-work payments, pay structures, working time, treatment of part-timers and training & The national agreements covering the whole economy deal with non-pay issues such as training, equality and remote working, and since 2002 have, in a series of three-year deals, set broad guidelines on pay increases.\newline{}Lower-level agreements normally cover pay and working time. & Wide range of issues,  industry-level negotiation is obligatory in: pay; equality between women and men and measures to tackle the inequalities identified; working conditions, staffing and career development and exposure to occupational risks; disabled workers; occupational training; job classification; employee saving schemes; and arrangements for organising part-time work & Pay is the main subject of collective bargaining although there are also negotiations on other issues such as working time, work organisation, health and safety, work-life balance and employers’ contributions to pensions & Some negotiations cover all aspects of pay and conditions, but others are limited to only a few areas, principally pay \\
    \midrule
    \textbf{Derogation clauses} & Opening clauses dealing with wages appeared in sector-level agreements. Scarcely used in practice, covering six (sub)sectors: engineering; metal manufacturing; food manufacturing; retail of food products; large retail stores; department stores & Favourability principle prevents firm level agreements to set less favourable terms than those provided in agreements stipulated at higher levels. Wage derogations allowed at company level in times of serious economic difficulties and also in times of more general competitive problems & A company agreement might depart from the wages fixed by a collective agreement negotiated at a higher level, when, as a result of the application of those wages, the economic situation and prospects of the company could be damaged and affect jobs & The inversion of the favourability principle  introduced in 2004, recognising to firm-level agreements the possibility to derogate from any condition settled at more centralised levels, if not explicitly prohibited.  Four major issues are exempted from any derogation at company level: minimum wages; job classifications; supplementary social protection measures; multi-company and cross-sector vocational training funds. & Legal provision of the favourability principle prevents firm level agreements to set less favourable terms than those provided in agreements stipulated at higher levels & Agreements do not establish legally binding norms and, as a rule, they contain no contractual obligations, they are not subject to legal regulation, and pay rates cannot be claimed in court \\
    \bottomrule
    \end{tabular}%
 %  }


