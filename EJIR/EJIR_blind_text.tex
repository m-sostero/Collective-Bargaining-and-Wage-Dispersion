%!TEX root = EJIR_blind_text.tex

\documentclass[Review,times,sageh,11pt]{sagej}

\usepackage{amsmath,amssymb,amsfonts}
% \usepackage{microtype}
\usepackage{endnotes}
\usepackage[toc,page]{appendix}
\usepackage{rotating}
\usepackage{booktabs}
\usepackage{multirow}
\usepackage{threeparttable}

\let\footnote=\endnote

%\bibliographystyle{SageH}

\begin{document}

\raggedright
\frenchspacing

\section*{Introduction}
\label{sec:intro}
The notable rise in income inequality observed in many countries since the 2008 global recession has reopened the debate on its causes among social scientists. 
Together with technological change, globalization, the decline in union power, and the role of finance, scholars suggested that the process of increasing the flexibility of labour market and wage-setting institutions played a major role in increasing wage inequality~\citep{cobb2016}. Reforms in this direction were implemented between the late 1990s and the 2000s, in line with the policy recommendations of the 1994 OECD Jobs Strategy report.

Although the increased flexibility of work and wages may have affected inequality through several channels, this article focuses on the role played by the devolution of bargaining levels, i.e., the progressive shift in the locus of collective wage setting from more centralized levels (national or industry) to the level of individual firms~\citep{undy1978}. This trend has affected wage-setting through changes in legislation implemented between the 1990s and 2000s -- particularly in Europe -- to cater to the specific needs of firms, by allowing them to change wages based on their needs and local market conditions.
Over time, the ‘corporatist’ system of industrial relations~\citep{wallerstein1997unions}, which included high union coverage and centralized collective bargaining and was prevalent in European countries in the second half of the 20th century, has gradually morphed into a ‘hybrid’ system~\citep{Braakmann}. 
Although coordinated (or ‘multi-employer’) collective bargaining conducted at centralized level may still predominate, firm-level (‘single-employer’)  collective agreements increasingly derogate to specific provisions stipulated at centralized levels~\citep{visser2013wage}.

The increased role of firm-level collective agreements has been connected in the academic literature to two types of wage inequality, that is between or within firms.
Although they are both relevant in explaining overall wage inequality, each accounting for around half of overall wage dispersion in most economies~\citep{lazear.shaw.2007,fournier.koske.2013,GlobalWageReport}, the vast majority of available studies focuses on between-firm wage inequality, asking whether the wages of workers covered only by a centralized agreement are more or less dispersed than wages observed among workers who also bargain at the firm-level, on top of centralized contracts~\citep[see][among others]{dellaringa.lucifora.1994, card.delarica.2006, dellaringa.pagani.2007,daouli.demoussis.ea.2013}.
For example, looking at between-firm wage dispersion across countries \citet{Berlingieri2017} find evidence that greater collective bargaining coordination is associated with lower between-firm wage inequality, a finding highlighted in the OECD 2018 Economic outlook \citep{OECD2018}. This may be explained by greater bargaining power of high-skilled employees under local bargaining than under centralized bargaining, leading to even higher wages to well-paid workers under firm-level bargaining, as suggested by \cite{dahl.lemaire.ea.2013}.

In this paper, we instead focus on within-firm inequality, and investigate whether firms that apply firm-level bargaining have a more unequal wage structure, compared to those under centralized (national or sector) bargaining schemes. While wage differences between firms are certainly relevant to the overall wage distribution, they provide an incomplete picture of wage-setting dynamics. For instance, firm-level bargaining may allow highly-productive firms to set higher wages across the board, to attract or retain more productive employees -- an intended outcome of decentralized bargaining. In this case, between-firm wage inequality prompted by decentralized bargaining, would simply reflect productivity differences across companies and not necessarily result in higher wage dispersion across the employees of the same firm. Focusing on within-firm inequality allows us to better describe how the ability to set wages locally affects wage dynamics at the organization level. In particular, it tells whether firm-level bargaining is associated with higher wages for already high-paid employees, or even lower wages for the low-paid, and hence how employers shape overall income inequality. 

The literature on the role of firm-level bargaining for within-firm inequalities is relatively underdeveloped, in part because of the scarcity of detailed linked employer-employee earnings datasets. Existing studies are based on relatively old data, referring to the 1990s, when the push towards reforming labour markets in Europe was only beginning, and provide mixed results~\citep{dellaringa.lucifora.1994,dellaringa.lucifora.ea.2004, canaldominguez.gutierrez.2004}.
We use matched employer-employee data on six European countries -- Belgium, Spain, France, Germany, the Czech Republic and the UK -- available for the years 2006 and 2010. These countries come from different institutional traditions of collective wage bargaining, and still show marked differences in the prevalent form of collective bargaining, but by 2006 the de-centralization reforms enacted in late 1990s and beginning of 2000s had already resulted in significant hybridization of collective bargaining systems. Moreover, the data for 2010 is especially interesting, because it covers part of the Great Recession. During that time, wages were under pressure, and companies may have used the flexibility of firm-level bargaining to restructure wages.

Theoretically, the relation between firm-level bargaining and within-firm wage inequality depends on a combination of firm-specific and institutional- or context-specific factors characterising the environment and the time where a firm operates. 

Economic theories primarily stress firm-specific incentives as drivers for the adoption of firm-level agreements. But whether the possibility to also bargaining locally (on top of more centralized agreements) should result in higher or lower within-firm wage inequality depends on the specific theory.
Within-firm inequality is predicted to be higher in decentralized bargaining every time firm-level agreements are designed to elicit or selectively compensate the contribution of different employees to the firms' performance and objectives~\citep{bayo2013diffusion}.
This may happen under performance-related pay or other differentials-in-compensation schemes consistent with Tournament theory~\citep{lazear.1979}. It may also selectively remunerate human capital, or particularly valuable firm-specific resources (according to the resource-based view of the firm). It may even solve transaction costs and agency problems arising for different occupational groups~\citep{eisenhardt1989agency,o1998structure}. On the other hand, local bargaining may reduce within-firm inequality, compared to centralized bargaining, if these types of workplace collective agreements respond to motives of re-distribution, fairness or equity pursued by workers. This may stem from the preference of workers or unions to equalize wages (across but also within firms), as described in insider-outsider models ‘with unions’~\citep{lindbeck1986wage,lindbeck2001insiders}, or in ‘fair wage’ theories~\citep{akerlof.1984}. 
Similarly uncertain predictions result also from other wage-setting practices usually associated to between-firm inequality, such as efficiency wages, rent-sharing or differential compensations for unmeasured workers' ability. These may all affect inequalities within a firm, if they are used by employers to selectively reshape the overall pay-scale, to adjust wages of specific groups of workers, and not of others.

Beyond the incentive motives analysed by economists, other literatures highlight  firm-specific characteristics that explain how and why firms act on their internal wage structure. Sociological or socio-economic research -- and in particular recent developments of organisational approaches to stratification that discuss the firm as the central locus of wage inequality creation~\citep{stainback2010,cobb2016} -- stress the crucial role of organisational inertia and the relative balance of power among groups within organisations. Resistance to change favours the continuation of positions of individuals and wage structure within a firm, whereas the resolution of conflicts among groups with different goals and power in the hierarchical, organisational and occupational structure~\citep{blau1967american, goldthorpe1972occupational, wright1980class, erikson2002intergenerational}, may result into either reducing or increasing inequalities within firms, both statically and over time. 

Overall, the implementation of firm-level bargaining provisions is likely to vary considerably in different firms, with uncertain outcomes on within-firm wage inequality, depending on the relative strength of the factors mentioned above.

On top of firm-specific factors, the institutional setting wherein firms operate -- epitomised by the national system of industrial relations -- clearly matters, as it frames the conditions for implementing firm-level bargaining. The literature on this issue can be split between contributions that stress the differences in bargaining systems across countries, and those that underline broad similarities among them, possibly evolving over time.

On the one hand, notwithstanding the general trend towards devolution of bargaining levels in the period under analysis, countries arguably continue to show significant differences in terms of the scope, coverage and extent of derogation of firm-level bargaining vis-à-vis centralized bargaining. This suggests that, although the sign of the relationship between firm-level bargaining and within-firm wage inequality is uncertain overall, the institutional setting in some countries may favour inequality-enhancing effects of firm-level agreements more than in others. On the other hand, in spite of relevant institutional differences, many countries do share broad common features. In particular, some of them belongs to the same bargaining ‘regimes’~\citep{fulton.2013} in terms of the prevailing locus of collective bargaining (firm, sectoral or national level). One may, thus, expect that countries sharing the same regime should show similar effects of firm-level agreements on inequality, possibly more similar than country-specific institutional features alone would predict. 

This paper pays specific attention to the role of institutional settings to ask whether, on balance, firm-level bargaining is associated with higher or lower within-firm inequality. In particular, we provide two main contributions.

First, we check if the direction of the relation between firm-level bargaining and within-firm wage inequality differs across the countries under study. To avoid any assumption of homogeneity across national institutional settings, we estimate the same regression model separately for each country, as opposed to a pooled model with country fixed-effects, thus allowing coefficient estimates of firm-level bargaining to vary by country. Comparing coefficient estimates across countries allows us to appreciate the extent of heterogeneity or similarity across countries and bargaining regimes. As we further discuss in the next Section, country-specific features of bargaining systems support that while firm-level bargaining is more likely associated with increased within-firm wage inequality in the UK and the Czech Republic, egalitarian pressures are more likely in Germany and Spain, whereas Belgium is a country where firm-level bargaining has likely no effect and France is an harder to predict case.
Instead, grouping countries by bargaining regimes, provide more uncertain predictions, since it is a-priori uncertain whether egalitarian or inequality-enhancing pressures should prevail in economies (such as the UK but also recently the Czech Republic) where firm-level bargaining has always been commonplace, or in countries (like Germany, France, Belgium and Spain) that traditionally favour more centralized forms of bargaining.
In fact, any predicted cross-country heterogeneity of the effect of firm-level bargaining may eventually lose significance, if the ongoing convergence of wage-setting institutions had completely blurred the institutional differences across countries, or the borders across regimes~\citep[see][]{baccaro2017trajectories}.
If this were the case, our estimates should reveal a complete uniformity in the effect of firm-level bargaining across countries. 

Our second contribution is to examine whether the relations linking firm-level bargaining to within-firm inequality change over time, in between the two years covered in our data (2006 and 2010). Although the legal provision to stipulate firm-level collective agreements was introduced in all countries before the time-span under study, the pressure toward assigning more relevance and wider scope to firm-level negotiations likely continued during the period. It's likely that the use of firm-level agreements to differentiate salaries has increased over the years, as part of the trend toward the devolution of bargaining level and the diminished role of unions. We thus expect firm-level bargaining in 2010 to be associated with more unequal within-firm wage distributions compared to 2006.

In general, measuring the effect of institutional changes like collective bargaining regimes is a complex problem. The unit of analysis cannot be only an individual firm in isolation, because collective bargaining institutions can shape the labour market for all firms, even across different sectors, by changing the prevailing wages for all firms.

At the same time, the national transition to hybrid collective bargaining system occur endogenously to other institutional, policy and macro-economic changes. For example, some countries happened to reform their collective bargaining systems in times of recession, which makes it more difficult to disentangle the effect of policy changes from those of business cycles. This endogenous relation is even more problematic when firms are allowed to choose between centralized and decentralized bargaining schemes, which raises questions of comparability and selection between firms that choose one regime over the other. Ideally, to draw precise causal inferences on bargaining decentralization, we would like to observe countries with comparable institutions and conditions enacting discrete reforms to their collective bargaining institutions. Absent this ideal experiment, to derive comparable estimates by country, we control for employee and firm characteristics and account for the propensity of individual firms to adopt decentralized bargaining. Controlling for sector fixed-effects, as we do in the analysis, is especially important, to account for the possibility~\citep[put forward in][]{bechter2012sectors,hassel2014paradox} that industrial relations may be primarily driven by cross-national tendencies specific to industrial sectors, playing a role above and beyond country-specific institutional settings.



\section*{Firm-level bargaining in selected countries: characteristics and implications for within-firm inequality}
\label{sec:bargaining_models}

The countries we cover in this study -- Belgium, the Czech Republic, Germany, Spain, France and the United Kingdom -- provide a good representation of collective wage-bargaining traditions in Europe. In this section we present a brief overview of similarities and differences in the role and scope of firm-level bargaining in the various national systems, over the time span covered in the analysis. This allows  sketching hypotheses about heterogeneous, as opposed to common effects of firm-level bargaining on within-firm inequality that we could expect to observe in the empirical analysis across national contexts.\footnote{We draw from a number of data-sources and reports, referenced in the text. See~\cite{fulton.2013,fulton.2015} for a broader discussion of legal and institutional aspects featuring the bargaining systems of different countries.} 

Collective wage bargaining in \textbf{Belgium} is highly structured, starting from agreements at the national level, covering the entire private sector, followed by industry-level agreements, and company level negotiations at the bottom.
Wage bargaining takes place predominantly at the national, cross-industry level.
Notwithstanding two reforms occurred during the period under analysis (as reported in the \textit{Labour Market Reforms}-LABREF database maintained by the European Commission\footnote{The LABREF dataset is available on-line at~https://webgate.ec.europa.eu/labref/public.}), the percentage of employees covered by collective bargaining has remained steady at 96\% over the period 2006--2010 (source: ILOstat database\footnote{See the section ‘Industrial Relations’ of the ILOstat website~http://www.ilo.org/ilostat.}).
According to the data from the~\textit{European Company Survey}-ECS (run by the Eurofound Industrial Relations Observatory), in 2009 66.08\% of companies apply a collective agreement which has been negotiated at higher level than the establishment or the company, while 88.2\% of companies applying national, inter-sectoral or sectoral collective bargaining declare it was not possible for them to derogate from these agreements. Elements of pay and work conditions -- including national minimum wage, job creation measures, training and childcare provision -- are set in binding national agreements, while industry and company bargaining mostly address non-pay issues, not affected by the ceiling imposed by the central agreement~\citep{visser2013wage}. The room for pay bargaining at the enterprise level is further limited due to indexation of wages in national agreements.
As a result, we do not expect firm-level bargaining to play a major role in this country: the scope for local contracting to affect internal wage structures is limited, with no major changes over time.

In \textbf{Germany}, wages are bargained mostly at the industry level between individual trade unions and employers' organisations, although the agreements allow for flexibility at the company level. Collective agreements regulate a wide range of issues such as pay, shift-work payments, pay structures, working time, treatment of part-timers and training. Work councils play a central role because they can reach agreements with individual employers on issues not covered by collective agreements, or negotiate improvements on pay-related and other issues already covered by collective agreements, under the so-called ‘favourability principle’.
During the period considered in our analysis, some reforms were implemented in the field of wage setting policies, such as the introduction of binding minimum wages in several sectors (LABREF data). However, the large prevalence of the higher bargaining level remains quite stable over time. Indeed, according to the ECS data, in 2009, the share of companies covered by forms of collective agreement above firm-level bargaining was around 66.92\%, and the possibility to derogate from these higher level agreements was open to only 17\% of the surveyed companies. Moreover, the percentage of employees covered by enterprise-level agreements amounted to less than 10\% in 2006 and did not significantly change in 2010 (source: the \textit{Institutional Characteristics of Trade Unions, Wage Setting, State Intervention and Social Pacts}-ICTWSS database). Given these features and the central role of workers/unions in the work councils, typically pushing toward wage standardization, we could conjecture that firm-level agreements signed on top of more centralized bargaining are especially likely to pursue egalitarian purposes in this country, thus compressing wage dispersion within firms that adopt them.

In \textbf{Spain}, the majority of firms (66.09\% in 2009 according to the ECS dataset) report to negotiate their wages outside the firm, and much like in Germany, less than 10\% of employees were covered by firm-level agreements in both 2006 and 2010 (according to ICTWSS data). The structure of wage bargaining system shows a predominant role of industry-level but there are features that are quite peculiar to this country. There is indeed a peculiarly complex coexistence and interaction of negotiations at national and province-level, within industries. On top of this, the majority of single-employer agreements in Spain are initiated by works
councils or trade union delegations and firms adopting firm-level collective bargaining traditionally feature a higher union density than multi-employer bargaining firms~\citep{plasman.rusinek.ea.2007}. This all suggests that in Spain the union's pressure to compress wage inequalities may be particularly strong in firm-level bargaining firms.

The \textbf{United Kingdom} epitomises the Anglo-Saxon tradition of industrial relations, where wage bargaining is mostly uncoordinated, with most workers bargaining work contracts individually with employers. In fact, only about a third of all employees (33.3\% in 2006 and 30\% in 2010, according to ILOstat) is covered by some forms of collective bargaining. When a collective agreement occurs, the majority of them are signed at the firm-level (53.4\% of companies in 2009, according to the ECS), but such agreements do not establish legally binding norms and, as a rule, they contain no contractual obligations, they are not subject to legal regulation, and pay rates cannot be claimed in court~\citep{visser2013wage}.
Also, collective agreements are very rare in the private sector, while in the public sector employee coverage is more comparable to other countries~\citep{fulton.2013}.
This warrants public servants some more protection, although in May 2010 an emergency budget was approved freezing wages for high earners in the public sector for a two-year period as a temporary measure to face the 2008 global crisis (see LABREF data). Altogether, in view of these features and of the traditionally high flexibility in the use and content of heterogeneous pay schemes at firm-level in this country, we suppose that in the UK within-firm pay inequalities are especially wide in firms bargaining locally compared to other firms.

The \textbf{Czech Republic} belongs in the Eastern European trend to embrace decentralized, market-oriented institutional settings in the post-Soviet era.
In line with this, uncoordinated wage setting occurring directly between firms and individuals are quite spread. In fact, the employees covered by some form of collective wage bargaining are 50.8\% in 2006 and 50.1\% in 2010, according to ILOstat data. When collective agreements are reached, they mostly occur at firm level: in Eurofound-ECS dataset, more than 80\% of companies declare to have conducted negotiations of wages at the firm or the establishment level. Also, when different forms of collective agreements coexist, the legal provision of the favorability principle prevents firm-level agreements to set less favourable terms than those provided in agreements stipulated at higher levels. However, collective agreements signed at the industry level last for at least two years, while those signed at company level may be renegotiated every year, thus allowing for a certain degree of flexibility in reshaping the wage ladder in the enterprise. These features suggest that in the Czech Republic, similarly to the UK, there is room for firm-level bargaining to result into more unequal within-firm pay structures.

The last country that we analyse, \textbf{France}, is characterised by a peculiarly complex system of industrial relations, where all the levels of collective negotiations -- inter-sectoral, industry or firm-level -- are closely intertwined and, in turn, they occur at both national or local level~\citep{fulton.2013,fulton.2015}.
More than 50\% of companies declare to apply centralized bargaining in 2009 (in the ECS data), with industry-level bargaining standing out as the most important in terms of numbers of employees covered (97.3\% in 2006 and 98\% in 2010 according to ILOstat data). But the vast majority of companies applies a combination of different levels.
The inversion of the favorability principle was introduced in 2004, recognising to firm-level agreements the possibility to derogate from any condition settled at more centralized levels, if not explicitly prohibited~\citep{keune2011decentralizing}.
This mostly concerned working time, however, and few firms exploited the opportunity to act on pay structures. Altogether, the combination of elements pushing to increase flexibility in the firm with the enduring and complex role of centralized bargaining levels, makes it particularly difficult to predict whether firm-level bargaining firms should show unequal wage structures than other firms in this country.

Overall, the review of national wage bargaining systems reveals clear cross-country differences during the period under study, in terms of bargaining coverage, structure and mechanisms of coordination. We thus expect heterogeneous effects should emerge in the estimated association between firm-level bargaining and within-firm inequality, depending on the national context. In particular, as the review shows, the national systems of the UK and the Czech Republic are more likely to result in firms bargaining locally having more unequal wage structures, compared to firms that only bargaining at more centralized levels. Conversely, more egalitarian outcomes associated to firm-level contracts seem likely to be in place in Germany and Spain, whereas no effect is predicted to emerge in Belgium and outcomes are uncertain in France. Our choice to perform separate analyses by country is exactly intended to verify such potentially heterogeneous effects.

At the same time, despite differences in tradition and legislation, the review also suggests elements of broad similarities, across at least some of the countries, in particular concerning the prevailing locus of collective bargaining. In fact, based on this characteristic, the six countries can be classified according to the following ‘bargaining regimes’~\citep[see][]{fulton.2013}: Belgium as an emblematic example of the “inter-industry/national regime”; the UK and the Czech Republic as representing instances of an opposite ‘individual-employer regime’; Spain and Germany as falling into the intermediate “sectoral regime”; and France somehow outlying due the specifically complex interaction across all levels.

This grouping by prevailing bargaining level allows to qualify the hypotheses about the role of firm-level bargaining for within-firm inequalities in different countries. In fact, broad tendencies common to all the countries in the same regime likely interact with country-specific predictions outlined above. On the one hand, it may be argued that where more centralized or complex models prevail -- like in Belgium, Spain, Germany and France-- there is stronger resistance (by the laws or due to workers' action) to allow for firm-level contracts to introduce inequalities in the firm internal wage structure. This would reduce the likelihood that in these countries, compared to the UK or the Czech Republic, firms bargaining locally show wage structures that are more unequal -- if at all -- than across firms bargaining at higher levels. On the other hand, conversely, one could argue that in the same countries, firm-level bargaining is used more markedly by firms to differentiate their pay structures, precisely to escape the rigidity and complexities of negotiation typical of more centralized regimes. Eventually, the progressive decentralization of wage-setting gained traction precisely to allow firms more freedom to shape internal incentives, compared to the limited margins of manoeuvre allowed for by corporatist industrial relations. If this second tendency prevails, we could expect firm-level bargaining firms to display larger within-firm inequalities than other firms in Spain, Germany, France or Belgium, than in the UK or the Czech Republic.


\section*{Data and main variables}
\label{sec:data}

\subsection*{Data source and sample}
\label{sec:sources}
The \emph{Structure of Earnings Survey}~(SES) dataset collected by Eurostat is an established source of information for labour dynamics across Europe, indeed repeatedly used in previous studies. It collects a rich set of earnings-related, personal and jobs-related variables for a vast set of workers, matched with information on some characteristics of the employing firms.

For this study, we had access to the 2006 and 2010 waves of the SES for Belgium, Germany, Spain, France, the Czech Republic and the United Kingdom. We pool the two waves of the survey in the empirical analysis, but the pooled data must be intended as a repeated cross-section, since the SES does not report any identification code that can be used to match the same firm or the same employee over time.

The structure of the SES data is such that, for each country, a random sample of firms (stratified by size, sector of activity and geographical location) is selected to be representative of the national industrial system. Then, within each selected firm, a representative sample of employees is drawn, and for those employees a large set of personal and job-related characteristics is provided, including wages, age, gender, education, type of contract, tenure, occupation type, and others. 

As such, the available dataset can be seen as a matched employer-employee dataset, representing a unique source for a consistent comparison across European economies.  Of course, SES data have their own limitations. First, the sample of business units considered in SES is restricted to those with at least 10 employees, which limits the analysis as far as micro firms are concerned. Second, the data are very rich concerning employees' personal and work-related characteristics, but the information on firms is limited to five variables: size class, geographical location, sector of activity, public vs.\ private control and -- crucial for our purposes -- the level of wage bargaining adopted in the firm. Third, while the surveying procedure provides information on an impressive number of workers across Europe (about 10 million per survey year), for the firms which enter the data the sampling rate of employees varies by firm size and by country. Since measuring within-firm wage inequality requires, by definition, to observe wages for at least two employees of the same firm, we define our working sample as including only firms with at least three sampled employees.

\subsection*{Identification of collective bargaining levels across firms}
\label{sec:decentralisation}
The information provided in SES which is central to our purposes is the type of wage bargaining in place at each firm. Across the countries in our study, the data allow to distinguish three broad cases of collective bargaining coverage. The first case includes firms negotiating under~\emph{centralized bargaining} agreements, meaning those classified by EUROSTAT as ‘national level or inter-confederal agreements’, ‘industry agreements’ or ‘agreements for individual industries in individual regions’ (also called ‘multi-employer agreements’ or ‘centralized agreements’ in the literature). The second case identifies~\emph{firm-level bargaining} firms, those that apply agreements classified by EUROSTAT as ‘enterprise or single employer agreements’ or ‘agreements applying only to workers in the local unit’, in addition to -- or departing from -- agreements signed at more centralized levels. The third case concerns the lack of any form of collective bargaining at all, which is the norm in market-oriented countries, and may apply to small firms elsewhere too.

Descriptive statistics of the available data (see Table~\ref{tab:summary_stats} and Table~\ref{tab:firm_size} in Appendix~A for details), show that centralized wage bargaining is the dominant form of wage-setting in Belgium, Spain, and France. Firm-level agreements in these countries cover~15--20\% of the employees, and between~6--18\% of firms, while a smaller share of firms and employees are not covered by any collective agreement. In Germany, although a majority of companies~(66--73\%) are not covered by collective agreements, these are mostly small in size, and thus represent less than half of employees sampled. A larger share of employees are covered by centralized agreements, while firm-level agreements are comparatively rare, covering around 6\% or firms, and around 5--7\% of employees. In the Czech Republic and United Kingdom, by contrast, a plurality of firms and employees are not covered by collective bargaining at all, consistent with our description of market-oriented regimes in the previous section. 
Over time, across all countries except the United Kingdom, firm-level agreements covered a larger share of employees in 2010 than in 2006, consistently with the intuition that devolution of bargaining levels became stronger over the observed period.

The focus of this paper is on bargaining decentralization, which involves a shift from coordinated central bargaining to decentralized firm bargaining, and is a separate phenomenon from the complete lack of collective wage agreements. Therefore, in our empirical analysis we only consider firms that do apply some form of collective bargaining. We define a variable $\mathit{FLB}$ which compare between the firms that apply firm-level bargaining ($\mathit{FLB}=1$) to those that only apply centralized bargaining ($\mathit{FLB}=0$).
This is also the most meaningful comparison to make across countries, considering that in Belgium, Spain, and France, nearly all employees are covered by some form of collective agreement.


\subsection*{Definition of the within-firm inequality metric}
\label{sec: inequality}
To build a measure of within-firm wage inequality, we start from the data on hourly compensation of employees reported in SES. However, using these figures directly to compute a measure of wage variability across the employees of the same firm would provide an incorrect estimate of wage inequality, because it would not account for the different workforce composition and corporate characteristics across firms. In fact, in line with an established practice in the literature on wage inequalities~\citep[dating at least since][]{winter.ebmer.1999}, a meaningful comparison of wages across individuals requires to isolate the component of individual wage that is not directly related to the average market compensation of job, personal and other characteristics of otherwise similar individuals.

As our measure of within-firm inequality for each firm $j$, we take the gap between the 90th and 10th percentile log-wage premia of the employees of the firm
\begin{equation}
  \label{eq:def_disp_90_10}
  \Delta w^{90/10}_j=\hat{w}^{90}_j - \hat{w}^{10}_j
\end{equation}
\noindent where $\hat{w}^p_j$ is the $p$-th percentile of the distribution of the residual wage $\hat{w}_{ij}$ obtained for each employee~$i$ of firm~$j$ from the following Mincer-type regression, estimated separately by country and by year:
\begin{equation}
\label{eq:mincer}
\begin{split}
\log \left(W_{ij} \right)
&= b_0 + b_2\,\mathit{tenure}_{ij} + b_3\,\mathit{tenure}_{ij}^2 + b_1\,\mathit{age}_{ij} + b_4\,\mathit{sex}_{ij} + b_5\,\mathit{educ}_{ij} \\
&+ b_6\,\mathit{contract}_{ij} + b_7\,\mathit{part\_time}_{ij} + b_8\,\mathit{occup}_{ij} + b_9\,\mathit{share}_{ij} \\
&= + \phi \,\mathit{FE}_j + w_{ij}
\end{split}
\end{equation}

In this Mincer equation, $\log \left(W_{ij} \right)$ is the logarithm of the hourly wage as reported in SES, which is regressed against a standard set of individual and job characteristics of employees (years of tenure, age, gender, ISCED class of education level, type of contract --permanent, temporary, or apprenticeship--, a dummy for part-time contract, ISCO classes of occupations, and the share of full-time working hours) plus a firm fixed-effect $\mathit{FE}_j$. 

Hence, the residual $w_{ij}$ is a wage-premium that captures the deviation of individual-specific wage from the average wage that could be expected for an employee of firm $j$ based on her characteristics and also controlling for firm-specific average wage premium paid in firm $j$, captured by firm fixed-effect $\mathit{FE}_j$. For example, if a firm had a policy of paying exactly a 10\% premium on average market wages to its employees, this firm-level premium would be accounted for by the coefficient~$\phi$, and this firm's wage policy would have no net effect on within-firm wage inequality in Equation~\eqref{eq:def_disp_90_10}, allowing for meaningful comparison across individuals and firms. 


\section*{Empirical models and estimation strategy}
\label{sec:empirics}
We here present details of the regression models and the estimation strategy that we apply to examine how firm-level bargaining impact within-firm inequality in the different institutional contexts and over time.

\subsection{Firm-level bargaining and within-firm inequality}
Our main aim is to estimate the relation between firm-level bargaining and within-firm inequality, separately by country and over time. We specify the following regression model for each country
\begin{equation}
\label{eq:reg_dispersion}
\begin{split}
  \Delta w^{90/10}_j &= \beta_0 + \beta_1\, \mathit{FLB}_j + \beta_2\, \mathit{Y}^{2010}_j + \beta_3\, \mathit{Y}^{2010}_j \times \mathit{FLB}_j + \gamma\, \widehat{\mathit{FLB}}_j \\
                     &+ \zeta\mathbf{X}_j + \eta\, \mathit{sector}_j + \theta\, \mathit{region}_j + \epsilon_j \;,
\end{split}
\end{equation}

\noindent where $\Delta w_j^{90/10}$ is the measure of within-firm wage inequality defined in Equation~\eqref{eq:def_disp_90_10}, computed for each firm $j$ which is sampled either in year $t=2006$ or $t=2010$;
$\mathit{FLB}_j$ indicates if firm~$j$ applies only centralized bargaining ($\mathit{FLB}_j=0$), or firm-level bargaining $\mathit{FLB}_j=1$;
$\mathit{Y}^{2010}_j$ is a dummy indicating if the firm $j$ is sampled in the year 2010; $\mathbf{X}_j$ is a set of firm characteristics and workforce composition (discussed below); $\mathit{sector}_j$ and $\mathit{region}_j$ are fixed-effects, for economic sector (reported in SES at 1-digit NACE) and geographical location (reported in SES at NUTS-1 level) of the firm; $\widehat{\mathit{FLB}}_j$ is a propensity score representing the probability that firm $j$ adopts firm-level bargaining, included to correct for potential endogenous selection effects; $\epsilon_j$ is an idiosyncratic error term. 

There are four parameters of main interest $\beta$. The intercept $\beta_0$ measures the baseline level of within-firm inequality, among firms under centralized bargaining that were sampled in 2006. The coefficient $\beta_1$ captures the gap in within-firm inequality between firm-level bargaining firms and fully centralized bargaining firms in 2006. Then, $\beta_2$ measures the change in baseline inequality that occurred over time between 2006 and 2010 for firms under centralized bargaining, while $\beta_3$ captures the additional growth in inequality between 2006 and 2010 for firms under firm-level bargaining. Separate estimates of the model (and thus of these four key parameters) by country allow to properly account for differences in bargaining systems across countries. In fact, as discussed above in introducing the FLB dummy, the definition of firm-level bargaining firms ($\mathit{FLB}=1$) is relatively homogeneous in SES across all countries, while there is greater variation across countries about the prevailing alternative bargaining system for the control group of firms that do not apply firm-level bargaining ($\mathit{FLB}=0$).

The identification of the key parameters proceed as follows. First of all, the inclusion of sector and regional fixed-effects, together with firm level controls in $\mathbf{X}_j$, accounts for factors that jointly determine inequality and adoption of firm-level bargaining, potentially creating an omitted variables bias if not included in the regression model. The vector $\mathbf{X}_j$ covers in particular two groups of variables available from SES for each firm $j$. The first are corporate characteristics: a size class (by number of employees), and a dummy for private vs.\ public control on the firm. The expectation is that within-firm wage dispersion is lower in large and publicly owned firms, as the unions tend to be more powerful in these contexts~\citep{canaldominguez.gutierrez.2004}. The second group covers the workforce characteristics of firm $j$, highlighted in previous studies as determinants of wage inequality. For every firm, we measure the share of women employed in the firm; a set of dummies for modal age of the workforce; the share of employees with secondary or tertiary education; the mean tenure of workers in the firm; the share of managers and professionals (according to 1-digit ISCO codes 1 and 2); the share of part-time employees; and the share of employees with a permanent contract. 

Although these controls are relevant in theory, their individual relationship with within-firm inequality is difficult to predict in isolation. Usually, within-firm wage differences are expected to rise with age, tenure and education, because wages tend to increase in all these characteristics, and dispersion is usually higher in firms where average wages are higher~\citep{canaldominguez.gutierrez.2004}. As for gender, the well-documented existence of female wage-gaps would predict wider inequality in firms where the proportion of women is lower. Also, earnings inequalities are expected to be lower in firms having a relatively larger proportion of full-time (vis-à-vis fixed-term), permanent (vis-à-vis part-time), blue-collar (vis-à-vis white-collar) workers, since these types of employees are generally more likely to unionize, and thus their firms to be more affected by unions' efforts to push for equalization of wages among members~\citep{canaldominguez.gutierrez.2004}.\footnote{Basic descriptive statistics on control variables are presented in Appendix~A. Notice that some of the controls are not available for the Czech Republic. First, in the data there are no Czech firms with modal employees' age in the range 20-29 years old, so we omit this age category. Second, the Czech Republic defines a single NUTS-1 region, so we cannot further exploit regional dummies in the estimates for this country.}

In addition to including fixed-effects and firm-level controls, in estimating Equation~\ref{eq:reg_dispersion} we also address the potential endogeneity of the FLB dummy that may arise due to non-random selection of firms that apply firm-level bargaining or not. Indeed, there might still be unobserved determinants of the decision to apply firm-level collective agreements that correlate with unobserved determinants of the dependent variable of interest. This may occur despite controlling for employer-specific components of wages and firm-level average wages through the preliminary Mincerian regression, and despite the inclusion of the extensive set of firm-level covariates. Following a solution adopted in the empirical literature~\citep{card.delarica.2006,daouli.demoussis.ea.2013}, we tackle this possible source of bias by augmenting the model with a preliminary Probit estimate of the probability (propensity score) that a given firm adopts firm-level collective bargaining. The overall rationale is that if FLB status is as good as randomly assigned conditional on observed controls, then conditioning also upon the propensity scores allows to clean any further bias due to unobserved firm characteristics (see Appendix~B for details).

\subsection*{Firm-level bargaining and top vs.~bottom wage premia}
To better understand the estimates derived from Equation~\eqref{eq:reg_dispersion}, we also explore the relation between firm-level bargaining and the two components of the wage-dispersion measure $\Delta w^{90/10}$. We estimate the following variation of Equation~\eqref{eq:reg_dispersion}:

\begin{equation}
\label{eq:reg_decomposition}
\begin{split}
  \Delta w^{p}_j &= \beta_0 + \beta_1\, \mathit{FLB}_j + \beta_2\, \mathit{Y}^{2010}_j + \beta_3\, \mathit{Y}^{2010}_j \times \mathit{FLB}_j + \gamma\, \widehat{\mathit{FLB}}_j \\
                     &+ \zeta\mathbf{X}_j + \eta\, \mathit{sector}_j + \theta\, \mathit{region}_j  + \epsilon_j \;,
 \end{split}
\end{equation}
\noindent where as dependent variable $w^p_j$ we employ, alternatively, the 90th or the 10th percentile of the distribution of log-wage residuals in firm $j$, estimated as earlier described in Equation~\eqref{eq:mincer}.

By separating the effects of firm-level bargaining on the components $w^{90}$ and $w^{10}$, this estimation provides hints about where the overall effects observed in Equation~\eqref{eq:reg_dispersion} stem from. In fact, suppose that a positive association between firm-level bargaining and $\Delta w^{90/10}$ is found for a given country from estimates of Equation~\eqref{eq:reg_dispersion}. Then, it would be interesting to understand if this stems from wage premia of top-paid employees ($w^{90}$) being higher under firm-level bargaining than in other firms, or from wage premia of bottom-paid workers ($w^{10}$) being lower in FLB firms, or from both effects being in place at the same time. Also, in case no statistical difference in overall inequality $\Delta w^{90/10}$ emerged between firm-level bargaining and other firms, Equation~\eqref{eq:reg_decomposition} could tell us if this is due to the two components being offset in the same direction under firm-level bargaining. 

Estimation of Equation~\eqref{eq:reg_decomposition} follows the same strategy employed to estimate the baseline model in~Equation~\eqref{eq:reg_dispersion}. We perform separate regressions by country, augmented with the same set of firm-level covariates and fixed-effects, plus preliminary first-step Probit estimates of firm-specific FLB propensity scores. The estimates of $\beta_1$ on the $\mathit{FLB}$ dummy give the difference in average outcomes across firm-level bargaining vs.~other firms in 2006, whereas the coefficient $\beta_3$ on the interaction term $\mathit{FLB} \times Y^{2010}$ accounts for changes in the FLB effect over time.


\section*{Results}
\label{sec:results}
Table~\ref{tab:disp_90_10} present the estimates of Equation~\eqref{eq:reg_dispersion}, looking at the difference in overall within-firm wage inequality between firms that apply firm-level bargaining and centralized bargaining.

\begin{center}
-- Table~~\ref{tab:disp_90_10} about here --
\end{center}

There is a substantial difference in the baseline level of inequality across countries, captured by the intercept $\beta_0$, which estimates the average within-firm difference in between 90th and 10th percentile of log-wage premia for companies under centralized bargaining in 2006. The lowest level of inequality is in Spain, where the baseline difference is estimated at $0.299$, or approximately 30 percentage points. France has nearly double the highest baseline inequality, of around 62 percentage points, while most other countries are around 45--55 percentage points.

In 2006, there was mostly no difference in inequality between firms that adopted firm-level bargaining and those under centralized bargaining. Indeed, the estimated  coefficient $\beta_1$ are not statistically different from zero in all countries, except in the UK. In this case, firm-level bargaining firms show a statistically significant less unequal wage distribution compared to other firms, by about 1.3 percentage points, or around 2\% of baseline inequality.

By 2010, within-firm wage inequality in firms under centralized bargaining dropped compared to 2006 (cf.~$\beta_2$), in all countries except Germany. For some countries, the change was quite sizeable, compared to baseline inequality in 2006: around 11\% in Spain, and 13\% in the UK. While this trend of reduction was equally characterising firm-level bargaining and other firms in most countries, in Spain and France this reduction in inequality was instead dampened (or reversed) in firms under firm-level bargaining (cf. $\beta_3$). In both countries, whereas in 2006 there was no significant difference between companies that applied centralized and firm-level bargaining, in 2010 there is a widening gap (around 2.2 percentage points in Spain, 3.6 in France). On balance, while within-firm inequality reduced appreciably in 2010 for Spanish and French firms under centralized bargaining, it narrowed substantially less for companies under firm-level bargaining in Spain (summing $\beta_2$ and $\beta_3$) and even increased for those in France. 

The estimates reveal heterogeneity also with regard to the correlation between the $\Delta w^{90/10}$ wage-gap and control variables. Starting from corporate characteristics, wage dispersion within firms increases with firm size in Germany, Spain and France, but larger firms display lower wage dispersion than the baseline in the UK. Publicly-controlled firms feature lower wage dispersion compared to private firms in Belgium, the Czech Republic and France. Further, moving to workforce characteristics, the modal age of employees shows mostly an insignificant association with wage inequality in Belgium, Germany, Spain and France, while the relation with $\Delta w^{90/10}$ is stronger (positive) in the Czech Republic and the UK. A common result across all countries is that within-firm wage inequality is larger for firms with the most senior workforce (60+ years old). The share of women in the workforce, the average on-the-job tenure of employees and the share of permanent contracts show a negative association with within-firm wage dispersion in most countries, while educational levels, the share of part-time employees and the share of higher professional occupations tend to display a positive association (when significant) with within-firm wage dispersion. Note, lastly, that the significant coefficient on the propensity score $\widehat{\mathit{FLB}}_j$ confirms the need to correct for endogenous selection into $\mathit{FLB}$ in most countries.

\begin{center}
-- Table~~\ref{tab:disp_avg_90_10} about here --
\end{center}


Estimates of the separate effects of firm-level bargaining on the 90th and 10th percentiles of wage premia, in Table~\ref{tab:disp_avg_90_10}, are particularly revealing of the dynamics underlying the results observed above for UK, France and Spain. The lower inequality for firm-level bargaining firms emerged in 2006 for the UK appears as due to sensibly higher wage premia paid at the bottom of internal wage distribution in firm-level bargaining firms compared to centralized bargaining firms (see $\beta_1$ estimated for $q_{10}$). In Spain, the reduction of overall wage inequality $\Delta w^{90/10}$ in 2010 observed earlier for firms under centralized bargaining, results from a relative reduction of the 90th percentile wage premia and a corresponding increase of those at the 10th percentile, leading to overall wage reduction by both extremes moving closer.
Crucially, this reduction in inequality was dampened for both $q_{90}$ and $q_{10}$ under firm-level bargaining (see coefficient $\beta_3$), leading to a total 2.17 percentage point difference between the two regimes in 2010. France experienced a more extreme version of the same dynamics: the wage premia at the top increased and those at the bottom decreased under FLB, leading to an overall difference of 3.6 percentage points.

\section*{Conclusion}
\label{sec:conclusion}
The impact of collective pay agreements on wage inequality between firms is well-documented. However, there is less evidence on whether wage-setting happening at the level of firms -- as opposed to more centralized bargaining levels -- can explain wage differences emerging within the firm. 

Exploiting matched employer-employee data for six European countries over 2006 and 2010, in this work we contribute to advance the existing literature by addressing three questions. First, is firm-level bargaining generally associated to higher or lower within-firm inequality? Second, have these relations remained stable or changed over the years under study, when a broad process of increasing emphasis on decentralization of wage bargaining took place and the Great Depression hit? Third, are there patterns common to all or at least to some of the selected countries, mapping into broad bargaining regimes?

A-priori, effects are difficult to predict, because of countervailing trends. On the one hand, by allowing firms more flexibility than higher level negotiations, firm-level agreements may induce an increase in within-firm inequality, if they are used to selectively provide incentives or rewards to specific employees or groups of employees. On the other hand, firm-level agreements may reduce inequalities within firms if they respond to fairness, egalitarian or redistributive motives. The balance between these contrasting forces may, in turn, depend on the institutional context, according to the changing scope of firm-level bargaining in different national frameworks and over time.

Considering the specific features of wage bargaining systems in place in the countries under study over the decade covered by our data, we could have expected higher inequality under firm-level bargaining in the UK and the Czech Republic, consistently with the wage setting in place in these countries, traditionally favouring selective use of firm-level pay schemes. In other countries, the national context made us predict that in Spain and Germany firm-level would be associated with less inequality, but our predictions were overall more uncertain, due to a non-trivial combination of country-specific and regime-specific factors typical of countries like Belgium, Germany, Spain and France. In fact, firm-level bargaining could increase within-firm inequality, to the extent that firm-level negotiations are used to gain flexibility over the complexity of negotiations and standardization of wages typical of these countries' regimes. But the same egalitarian pressures could be so strong as to justify the hypothesis that firm-level bargaining reduced inequality or had no effect in those countries. Lastly, we also put forward the hypothesis that, if anything, the inequality-enhancing effects of firm-level bargaining could have been stronger in 2010 in all countries, in line with a general trend toward more and more devolution of bargaining levels occurring over time.

Our results only partially match with these predictions, however. First, they confirm that the process of devolution of bargaining level started in all countries in the 90s did not go so far as to completely blur cross-country differences in the use and effect of firm-level negotiations. We indeed find that firms bargaining locally may have similar levels of inequality than those under centralized bargaining, but may also have higher or lower inequality as well. This can change over time, even within the same country. In 2006, firms bargaining locally were less unequal than under centralized bargaining in the UK, while no difference emerged in the other countries. Decomposition of wage premia reveals that the result for the UK is driven by low-paid workers being paid more in firm-level bargaining firms than under centralized bargaining. By 2010, it is only in Spain and France that we observe a divergence in the trend of inequality between firms under centralized and firm-level bargaining firms, with the latter ending up as more unequal in 2010. In both countries, while the top wage premia dropped and bottom wage premia rose in companies under centralized bargaining, this movement either didn't occur or was reversed under firm-level bargaining, suggesting that firm-level negotiations were increasingly used to escape standardization of wages in these two countries. 

At the same time, the heterogeneity of effects estimated across countries do not map neatly into country-specific features of national bargaining systems, or into broad classification of countries based on prevailing bargaining levels. On the one hand, we could have anticipated the insignificant results obtained for Belgium, and also the significant effects estimated for the UK, Spain and France. However, sets of countries that shared similar institutional characteristics turned out to be different: the UK versus the Czech Republic, or Germany and Belgium versus France and Spain. And, conversely, the same results emerged for countries with a-priori quite diverse collective bargaining institutions and traditions, like Germany and the Czech-Republic. Ultimately, our findings provide an average picture about how countervailing firm-specific drivers of the use of firm-level negotiations, such as incentive motives, inertia and conflicts of power, are resolved. While more detailed data on these factors and perhaps specific case study of single firms could help highlighting the underlying dynamics, our estimates support that these processes do not systematically relate to a peculiar prevailing regime. Actually, a possible avenue for future research, complementary to our study, would be to investigate the existence of common patterns in the use of firm-level bargaining across sectors, instead of across countries. If international sectoral patterns mattered more than country-specific ones~\citep[as suggested in ][]{bechter2012sectors}, we should envisage different uses -- more or less inequality-enhancing -- of firm-level contracts in some sectors than others, across countries. Although we control for this possible trend via sector fixed-effects, direct tackling this hypothesis would lead to clustering the analysis by sectors instead of by countries. 

Further, the availability of matched employer-employee panel data --following the same workers and firms continuously for many years-- would allow to evaluate more precisely if inequality-enhancing vs.~redistributive uses of firm-level agreements are more likely during upswing or downswing of the business cycle, giving firms more flexibility and discretionality to reshape the wage ladder in case of need. This appears to be even more relevant in more recent years in a context of rapidly changing work, where firms are reshaping their organisational structures in relation to processes of digitalization~\citep{OECD2019} and the locus of bargaining is likely to affect how productivity gains will be distributed within-firms.

Overall, our study offers new evidence and methods to inform the renewed debate on the determinants of wage inequality. We highlight the importance of the locus of collective wage bargaining and show that firm-level bargaining can act not only as a driver of wage inequality between firms, but also within them.

\clearpage
\theendnotes

\begin{thebibliography}{38}
\providecommand{\natexlab}[1]{#1}
\providecommand{\url}[1]{\texttt{#1}}
\providecommand{\urlprefix}{URL }
\expandafter\ifx\csname urlstyle\endcsname\relax
  \providecommand{\doi}[1]{DOI:\discretionary{}{}{}#1}\else
  \providecommand{\doi}{DOI:\discretionary{}{}{}\begingroup
  \urlstyle{rm}\Url}\fi

\bibitem[{Akerlof(1984)}]{akerlof.1984}
Akerlof GA (1984) Gift exchange and efficiency-wage theory: Four views.
\newblock \emph{The American Economic Review} 74(2): 79--83.

\bibitem[{Baccaro and Howell(2017)}]{baccaro2017trajectories}
Baccaro L and Howell C (2017) \emph{Trajectories of neoliberal transformation:
  European industrial relations since the 1970s}.
\newblock Cambridge University Press.

\bibitem[{Bayo-Moriones et~al.(2013)Bayo-Moriones, Galdon-Sanchez and
  Martinez-de Morentin}]{bayo2013diffusion}
Bayo-Moriones A, Galdon-Sanchez JE and Martinez-de Morentin S (2013) The
  diffusion of pay for performance across occupations.
\newblock \emph{Industrial \& Labor Relations} 66(5): 1115--1148.

\bibitem[{Bechter et~al.(2012)Bechter, Brandl and Meardi}]{bechter2012sectors}
Bechter B, Brandl B and Meardi G (2012) Sectors or countries? typologies and
  levels of analysis in comparative industrial relations.
\newblock \emph{European Journal of Industrial Relations} 18(3): 185--202.

\bibitem[{Berlingieri et~al.(2017)Berlingieri, Blanchenay and
  Criscuolo}]{Berlingieri2017}
Berlingieri G, Blanchenay P and Criscuolo C (2017) The great divergence(s).
\newblock \emph{OECD Science, Technology and Industry Policy Papers}
  \doi{10.1787/953f3853-en}.
\newblock
  \urlprefix\url{oecd-ilibrary.org/content/paper/953f3853-en}.

\bibitem[{Blau and Duncan(1967)}]{blau1967american}
Blau PM and Duncan OD (1967) \emph{The American occupational structure}.
\newblock John Wiley \& Sons, NY.

\bibitem[{Braakmann and Brandl(2016)}]{Braakmann}
Braakmann N and Brandl B (2016) {The Efficacy of Hybrid Collective Bargaining
  Systems: An Analysis of the Impact of Collective Bargaining on Company
  Performance in Europe}.
\newblock MPRA Paper 70025, University Library of Munich, Germany.

\bibitem[{Canal~Dominguez and Gutierrez(2004)}]{canaldominguez.gutierrez.2004}
Canal~Dominguez JF and Gutierrez CR (2004) Collective bargaining and
  within-firm wage dispersion in spain.
\newblock \emph{British Journal of Industrial Relations} 42(3): 481--506.

\bibitem[{Card and De~La~Rica(2006)}]{card.delarica.2006}
Card D and De~La~Rica S (2006) Firm-level contracting and the structure of
  wages in spain.
\newblock \emph{Industrial \& Labor Relations Review} 59(4): 573--592.

\bibitem[{Cobb(2016)}]{cobb2016}
Cobb JA (2016) How firms shape income inequality: Stakeholder power, executive
  decision making, and the structuring of employment relationships.
\newblock \emph{Academy of Management Review} 41(2): 324--348.

\bibitem[{Dahl et~al.(2013)Dahl, Le~Maire and Munch}]{dahl.lemaire.ea.2013}
Dahl CM, Le~Maire D and Munch JR (2013) Wage dispersion and decentralization of
  wage bargaining.
\newblock \emph{Journal of Labor Economics} 31(3): 501--533.

\bibitem[{Daouli et~al.(2013)Daouli, Demoussis, Giannakopoulos and
  Laliotis}]{daouli.demoussis.ea.2013}
Daouli J, Demoussis M, Giannakopoulos N and Laliotis I (2013) Firm-level
  collective bargaining and wages in greece: A quantile decomposition analysis.
\newblock \emph{British Journal of Industrial Relations} 51(1): 80--103.

\bibitem[{{Dell'Aringa} and Lucifora(1994)}]{dellaringa.lucifora.1994}
{Dell'Aringa} C and Lucifora C (1994) Wage dispersion and unionism: Do unions
  protect low pay.
\newblock \emph{International Journal of Manpower} 15: 150--170.

\bibitem[{{Dell'Aringa} et~al.(2004){Dell'Aringa}, Lucifora, Orlando and
  Cottini}]{dellaringa.lucifora.ea.2004}
{Dell'Aringa} C, Lucifora C, Orlando N and Cottini E (2004) Bargaining
  structure and intra-establishment pay inequality in four european countries:
  Evidence from matched employer-employee data.
\newblock Technical report, CEP-LSE.

\bibitem[{{Dell'Aringa} and Pagani(2007)}]{dellaringa.pagani.2007}
{Dell'Aringa} C and Pagani L (2007) Collective bargaining and wage dispersion
  in europe.
\newblock \emph{British Journal of Industrial Relations} 45(1): 29--54.

\bibitem[{Eisenhardt(1989)}]{eisenhardt1989agency}
Eisenhardt KM (1989) Agency theory: An assessment and review.
\newblock \emph{Academy of management review} 14(1): 57--74.

\bibitem[{Erikson and Goldthorpe(2002)}]{erikson2002intergenerational}
Erikson R and Goldthorpe JH (2002) Intergenerational inequality: A sociological
  perspective.
\newblock \emph{Journal of Economic Perspectives} 16(3): 31--44.

\bibitem[{Fournier and Koske(2013)}]{fournier.koske.2013}
Fournier JM and Koske I (2013) The determinants of earnings inequality.
\newblock \emph{OECD Journal: Economic Studies} 2012(1): 7--36.

\bibitem[{Fulton(2013)}]{fulton.2013}
Fulton L (2013) Worker representation in Europe.
\newblock Labour Research Department and European Trade Union Institute (ETUI).
\newblock \urlprefix\url{http://www.worker-participation.eu/National-Industrial-Relations}

\bibitem[{Fulton(2015)}]{fulton.2015}
Fulton L (2015) Worker representation in Europe.
\newblock Labour Research Department and European Trade Union Institute (ETUI).
\newblock \urlprefix\url{http://www.worker-participation.eu/National-Industrial-Relations}

\bibitem[{Goldthorpe and Hope(1972)}]{goldthorpe1972occupational}
Goldthorpe JH and Hope K (1972) Occupational grading and occupational prestige.
\newblock \emph{Social Science Information} 11(5): 17--73.

\bibitem[{Hassel(2014)}]{hassel2014paradox}
Hassel A (2014) The paradox of liberalization—understanding dualism and the
  recovery of the g erman political economy.
\newblock \emph{British Journal of Industrial Relations} 52(1): 57--81.

\bibitem[{{ILO}(2016)}]{GlobalWageReport}
{ILO} (2016) Global wage report 2016/17: Wage inequality in the workplace.
\newblock {International Labour Office Publications}, Geneva: International
  Labour Organization.

\bibitem[{Keune(2011)}]{keune2011decentralizing}
Keune M (2011) Decentralizing wage setting in times of crisis? the regulation
  and use of wage-related derogation clauses in seven european countries.
\newblock \emph{European Labour Law Journal} 2(1): 86--95.

\bibitem[{Lazear and Rosen(1979)}]{lazear.1979}
Lazear EP and Rosen S (1979) Rank-order tournaments as optimum labor contracts.
\newblock \emph{Journal of Political Economy} 89(5): 841--864.

\bibitem[{Lazear and Shaw(2007)}]{lazear.shaw.2007}
Lazear EP and Shaw KL (2007) Wage structure, raises and mobility: International
  comparisons of the structure of wages within and across firms.
\newblock NBER Working Papers 13654, National Bureau of Economic Research.

\bibitem[{Lindbeck and Snower(1986)}]{lindbeck1986wage}
Lindbeck A and Snower DJ (1986) Wage setting, unemployment, and
  insider-outsider relations.
\newblock \emph{The American Economic Review} 76(2): 235--239.

\bibitem[{Lindbeck and Snower(2001)}]{lindbeck2001insiders}
Lindbeck A and Snower DJ (2001) Insiders versus outsiders.
\newblock \emph{The Journal of Economic Perspectives} 15(1): 165--188.

\bibitem[{OECD(2018)}]{OECD2018}
OECD (2018) \emph{OECD Economic Outlook, Volume 2018 Issue 2}.
\newblock OECD Publishing.
\newblock
  \doi{10.1787/eco\_outlook-v2018-2-en}.
\newblock
  \urlprefix\url{oecd-ilibrary.org/content/publication/eco\_outlook-v2018-2-en}.

\bibitem[{OECD(2019)}]{OECD2019}
OECD (2019) \emph{Negotiating Our Way Up}.
\newblock \doi{10.1787/1fd2da34-en}.
\newblock
  \urlprefix\url{oecd-ilibrary.org/content/publication/1fd2da34-en}.

\bibitem[{O'Shaughnessy(1998)}]{o1998structure}
O'Shaughnessy KC (1998) The structure of white-collar compensation and
  organizational performance.
\newblock \emph{Relations industrielles/Industrial Relations} : 458--485.

\bibitem[{Plasman et~al.(2007)Plasman, Rusinek and
  Rycx}]{plasman.rusinek.ea.2007}
Plasman R, Rusinek M and Rycx F (2007) Wages and the bargaining regime under
  multi-level bargaining: Belgium, denmark and spain.
\newblock \emph{European Journal of Industrial Relations} 13(2): 161--180.

\bibitem[{Stainback et~al.(2010)Stainback, Tomaskovic-Devey and
  Skaggs}]{stainback2010}
Stainback K, Tomaskovic-Devey D and Skaggs S (2010) Organizational approaches
  to inequality: Inertia, relative power, and environments.
\newblock \emph{Annual Review of Sociology} 36.

\bibitem[{Undy(1978)}]{undy1978}
Undy R (1978) {The Devolution of Bargaining Levels and Responsibilities in the
  TCWU, 1965-75}.
\newblock \emph{Industrial Relations Journal} 9(3): 44--56.

\bibitem[{Visser et~al.(2013)}]{visser2013wage}
Visser J et~al. (2013) Wage bargaining institutions--from crisis to crisis.
\newblock Technical report, Directorate General Economic and Financial Affairs
  (DG ECFIN), European Commission.

\bibitem[{Wallerstein et~al.(1997)Wallerstein, Golden and
  Lange}]{wallerstein1997unions}
Wallerstein M, Golden M and Lange P (1997) Unions, employers' associations, and
  wage-setting institutions in northern and central europe, 1950--1992.
\newblock \emph{Industrial \& Labor Relations} 50(3): 379--401.

\bibitem[{Winter-Ebmer and Zweim{\"u}ller(1999)}]{winter.ebmer.1999}
Winter-Ebmer R and Zweim{\"u}ller J (1999) Intra-firm wage dispersion and firm
  performance.
\newblock \emph{Kyklos} 52(4): 555--572.

\bibitem[{Wright(1980)}]{wright1980class}
Wright EO (1980) Class and occupation.
\newblock \emph{Theory and Society} 9(1): 177--214.

\end{thebibliography}

\clearpage

\section*{Regression Tables}
\label{sec:tables}

\begin{table}[hbt]
\caption{90th-10th percentile within-firm inequality and Firm-level bargaining (FLB)}
\label{tab:disp_90_10}
\centering
\resizebox{\textwidth}{!}{
\begin{threeparttable}
\begin{tabular}{l*{6}{l}}
\toprule
& \multicolumn{1}{c}{BE} & \multicolumn{1}{c}{DE} & \multicolumn{1}{c}{ES} & \multicolumn{1}{c}{CZ} & \multicolumn{1}{c}{UK} & \multicolumn{1}{c}{FR} \\
\midrule
$\beta_0$: Intercept                         &       0.455***    & 0.529***    & 0.299***   & 0.463***    & 0.596***   & 0.624***     \\
$\hookrightarrow$ \textit{Base inequality (FLB=0 in 2006)} &       (0.0340)    & (0.0539)    & (0.0259)   & (0.0427)    & (0.122)    & (0.0401)     \\[1ex]
$\beta_1$: FLB                              &       -0.00158    & -0.00285    & 0.00510    & -0.0103     & -0.0129**  & -0.00478     \\
$\hookrightarrow$ \textit{Additional ineq. of FLB=1 in 2006} &       (0.00420)   & (0.00638)   & (0.00458)  & (0.0100)    & (0.00634)  & (0.00990)    \\[1ex]
$\beta_2$: Year 2010                        &       -0.0334***  & 0.00454     & -0.0326*** & -0.0142     & -0.0798*** & -0.0151***   \\
$\hookrightarrow$ \textit{Add. ineq. in 2010}        &       (0.00265)   & (0.00424)   & (0.00273)  & (0.0119)    & (0.00842)  & (0.00348)    \\[1ex]
$\beta_3$: FLB$\times$2010                  &       0.00148     & 0.00323     & 0.0217***  & 0.00496     & -0.00420   & 0.0362***    \\
$\hookrightarrow$ \textit{Add. ineq. of FLB in 2010} &       (0.00592)   & (0.00846)   & (0.00672)  & (0.0129)    & (0.00849)  & (0.0114)     \\[1ex]
$\gamma$: Prob. FLB                         &       0.0953***   & -0.000383   & -0.226***  & 0.117**     & -0.00772   & 0.105***     \\
$\hookrightarrow$ \textit{Add. ineq. of predicted FLB status}                                            &       (0.0355)    & (0.0520)    & (0.0250)   & (0.0458)    & (0.125)    & (0.0359)     \\[1ex]
\cmidrule(lr){1-7}
Modal age workers:                          \\[1ex]
\quad \textit{20-29}                        &       -0.0297     & 0.0330      & 0.0101     &             & 0.00383    & -0.0654*     \\
                                            &       (0.0242)    & (0.0455)    & (0.0242)   &             & (0.0127)   & (0.0369)     \\[1ex]
\quad \textit{30-39}                        &       -0.0133     & 0.0204      & 0.0264     & 0.0363***   & 0.0378***  & -0.0557      \\
                                            &       (0.0244)    & (0.0446)    & (0.0244)   & (0.00985)   & (0.0127)   & (0.0370)     \\[1ex]
\quad \textit{40-49}                        &       -0.00313    & 0.0156      & 0.0177     & 0.0120      & 0.0466***  & -0.0577      \\
                                            &       (0.0242)    & (0.0446)    & (0.0243)   & (0.0119)    & (0.0131)   & (0.0360)     \\[1ex]
\quad \textit{50-59}                        &       0.0197      & 0.0124      & 0.0157     & 0.0152      & 0.0470***  & -0.0389      \\
                                            &       (0.0250)    & (0.0451)    & (0.0248)   & (0.0103)    & (0.0139)   & (0.0366)     \\[1ex]
\quad \textit{60+}                          &       0.0322      & 0.103*      & 0.0785***  & 0.0955**    & 0.0309*    & -0.0315      \\
                                            &       (0.0311)    & (0.0572)    & (0.0284)   & (0.0381)    & (0.0166)   & (0.0404)     \\[1ex]
\% of women empl.                           &       -0.0616***  & -0.0490***  & -0.0392*** & -0.0230     & -0.0386*** & -0.0451***   \\
                                            &       (0.00599)   & (0.0119)    & (0.00467)  & (0.0157)    & (0.00986)  & (0.00603)    \\[1ex]
Mean experience empl.                       &       -0.00216*** & -0.00245*** & 0.00344*** & -0.00468*** & -5.36e-05  & -0.000761*** \\
                                            &       (0.000392)  & (0.000522)  & (0.000360) & (0.000980)  & (0.000576) & (0.000287)   \\[1ex]
\% empl. with tert. educ.                   &       0.116***    & 0.0847***   & 0.165***   & 0.249***    & 0.103***   & 0.0799***    \\
                                            &       (0.00836)   & (0.0272)    & (0.00634)  & (0.0386)    & (0.0159)   & (0.00814)    \\[1ex]
\% empl. with sec. educ.                    &       0.0184***   & 0.0532***   & 0.0730***  & 0.0206      & 0.0592***  & 0.00665      \\
                                            &       (0.00498)   & (0.0185)    & (0.00480)  & (0.0275)    & (0.0134)   & (0.00802)    \\[1ex]
\% managers and profess.                    &       0.0931***   & 0.0688***   & 0.0745***  & 0.121***    & 0.251***   & 0.148***     \\
                                            &       (0.00971)   & (0.0204)    & (0.00949)  & (0.0260)    & (0.0111)   & (0.00853)    \\[1ex]
\% part-time empl.                          &       -0.0108     & 0.140***    & 0.109***   & 0.175***    & 0.0419***  & 0.00672      \\
                                            &       (0.00773)   & (0.0126)    & (0.00664)  & (0.0529)    & (0.0108)   & (0.00832)    \\[1ex]
\% permanent contracts                      &       -0.0781***  & -0.0861***  & -0.00416   & -0.00757    & -0.0464**  & -0.160***    \\
                                            &       (0.0101)    & (0.0184)    & (0.00509)  & (0.0173)    & (0.0190)   & (0.0136)     \\[1ex]

Firm size:                                  \\[1ex]
\quad \textit{50--249 empl.}                &       0.000931    & 0.0361***   & 0.121***   & 0.0163      & -0.0631*** & 0.0378***    \\
                                            &       (0.00562)   & (0.00545)   & (0.00357)  & (0.0108)    & (0.0131)   & (0.00429)    \\[1ex]
\quad \textit{$\geq$ 250 empl.}             &       -0.00323    & 0.0340***   & 0.193***   & 0.0112      & -0.0652*** & 0.0490***    \\
                                            &       (0.00967)   & (0.00513)   & (0.00641)  & (0.0164)    & (0.0117)   & (0.00490)    \\[1ex]
Public firm                                 &       -0.0502***  & -0.00360    & 0.0287     & -0.0797***  & 0.0194     & -0.0694***   \\
                                            &       (0.0119)    & (0.0140)    & (0.00823)  & (0.0116)    & (0.0598)   & (0.00937)    \\[1ex]
\midrule
Observations                                &       13,765      & 12,312      & 37,887     & 3,498       & 14,502     & 30,009       \\
R-squared                                   &       0.187       & 0.064       & 0.197      & 0.230       & 0.123      & 0.118        \\
Region FE                                   &       \checkmark  & \checkmark  & \checkmark & \checkmark  & \checkmark & \checkmark   \\
Sector FE                                   &       \checkmark  & \checkmark  & \checkmark & \checkmark  & \checkmark & \checkmark   \\
\bottomrule \end{tabular}
\begin{tablenotes}
\item Notes: Bootsrapped standard errors in parentheses (200 repetitions);
\item Asterisks denote significance levels: $^{*}$ p$<$0.05, $^{**}$ p$<$0.01, $^{***}$ p$<$0.001
\end{tablenotes}
\end{threeparttable}
}
 \end{table}

\clearpage


\begin{sidewaystable}[htb]
\caption{Decomposition 90th and 10th percentiles and Firm-level bargaining (FLB)}
\label{tab:disp_avg_90_10}
\centering
\centering
\resizebox{\textwidth-2cm}{!}{\begin{threeparttable}
\begin{tabular}{l*{12}{l}}
\toprule
& \multicolumn{2}{c}{BE} & \multicolumn{2}{c}{DE} & \multicolumn{2}{c}{ES} & \multicolumn{2}{c}{CZ} & \multicolumn{2}{c}{UK} & \multicolumn{2}{c}{FR} \\
 & \multicolumn{1}{c}{q\_90} & \multicolumn{1}{c}{q\_10} & \multicolumn{1}{c}{q\_90} & \multicolumn{1}{c}{q\_10} & \multicolumn{1}{c}{q\_90} & \multicolumn{1}{c}{q\_10} & \multicolumn{1}{c}{q\_90} & \multicolumn{1}{c}{q\_10} & \multicolumn{1}{c}{q\_90} & \multicolumn{1}{c}{q\_10} & \multicolumn{1}{c}{q\_90} & \multicolumn{1}{c}{q\_10} \\

\midrule
$\beta_0$: Intercept             &       0.226***    & -0.228***   & 0.270***     & -0.259***  & 0.148***   & -0.151***   & 0.244***    & -0.219***  & 0.307***   & -0.290***  & 0.315***     & -0.309***  \\
                                &       (0.0203)    & (0.0173)    & (0.0289)     & (0.0265)   & (0.0134)   & (0.0112)    & (0.0223)    & (0.0195)   & (0.0722)   & (0.0571)   & (0.0234)     & (0.0173)   \\[1ex]
$\beta_1$: FLB                  &       -0.00106    & 0.000520    & -0.00945***  & -0.00660*  & -0.000250  & -0.00535**  & -0.00479    & 0.00548    & -0.00414   & 0.00877*** & -0.00187     & 0.00292    \\
                                &       (0.00276)   & (0.00244)   & (0.00343)    & (0.00394)  & (0.00260)  & (0.00239)   & (0.00521)   & (0.00535)  & (0.00373)  & (0.00320)  & (0.00500)    & (0.00601)  \\[1ex]
$\beta_2$: Year 2010            &       -0.0189***  & 0.0145***   & 0.00678***   & 0.00223    & -0.0174*** & 0.0151***   & -0.00548    & 0.00871    & -0.0392*** & 0.0406***  & -0.00764***  & 0.00741*** \\
                                &       (0.00158)   & (0.00135)   & (0.00236)    & (0.00262)  & (0.00139)  & (0.00131)   & (0.00623)   & (0.00728)  & (0.00530)  & (0.00419)  & (0.00202)    & (0.00162)  \\[1ex]
$\beta_3$: FLB$\times$2010      &       0.00159     & 0.000108    & 0.00578      & 0.00255    & 0.0148***  & -0.00693**  & 0.00125     & -0.00371   & -0.00431   & -0.000113  & 0.0195***    & -0.0167*** \\
                                &       (0.00332)   & (0.00298)   & (0.00451)    & (0.00452)  & (0.00346)  & (0.00316)   & (0.00693)   & (0.00771)  & (0.00533)  & (0.00443)  & (0.00600)    & (0.00598)  \\[1ex]
$\gamma$: Prob. FLB              &       0.0539***   & -0.0415**   & -0.0153      & -0.0149    & -0.133***  & 0.0928***   & 0.0595**    & -0.0576*** & -0.0112    & -0.00350   & 0.0678***    & -0.0374**  \\
                                &       (0.0173)    & (0.0171)    & (0.0238)     & (0.0296)   & (0.0151)   & (0.0129)    & (0.0259)    & (0.0212)   & (0.0739)   & (0.0573)   & (0.0183)     & (0.0167)   \\[1ex]
\cmidrule(lr){1-13}
Modal age workers:             \\[1ex]
\quad \textit{20-29}            &       -0.0141     & 0.0156      & -0.00923     & -0.0423*   & 0.00651    & -0.00356    &             &            & 0.00736    & 0.00353    & -0.0306      & 0.0348**   \\
                                &       (0.0136)    & (0.0117)    & (0.0251)     & (0.0244)   & (0.0131)   & (0.0109)    &             &            & (0.00635)  & (0.00660)  & (0.0211)     & (0.0165)   \\[1ex]
\quad \textit{30-39}            &       -0.00464    & 0.00867     & -0.0135      & -0.0339    & 0.0153     & -0.0111     & 0.0183***   & -0.0180*** & 0.0267***  & -0.0111*   & -0.0250      & 0.0308*    \\
                                &       (0.0138)    & (0.0115)    & (0.0250)     & (0.0239)   & (0.0129)   & (0.0108)    & (0.00523)   & (0.00603)  & (0.00684)  & (0.00663)  & (0.0209)     & (0.0166)   \\[1ex]
\quad \textit{40-49}            &       0.00105     & 0.00418     & -0.0175      & -0.0331    & 0.0108     & -0.00687    & 0.00668     & -0.00533   & 0.0315***  & -0.0151**  & -0.0249      & 0.0329**   \\
                                &       (0.0139)    & (0.0116)    & (0.0249)     & (0.0238)   & (0.0131)   & (0.0108)    & (0.00628)   & (0.00640)  & (0.00635)  & (0.00674)  & (0.0208)     & (0.0166)   \\[1ex]
\quad \textit{50-59}            &       0.0127      & -0.00702    & -0.0167      & -0.0291    & 0.0101     & -0.00561    & 0.00820     & -0.00701   & 0.0332***  & -0.0138*   & -0.0146      & 0.0243     \\
                                &       (0.0141)    & (0.0118)    & (0.0252)     & (0.0237)   & (0.0133)   & (0.0110)    & (0.00538)   & (0.00583)  & (0.00681)  & (0.00723)  & (0.0210)     & (0.0164)   \\[1ex]
\quad \textit{60+}              &       0.0193      & -0.0129     & 0.0232       & -0.0798**  & 0.0437***  & -0.0348***  & 0.0538***   & -0.0417**  & 0.0233***  & -0.00762   & -0.0108      & 0.0206     \\
                                &       (0.0175)    & (0.0160)    & (0.0317)     & (0.0325)   & (0.0147)   & (0.0128)    & (0.0184)    & (0.0176)   & (0.00895)  & (0.00815)  & (0.0224)     & (0.0186)   \\[1ex]
\% of women empl.               &       -0.0294***  & 0.0323***   & -0.0203***   & 0.0287***  & -0.0176*** & 0.0216***   & -0.00670    & 0.0163**   & -0.0215*** & 0.0171***  & -0.0255***   & 0.0196***  \\
                                &       (0.00340)   & (0.00323)   & (0.00576)    & (0.00646)  & (0.00280)  & (0.00247)   & (0.00815)   & (0.00750)  & (0.00520)  & (0.00427)  & (0.00333)    & (0.00321)  \\[1ex]
Mean experience empl.           &       -0.00125*** & 0.000907*** & -0.000873*** & 0.00158*** & 0.00188*** & -0.00156*** & -0.00276*** & 0.00192*** & -3.41e-05  & 1.95e-05   & -0.000512*** & 0.000249*  \\
                                &       (0.000228)  & (0.000190)  & (0.000266)   & (0.000301) & (0.000215) & (0.000176)  & (0.000548)  & (0.000481) & (0.000359) & (0.000287) & (0.000158)   & (0.000144) \\[1ex]
\% empl. with tert. educ.       &       0.0587***   & -0.0577***  & 0.0597***    & -0.0251    & 0.0844***  & -0.0804***  & 0.130***    & -0.119***  & 0.0494***  & -0.0531*** & 0.0369***    & -0.0430*** \\
                                &       (0.00431)   & (0.00406)   & (0.0134)     & (0.0169)   & (0.00319)  & (0.00306)   & (0.0202)    & (0.0193)   & (0.00939)  & (0.00736)  & (0.00473)    & (0.00388)  \\[1ex]
\% empl. with sec. educ.        &       0.00871***  & -0.00971*** & 0.0322***    & -0.0210*   & 0.0384***  & -0.0346***  & -0.00260    & -0.0232    & 0.0264***  & -0.0328*** & -0.000259    & -0.00690*  \\
                                &       (0.00272)   & (0.00263)   & (0.00874)    & (0.0114)   & (0.00267)  & (0.00239)   & (0.0150)    & (0.0145)   & (0.00822)  & (0.00699)  & (0.00400)    & (0.00370)  \\[1ex]
\% managers and profess.        &       0.0438***   & -0.0493***  & 0.0319***    & -0.0369*** & 0.0382***  & -0.0364***  & 0.0571***   & -0.0637*** & 0.125***   & -0.126***  & 0.0820***    & -0.0660*** \\
                                &       (0.00600)   & (0.00515)   & (0.00957)    & (0.0108)   & (0.00466)  & (0.00488)   & (0.0166)    & (0.0131)   & (0.00592)  & (0.00558)  & (0.00484)    & (0.00412)  \\[1ex]
\% part-time empl.              &       -0.00418    & 0.00662*    & 0.0570***    & -0.0829*** & 0.0551***  & -0.0535***  & 0.0777***   & -0.0969*** & 0.0258***  & -0.0161*** & 0.00355      & -0.00317   \\
                                &       (0.00418)   & (0.00375)   & (0.00675)    & (0.00791)  & (0.00342)  & (0.00332)   & (0.0301)    & (0.0255)   & (0.00612)  & (0.00545)  & (0.00412)    & (0.00395)  \\[1ex]
\% permanent contracts          &       -0.0345***  & 0.0436***   & -0.0242***   & 0.0619***  & 0.000483   & 0.00465*    & -0.00203    & 0.00554    & -0.0277*** & 0.0188*    & -0.0778***   & 0.0826***  \\
                                &       (0.00567)   & (0.00617)   & (0.00906)    & (0.0107)   & (0.00255)  & (0.00238)   & (0.00795)   & (0.00835)  & (0.00961)  & (0.00979)  & (0.00750)    & (0.00651)  \\[1ex]

Firm size:                      \\[1ex]
\quad \textit{50--249 empl.}    &       0.000439    & -0.000491   & 0.0217***    & -0.0144*** & 0.0657***  & -0.0554***  & 0.00895     & -0.00736   & -0.0330*** & 0.0301***  & 0.0201***    & -0.0177*** \\
                                &       (0.00275)   & (0.00287)   & (0.00268)    & (0.00314)  & (0.00235)  & (0.00182)   & (0.00578)   & (0.00566)  & (0.00776)  & (0.00724)  & (0.00258)    & (0.00217)  \\[1ex]
\quad \textit{$\geq$ 250 empl.} &       -0.00293    & 0.000297    & 0.0239***    & -0.0101*** & 0.106***   & -0.0876***  & 0.00620     & -0.00498   & -0.0332*** & 0.0320***  & 0.0226***    & -0.0264*** \\
                                &       (0.00469)   & (0.00472)   & (0.00275)    & (0.00313)  & (0.00397)  & (0.00347)   & (0.00880)   & (0.00795)  & (0.00663)  & (0.00592)  & (0.00256)    & (0.00230)  \\[1ex]
Public firm                     &       -0.0219***  & 0.0283***   & -0.0129*     & -0.00926   & 0.0183***  & -0.0105***  & -0.0448***  & 0.0349***  & 0.00734    & -0.0121    & -0.0402***   & 0.0292***  \\
                                &       (0.00592)   & (0.00583)   & (0.00663)    & (0.00830)  & (0.00436)  & (0.00379)   & (0.00719)   & (0.00558)  & (0.0352)   & (0.0268)   & (0.00458)    & (0.00407)  \\[1ex]
\midrule
Observations                    &       13,765      & 13,765      & 12,312       & 12,312     & 37,887     & 37,887      & 3,498       & 3,498      & 14,502     & 14,502     & 30,009       & 30,009     \\
R-squared                       &       0.138       & 0.199       & 0.059        & 0.059      & 0.174      & 0.191       & 0.226       & 0.191      & 0.110      & 0.124      & 0.105        & 0.115      \\
Region FE                       &       \checkmark  & \checkmark  & \checkmark   & \checkmark & \checkmark & \checkmark  & \checkmark  & \checkmark & \checkmark & \checkmark & \checkmark   & \checkmark \\
Sector FE                       &       \checkmark  & \checkmark  & \checkmark   & \checkmark & \checkmark & \checkmark  & \checkmark  & \checkmark & \checkmark & \checkmark & \checkmark   & \checkmark \\
\bottomrule
\end{tabular}
\begin{tablenotes}
\item Notes: Bootsrapped standard errors in parentheses (200 repetitions); asterisks denote significance levels: $^{*}$ p$<$0.05, $^{**}$ p$<$0.01, $^{***}$ p$<$0.001
\end{tablenotes}
\end{threeparttable}
}
\end{sidewaystable}

\clearpage

\appendix


\section*{Appendix~A: Descriptive statistics}
\label{sec:appendix_desc}
\setcounter{table}{0}
\renewcommand{\thetable}{A\arabic{table}}

\noindent Table~\ref{tab:summary_stats} shows shares of employees and firms falling in different categories of bargaining in our working sample, by country and by year, also including firms that do not apply any form of collective bargaining (\emph{i.e.,} contract wages separately with each single employee).

\begin{table}[hbp]
\centering
\caption{Share of firms and employees under different bargaining regimes, by country and year}
\label{tab:summary_stats}
\resizebox*{0.9\linewidth}{!}{
\begin{tabular}{ll|rrrrrr|rr} \toprule
\multicolumn{2}{r}{\textbf{Bargaining:}} & \multicolumn{2}{c}{\textbf{Centralized}} & \multicolumn{2}{c}{\textbf{Firm-level}} & \multicolumn{2}{c}{\textbf{None}} & \multicolumn{2}{c}{\textbf{Total}} \\
\cmidrule(lr){1-2} 
Country & Year & \% firms & \% empl & \% firms & \% empl & \% firms  & \% empl & N empl & N firms\\
\cmidrule(lr){1-2} \cmidrule(lr){3-4} \cmidrule(lr){5-6} \cmidrule(lr){7-8} \cmidrule(lr){9-10} BE      & 2006 & 82.0   & 79.5 & 18.0  & 20.5   & 0.0  & 0.0   & 165,191   & 8,947   \\
BE      & 2010 & 81.1   & 78.8 & 18.9  & 21.2   & 0.0  & 0.0   & 137,254   & 6,885   \\[1ex]
CZ      & 2006 & 2.6    & 6.2  & 7.3   & 47.8   & 90.1 & 46.0  & 1,970,864 & 18,059  \\
CZ      & 2010 & 2.9    & 4.6  & 8.3   & 49.4   & 88.8 & 46.0  & 1,993,625 & 18,046  \\[1ex]
DE      & 2006 & 20.3   & 58.4 & 6.5   & 5.6    & 73.2 & 36.0  & 2,892,881 & 37,132  \\
DE      & 2010 & 27.0   & 45.5 & 6.0   & 7.1    & 66.9 & 47.4  & 1,701,358 & 29,603  \\[1ex]
ES      & 2006 & 87.5   & 80.5 & 12.5  & 19.5   & 0.0  & 0.0   & 235,272   & 27,301  \\
ES      & 2010 & 76.9   & 65.3 & 15.9  & 26.6   & 7.2  & 8.1   & 216,769   & 25,104  \\[1ex]
FR      & 2006 & 88.3   & 76.2 & 6.4   & 16.8   & 5.3  & 7.0   & 113,641   & 15,386  \\
FR      & 2010 & 89.1   & 81.9 & 10.1  & 17.2   & 0.9  & 0.9   & 220,369   & 30,693  \\[1ex]
UK      & 2006 & 22.4   & 24.1 & 28.1  & 25.1   & 49.4 & 50.8  & 133,343   & 43,011  \\
UK      & 2010 & 17.1   & 24.4 & 20.7  & 23.3   & 62.2 & 52.4  & 179,116   & 104,227 \\
\bottomrule
\end{tabular}
}
\end{table}

\bigskip

\noindent Table~\ref{tab:firm_size}, Table~\ref{tab:sum_reg_var_cont} and Table~\ref{tab:sum_reg_var_mod_age} provide basic descriptive statistics for variables entering as controls in the regression analysis.

\begin{table}[ht]
\centering
\small
\caption{Distribution of firms by size by country}
\label{tab:firm_size}
\begin{tabular}{lrrrrrr}
\toprule
Firm size: & BE    & CZ     & DE     & ES     & FR     & UK     \\
\midrule
1--49      & 4,547 & 22,218 & 26,592 & 19,695 & 7,818  & 9,343  \\
50--249    & 4,533 & 6,633  & 19,021 & 8,662  & 9,019  & 2,765  \\
$\geq$250  & 5,766 & 3,251  & 15,177 & 10,400 & 14,058 & 20,332 \\
\bottomrule
\end{tabular}

 \end{table}

\begin{table}[htp]
\caption{Summary means and standard deviations for continuous variables in regression}
\label{tab:sum_reg_var_cont}
\centering
\small
\begin{tabular}{llrrrrrr}
\toprule
country                     & Stat. & BE    & CZ    & DE    & ES    & FR     & UK    \\
\midrule
Mean experience empl. (yrs) & mean & 9.506 & 9.061 & 8.574 & 7.235 & 10.748 & 7.079 \\
                            & s.d. & 5.810 & 5.229 & 6.051 & 6.202 & 6.361  & 5.278 \\[1ex]
\% empl. with tert. educ.   & mean & 0.329 & 0.244 & 0.123 & 0.318 & 0.408  & 0.379 \\
                            & s.d. & 0.330 & 0.253 & 0.195 & 0.327 & 0.298  & 0.266 \\[1ex]
\% empl. with sec. educ.    & mean & 0.419 & 0.654 & 0.706 & 0.190 & 0.420  & 0.523 \\
                            & s.d. & 0.305 & 0.255 & 0.236 & 0.244 & 0.270  & 0.262 \\[1ex]
\% managers and profess.    & mean & 0.197 & 0.398 & 0.110 & 0.142 & 0.293  & 0.273 \\
                            & s.d. & 0.287 & 0.303 & 0.175 & 0.241 & 0.256  & 0.266 \\[1ex]
\% part-time empl.          & mean & 0.229 & 0.139 & 0.266 & 0.152 & 0.159  & 0.269 \\
                            & s.d. & 0.260 & 0.186 & 0.248 & 0.247 & 0.231  & 0.268 \\[1ex]
\% permanent contracts      & mean & 0.925 & 0.785 & 0.885 & 0.750 & 0.921  & 0.938 \\
                            & s.d. & 0.162 & 0.215 & 0.148 & 0.303 & 0.171  & 0.156 \\
\bottomrule                            
\end{tabular}

 \end{table}

\begin{table}[ht]
\caption{Distribution of firms by modal age of employees by country}
\label{tab:sum_reg_var_mod_age}
\centering
\small
\begin{tabular}{lrrrrrr}
\toprule
Modal age workers: & BE    & CZ     & DE     & ES     & FR    & UK    \\
\midrule
14-19              & 35    & 19     & 276    & 127    & 68    & 987   \\
20-29              & 2,414 & 2,288  & 9,277  & 8,726  & 3,664 & 6,954 \\
30-39              & 4,681 & 6,254  & 10,543 & 18,674 & 9,098 & 8,052 \\
40-49              & 5,671 & 10,787 & 28,528 & 11,421 & 9,926 & 8,929 \\
50-59              & 2,417 & 12,166 & 11,563 & 5,275  & 7,648 & 6,174 \\
60+                & 54    & 588    & 603    & 564    & 491   & 1,371 \\
\bottomrule
\end{tabular}
\end{table}

\clearpage

\section*{Appendix~B: FLB propensity score estimation details}
\noindent In order to address possible endogeneity driven by the potential self-selection of firms into a particular bargaining regime ($\mathit{FLB}=1)$, we essentially apply a two-step procedure based on propensity score estimates. We first estimate, separately by country, a preliminary first-step Probit

\begin{equation}
\label{eq:reg_propensity}
  \mathit{FLB}_j = \mathit{P}\left( \alpha_0 + \alpha_1\mathbf{V}_j \right)
\end{equation}
where $\mathit{FLB}_j$ is the dummy for the \emph{observed} presence of firm-level bargaining in firm $j$, $\mathit{P}$ is the Probit link function, and $\mathbf{V}$ a set of covariates that affect the choice to bargain at firm-level.

In the second step, the predicted probabilities (propensity scores)
$\widehat{\mathit{FLB}}_j = \mathit{P}\left( \hat{\alpha}_0 +
  \hat{\alpha}_1\mathbf{V}_j \right)$ obtained for each firm are included
as an additional control variable, as shown in Equation~\ref{eq:reg_dispersion} and Equation~\ref{eq:reg_decomposition}. The idea is that conditioning also upon the p-scores in addition to other controls solves selection due to unobserved factors, if FLB status is assigned as good as random based on observables. Thus, a simple OLS on the ‘p-score augmented’ second step regressions will return correct estimates of the FLB dummy coefficient.

Table~\ref{tab:propensity} reports first-step Probit estimates, that
we use to compute FLB p-scores for each firm. They show a satisfactory
goodness of fit, in terms of relatively high values of the area under
the ROC curve.

\begin{table}[thb]
\caption{Probit estimates of FLB propensity}
\label{tab:propensity}
\centering
\resizebox*{0.75\linewidth}{!}{
\begin{threeparttable}
\begin{tabular}{l*{6}{l}}
\toprule
                          & \multicolumn{1}{c}{BE}  & \multicolumn{1}{c}{DE}  & \multicolumn{1}{c}{ES}  & \multicolumn{1}{c}{CZ}  & \multicolumn{1}{c}{UK}  & \multicolumn{1}{c}{FR}  \\
\midrule
Mean experience empl.     &  0.0300*** & 0.0243*** & 0.0409***  & 0.0755*** & -0.0127*** & 0.00221    \\
                          &  (0.00313) & (0.00276) & (0.00184)  & (0.00730) & (0.00277)  & (0.00282)  \\[0.5ex]

Modal age workers:        \\[1ex]
\quad 20-29               &  -0.0844   & -0.890*** & $-$0.110     &           & 0.115      & 0.125      \\
                          &  (0.252)   & (0.246)   & (0.218)    &           & (0.0936)   & (0.142)    \\[0.5ex]

\quad 30-39               &  0.0366    & -0.564**  & -0.0319    & -0.0328   & 0.141      & 0.285**    \\
                          &  (0.250)   & (0.245)   & (0.218)    & (0.104)   & (0.0938)   & (0.136)    \\[0.5ex]

\quad 40-49               &  0.00410   & -0.498**  & -0.0881    & 0.0855    & 0.124      & 0.395***   \\
                          &  (0.249)   & (0.244)   & (0.218)    & (0.117)   & (0.0934)   & (0.134)    \\[0.5ex]

\quad 50-59               &  -0.124    & -0.559**  & -0.0560    & 0.0737    & 0.152      & 0.420***   \\
                          &  (0.251)   & (0.245)   & (0.219)    & (0.107)   & (0.0956)   & (0.134)    \\[0.5ex]

\quad 60+                 &            & -0.829*** & -0.113     & -0.357    & 0.120      &            \\
                          &            & (0.298)   & (0.232)    & (0.299)   & (0.110)    &            \\[0.5ex]

\% empl. with tert. educ. &  0.109     & 0.662***  & 0.405***   & 0.455     & -0.198**   & 0.108      \\
                          &  (0.0793)  & (0.135)   & (0.0393)   & (0.347)   & (0.0991)   & (0.0759)   \\[0.5ex]

\% empl. with sec. educ.  &  0.137**   & 0.728***  & 0.159***   & 0.117     & -0.133     & 0.522***   \\
                          &  (0.0557)  & (0.0982)  & (0.0382)   & (0.237)   & (0.0945)   & (0.0699)   \\[0.5ex]

\% managers and profess.  &  -0.0209   & 0.0162    & -0.186***  & 0.206     & -0.00602   & -0.581***  \\
                          &  (0.0906)  & (0.110)   & (0.0604)   & (0.290)   & (0.0594)   & (0.0785)   \\[0.5ex]

\% part-time empl.        &  -0.152**  & 0.115*    & -0.0164    & 0.0864    & -0.0706    & 0.386***   \\
                          &  (0.0682)  & (0.0648)  & (0.0430)   & (0.366)   & (0.0560)   & (0.0558)   \\[0.5ex]

\% permanent contracts    &  0.510***  & 0.116     & -0.156***  & -0.279*   & 0.242**    & -0.262***  \\
                          &  (0.115)   & (0.127)   & (0.0385)   & (0.151)   & (0.0978)   & (0.0797)   \\[0.5ex]

Firm size:                \\[1ex]
\quad \textit{50-249 empl.}              &  0.704***  & 0.105***  & 0.624***   & 0.440***  & -0.184**   & 0.473***   \\
                          &  (0.0408)  & (0.0383)  & (0.0235)   & (0.0855)  & (0.0720)   & (0.0385)   \\[0.5ex]

\quad \textit{$\geq$ 250 empl.}         &  1.135***  & 0.179***  & 1.133***   & 1.164***  & -0.290***  & 0.624***   \\
                          &  (0.0425)  & (0.0375)  & (0.0233)   & (0.0916)  & (0.0511)   & (0.0387)   \\[0.5ex]

Public firm               &  -1.230*** & -0.921*** & 0.570***   & 0.202     & -1.346***  & 0.838***   \\
                          &  (0.0964)  & (0.0404)  & (0.0405)   & (0.139)   & (0.0403)   & (0.0428)   \\[0.5ex]

NACE Sector:              \\[1ex]
\quad D                   &  -0.413*   & -0.293**  & -0.380***  & 0.609***  & 0.201      & -1.373***  \\
                          &  (0.243)   & (0.137)   & (0.0667)   & (0.218)   & (0.311)    & (0.180)    \\[0.5ex]

\quad E                   &  -0.809*** & 0.407***  & 0.272***   & 1.075***  & 0.141      & 1.157***   \\
                          &  (0.304)   & (0.141)   & (0.0803)   & (0.305)   & (0.328)    & (0.155)    \\[0.5ex]

\quad F                   &  -1.140*** & -1.096*** & -0.882***  & -0.894*** & -1.032***  & -1.334***  \\
                          &  (0.253)   & (0.203)   & (0.0770)   & (0.225)   & (0.318)    & (0.278)    \\[0.5ex]

\quad G                   &  -0.853*** & -0.451*** & -0.493***  & 0.799***  & 0.405      & -1.270***  \\
                          &  (0.245)   & (0.156)   & (0.0715)   & (0.254)   & (0.311)    & (0.220)    \\[0.5ex]

\quad H                   &  -0.999*** & 0.275     & -0.808***  &           & -0.510     & -0.486***  \\
                          &  (0.268)   & (0.168)   & (0.0832)   &           & (0.334)    & (0.182)    \\[0.5ex]

\quad I                   &  -0.441*   & 1.229***  & -0.112     & -0.168    & -0.248     & 0.573***   \\
                          &  (0.248)   & (0.136)   & (0.0705)   & (0.228)   & (0.310)    & (0.152)    \\[0.5ex]

\quad J                   &  -0.0811   & -0.816*** & -1.105***  & 0.216     & 0.562*     & -0.00925   \\
                          &  (0.255)   & (0.161)   & (0.0818)   & (0.309)   & (0.315)    & (0.158)    \\[0.5ex]

\quad K                   &  -0.744*** & 0.117     & -0.509***  & 1.391***  & 0.0816     & 0.169      \\
                          &  (0.247)   & (0.139)   & (0.0728)   & (0.286)   & (0.314)    & (0.155)    \\[0.5ex]

\quad L                   &            &           & -0.101     & 2.351***  & 0.324      & 1.735***   \\
                          &            &           & (0.111)    & (0.364)   & (0.311)    & (0.159)    \\[0.5ex]

\quad M                   &  -0.966*** & -0.0856   & -0.953***  & 0.142     & -0.676**   & -0.510***  \\
                          &  (0.282)   & (0.157)   & (0.0920)   & (0.286)   & (0.311)    & (0.189)    \\[0.5ex]

\quad N                   &  -0.860*** & 0.973***  & -0.670***  &           & -1.004***  & -0.554***  \\
                          &  (0.248)   & (0.138)   & (0.0834)   &           & (0.312)    & (0.163)    \\[0.5ex]

\quad O                   &  -0.602**  & 0.394***  & 0.126*     & 1.703***  & -0.496     & 0.338**    \\
                          &  (0.254)   & (0.139)   & (0.0744)   & (0.357)   & (0.311)    & (0.155)    \\[0.5ex]

Regional GDP pps          &  0.00120   & 0.000710  & 0.00371    & 0.302***  & 0.00649*** & -0.00184   \\
                          &  (0.00157) & (0.00397) & (0.00231)  & (0.0712)  & (0.00181)  & (0.00168)  \\[0.5ex]

Regional unemp. rate      &  0.00426   & 0.0357*** & 0.00866*** &           & 0.0716***  & -0.0433*** \\
                          &  (0.00366) & (0.00585) & (0.00161)  &           & (0.00832)  & (0.0101)   \\[0.5ex]

Constant                  &  -1.770*** & -1.832*** & -1.679***  & -7.023*** & 0.544      & -2.022***  \\
                          &  (0.374)   & (0.329)   & (0.235)    & (1.437)   & (0.351)    & (0.243)    \\
\midrule
Observations              &  13,730    & 12,312    & 37,887     & 3,498     & 14,502     & 29,943     \\
Area under ROC curve      &  0.781     & 0.811     & 0.825      & 0.870     & 0.875      & 0.935      \\
\bottomrule
\end{tabular}
\begin{tablenotes}
\item Notes: Dependent variable is FLB dummy. Standard errors in parentheses; asterisks denote significance levels: $^{*}$ p$<$0.05, $^{**}$ p$<$0.01, $^{***}$ p$<$0.001
\end{tablenotes}
\end{threeparttable}
} 
\end{table}

Notice that the predictors $\mathbf{V}$ are for the most part the same as
the controls appearing in the set $\mathbf{X}$ in the main equations.
However, to ease identification, we exclude average tenure of the
workforce, as it is sensible to assume that tenure affects wages and
wage inequalities, but it does not directly impact on the decision to
adopt FLB. Also notice that, in place of the sector and region
fixed-effects included in the controls $\mathbf{X}$ (likely subject to
incidental parameter problems in Probit estimates), the set of
covariates $\mathbf{V}$ includes the GDP per capita (at
purchasing power parity, base year 2006) and the unemployment rate in
the region where each firm is located, thus controlling for
macroeconomic-and-regional dynamics that may play a direct influence
on the decision to apply firm-level bargaining.\footnote{These
  additional variables are taken from EUROSTAT-Regional Statistics and
  measured at the level of NUTS-1 regions, since this is the
  precision of the information on firms' geographical location in
  SES.}



\end{document}
