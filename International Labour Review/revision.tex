\documentclass[12pt]{article}
\usepackage[a4paper,left=2cm,right=2cm,top=1.0in,bottom=1.0in]{geometry}

\usepackage[english]{babel}
% \usepackage[utf8]{inputenc}
% \usepackage[T1]{fontenc}
% \usepackage{lmodern} 
% Punteggiatura in stile TeX: ``virgolette", -- incisi -- 

\usepackage{rotating}
\usepackage{pdflscape}
\usepackage{graphicx}
% \usepackage{epstopd}
% \usepackage{color}

%\usepackage{endnotes} % Move footnote at the end of the document
%\let\footnote=\endnote

% \usepackage{calc}
\usepackage{indentfirst}
\usepackage[toc,page]{appendix}
\usepackage{amsmath,amsfonts,bm,amssymb}
% \usepackage{hyperref}

%\usepackage{caption3}
%\usepackage{float}

% \usepackage{longtable}
% \usepackage{colortbl}
% \usepackage[flushleft]{threeparttable}
\usepackage{booktabs}
\usepackage{multirow}
\usepackage{bigdelim}
% to hide table columns, using H type
\usepackage{array}
\newcolumntype{H}{>{\setbox0=\hbox\bgroup}c<{\egroup}@{}}
% \usepackage{supertabular}

\usepackage{siunitx} % For regression tables
\sisetup{%
    detect-all, % current font
    parse-numbers = false,%
    table-number-alignment  = center,%
    group-digits            = false,%
    table-format            = -2.6,%
    table-auto-round,%
    input-symbols           = {()},  % redefine ( ) as text symbols
    table-space-text-pre    = {$-$},   % allow for proper spacing of -(
    table-space-text-post   = {$^{***}$},
    table-align-text-post = false, % toggle alignment of *** after estimates
  }

\usepackage{threeparttable}
\usepackage{authblk}
\usepackage{setspace}

% Questo il formato che richiede EJIR per la biblio
\usepackage[authoryear]{natbib}
\bibpunct{(}{)}{;}{}{,}{,} % redefine punctuation of natbib output
%\bibliographystyle{SageH}
\bibliographystyle{chicago}

\begin{document}



\title{Firm-level bargaining and within-firm wage inequality: Evidence across Europe}
%\title{Bargaining decentralisation and within-firm wage inequality: Evidence across Europe}
% (collective) wage bargaining; Pay agreements
% Firm-level or Decentralised 

\date{}

%\author[a]{Valeria Cirillo}
%\author[b,c]{Matteo Sostero}
%\author[c]{Federico Tamagni\footnote{\textit{Corresponding author}: Federico Tamagni, Scuola Superiore Sant'Anna, Pisa, Italy. Postal address: c/o Institute of Economics, Scuola Superiore Sant'Anna, Piazza Martiri 33, 56127, Pisa, Italy, \emph{E-mail}~f.tamagni@sssup.it, \emph{Tel}~+39-050-883343.}}

%\affil[a]{\footnotesize{Department of Political Sciences, University of Bari "Aldo Moro", Bari, Italy}}
%\affil[b]{Joint Research Center, European Commission, Seville.\footnote{This work was primarily conducted prior to joining the JRC. The scientific output expressed here does not imply a policy position of the European Commission. Neither the European Commission nor any person acting on behalf of the Commission is responsible for the use which might be made of this publication.}}
%\affil[c]{Institute of Economics, Sant'Anna School of Advanced Studies, Pisa, Italy}

% articolo totale 10 000 words

\maketitle

%EJIR: abstracts generally ~ 150 words, currently 143
\begin{abstract}
The collective bargaining process in European economies has since the 1990s gradually shifted from centralised bargaining to firm-level agreements. This transition allows firms to change their internal wage structure responding to local conditions, with potentially contrasting effects on within-firm inequality. This paper examines the empirical association between firm-level bargaining and within-firm wage inequality, particularly the distance between wages of highly-paid and low-paid employees. We exploit data over the period 2006-2010 for six European economies -Belgium, Spain, Germany, France, the Czech Republic and the UK- allowing to test for heterogeneity of main effects across different collective bargaining traditions. We find that within-firm inequality was narrower under firm-level bargaining than in centralised bargaining in the UK, while it widened over time in France and Spain.
\bigskip
 
\noindent \textbf{Keywords}: within-firm wage inequality, firm-level bargaining, coordinated bargaining, wage setting regimes, matched employer-employee data


\noindent \textbf{JEL classification}: J31, J33, J51, J52

\end{abstract}

\cleardoublepage

\onehalfspacing

\section{Introduction}
\label{sec:intro}
The notable rise in inequality observed in many countries since the 2008 global recession has reopened the debate on its causes. Together with technological change, globalisation, the decline in union power, and the role of finance, scholars suggested that the process of increasing the flexibility of labour market and wage-setting institutions played a major role in increasing wage inequality~\citep{cobb2016}. Various reforms were implemented in many different countries between the late 1990s and the early 2000s, in line with the policy recommendations of the 1994 OECD Jobs Strategy report. 

This article examines the role played for inequality by the devolution of bargaining levels, i.e., the progressive shift in the locus of collective wage setting from more centralised levels (national or industry) to the level of individual firms, with the intended aim at catering to the specific needs of firms, by allowing them to change wages based on their internal and local market conditions~\citep{undy1978}. This trend has affected wage-setting legislation particularly in Europe, where the “corporatist” system of industrial relations~\citep{wallerstein1997unions}, which included high union coverage and centralised collective bargaining and was prevalent in European countries in the second half of the 20th century, has gradually morphed into a “hybrid” system~\citep{Braakmann}. Although coordinated (or “multi-employer”) collective bargaining conducted at centralised level may still predominate, firm-level (“single-employer”) collective agreements increasingly derogate to specific provisions stipulated at centralised levels~\citep{visser2013wage}.

The increased role of firm-level collective agreements has been connected in the academic literature to two types of wage inequality, that is between or within firms. The vast majority of studies focuses on between-firm wage inequality, asking whether firm-level bargaining on top of centralised wage setting can explain that otherwise similar workers are paid differently in different firms. The general finding, confirmed also in the recent cross-country effort coordinated by the OECD~\citep{criscuolo2020,criscuolo2021,criscuolo2023} is that between-firm wage inequality is determined to a significant extent by firm wage-setting practices rather than by workers' characteristics alone. In particular, firm-level agreements tend to increase between-firm wage dispersion across countries~\citep{Berlingieri2017}, a finding already highlighted in the 2018 OECD Economic Outlook \citep{OECD2018}. This may be explained by greater bargaining power of high-skilled employees under local bargaining than under centralised bargaining, leading to even higher wages to well-paid workers under firm-level bargaining, as suggested by \cite{dahl.lemaire.ea.2013}.

In this paper, we instead address the much less studied effect of wage-setting decentralisation on within-firm inequality, thus addressing whether firms that apply firm-level bargaining have a more unequal internal wage structure, compared to those under centralised (national or sector) bargaining schemes. Considering within-firm inequalities is important for two reasons. First, at a general level, because within-firm wage differences explain a similar share as between-firm inequalities in overall wage dispersion. In fact, despite some extreme cases of countries where between-firm inequality gets the lion's share, such as in the US~\citep{barthetal_2016}, between-firm and within-firm inequality each accounts for around half of total wage dispersion in most economies~\citep{lazear.shaw.2007,fournier.koske.2013,GlobalWageReport,criscuolo2020}. Second, and more specifically, because wage differences between firms only provide an incomplete picture of wage-setting dynamics. Focusing on within-firm inequality allows to describe how the ability to set wages locally affects wage dynamics at the organisation level, telling whether firm-level bargaining is used by firms to differently compensate different employees, and hence how employers shape overall income inequality through this channel. 

As we detail in~Section~2, the literature on the role of firm-level bargaining for within-firm inequalities provides contrasting theoretical predictions, while the few existing empirical studies find mixed results, mostly based on relatively old data, referring to the 1990s, when the push towards reforming labour markets started. In this paper, exploiting the matched employer-employee information recorded in the European Structure of Earnings Survey (SES) available for six European countries -- Belgium, Spain, France, Germany, the Czech Republic and the UK -- for the years 2006 and 2010, we provide three main contributions to this relatively underdeveloped literature.  

First, our general aim is at estimating the effect of firm-level bargaining on within-firm inequality at the firm level, whereas previous studies are concerned with the aggregate explanatory power of different bargaining levels vis-a-vis other determinants of wage inequality, typically within an inequality decomposition framework. We instead design a regression exercise allowing to directly estimate whether firms under firm-level bargaining show larger or smaller within-firm wage inequality compared to firms that only bargain at more centralised levels. In particular, we take the interdecile range in the within-firm distribution of residual wages as our measure of within-firm inequality, thus focusing on whether, controlling for workers and firm characteristics, the possibility to bargain locally -on top of at centralised level- is used by firms to adjust wages of already high-paid and/or low-paid employees. We can thus address whether the extent of within-firm wage inequality due to firm-level bargaining is driven by high-paid workers being paid even more under firm-level bargaining, or low-paid workers being paid even less, or vice-versa, or a combination of the former cases.

Second, by taking a cross-country perspective, in this paper we pay specific attention to how the association between firm-level bargaining and within-firm inequality may possibly vary according to the different institutional context. In fact, the institutional setting wherein firms operate -- epitomised by the national system of industrial relations -- frames the conditions for implementing firm-level bargaining. Although we couldn't expand the analysis to a wider set of countries, due to missing data on firm-level practices in other countries in principle available in SES, the included countries provide a good representation of collective wage-bargaining traditions in Europe. Despite the common trend toward decentralisation of bargaining level and some cross-country similarity in terms of "bargaining regime"~\citep{fulton.2013}, they still show marked differences in the prevalent form of collective bargaining in the years under study. To avoid any assumption of homogeneity across national institutional settings, we estimate the same regression models separately for each country, as opposed to a pooled model with country fixed-effects, thus allowing coefficient estimates of firm-level bargaining to vary by country. Comparing coefficient estimates across countries allows us to appreciate the extent of heterogeneity or similarity across countries and bargaining regimes. As we further discuss in Section~2, country-specific features of bargaining systems support that while firm-level bargaining is more likely associated with increased within-firm wage inequality in the UK and the Czech Republic, egalitarian pressures are more likely in Germany and Spain, whereas Belgium is a country where firm-level bargaining has likely no effect and France is an harder to predict case. At the same time, if the ongoing convergence of wage-setting institutions had completely blurred the institutional differences across countries~\citep[see][]{baccaro2017trajectories}, our estimates should reveal a complete uniformity in the effect of firm-level bargaining across countries.

Third, we contribute by examining whether the relations linking firm-level bargaining to within-firm inequality change over time, in between the two years covered in our data (2006 and 2010). The time period under study represents an interesting test-bed, since it is a phase of relatively stable institutional setting, after major reforms took place. In fact, the legal provision to stipulate firm-level collective agreements was introduced in all countries before the time-span under study, but the pressure toward assigning more relevance and wider scope to firm-level negotiations likely continued during the period. Thus, 2006 is likely providing a snapshot of the period where reforms toward devolution of bargaining level unfolded their effects, while it is likely that by 2010 the use of firm-level agreements to differentiate salaries has further increased. We thus expect firm-level bargaining in 2010 to be associated with more unequal within-firm wage distributions compared to 2006. Moreover, the data for 2010 is especially interesting, because it covers part of the Great Recession. During that time, wages were under pressure, and companies may have used the flexibility of firm-level bargaining to restructure wages.



\section{Background and hypotheses}
\label{sec:background}

A vast and established literature examines the relative importance of between-firm and within-firm wage inequality for overall inequality, typically via decomposition of aggregate wage dispersion into these two components. A disproportionate majority of these studies seeks then to relate firm-level attributes or institutional settings, such as the bargaining level adopted by a firm, to between-firm wage inequality only, asking whether firm-level bargaining or other firm attributes can explain why similar workers are paid differently in different firms~\citep{dellaringa.lucifora.1994,hibbs.locking.1996,palenzuela.jimeno.1996,hartog.leuven.ea.2002,rycx.2003,cardoso.portugal.2005,checchi.pagani.2005,plasman.rusinek.ea.2007,card.delarica.2006,dellaringa.pagani.2007,daouli.demoussis.ea.2013,ehrl2017}. This attention to between-firm inequalities, highlighting the central role of the firm as the key locus of wage inequality creation, has seen a renewed interest in recent years, driven by the OECD effort to understand the origins of inequality across countries~\citep{criscuolo2020,criscuolo2021,criscuolo2023}, and also in relations to emergence of new technological trends such as firms' use of big-data~\citep{silva_etal2022}.

This paper relates to a much narrower literature which examines empirically how firm-level bargaining possibly affect the internal wage structure of firms, that is within-firm wage inequality. In this section we first present the main theories that may offer a guidance in addressing this research question and the scant empirical evidence on the subject. Then, we briefly discuss the key institutional features of wage bargaining systems of the countries that we analyse, providing some hypotheses about heterogeneity of the effects we could expect to observe in the data.  


\subsection{Theoretical and empirical literature}

Theoretically, the links between the level of collective bargaining and within-firm wage inequalities can be framed within several approaches, with contrasting predictions about whether bargaining locally on top of more centralised agreements should result in higher or lower within-firm wage inequality.

Economic theories primarily stress firm-specific incentives as drivers for the adoption of firm-level agreements. Within-firm inequality is predicted to be higher in decentralised bargaining every time firm-level agreements are designed to elicit or selectively compensate the contribution of different employees to the firms' performance and objectives~\citep{bayo2013diffusion}.
This may happen under performance-related pay or other differentials-in-compensation schemes consistent with Tournament theory~\citep{lazear.1979}. It may also selectively remunerate human capital, or particularly valuable firm-specific resources (according to the resource-based view of the firm). It may even solve transaction costs and agency problems arising for different occupational groups~\citep{eisenhardt1989agency,o1998structure}. On the other hand, local bargaining may reduce within-firm inequality, compared to centralised bargaining, if these types of workplace collective agreements respond to motives of re-distribution, fairness or equity pursued by workers. This may stem from the preference of workers or unions to equalise wages (across but also within firms), as described in insider-outsider models “with unions”~\citep{lindbeck1986wage,lindbeck2001insiders}, or in “fair wage” theories~\citep{akerlof.1984}. 

Other mechanisms linking within-firm wage inequalities to centralised vs. decentralized bargaining are suggested by models of wage-setting that explain the wage-gap between the so called market-clearing wage and the wage actually paid to workers, on the basis of efficiency-wages, rent-sharing or differential compensations for unmeasured workers' ability. Although these practices
are more directly related to between-firm wage inequalities, they may also affect inequalities within a firm, if they are used by employers to selectively reshape the overall pay-scale, to adjust wages of specific groups of workers, and not of others. It is however difficult to formulate predictions on whether they should result into increased or decreased within-firm pay inequality. Their effect indeed depends on the actual implementation, and on the willingness of workers and unions to pursue egalitarian objectives aiming at standardization of wages in the firm-level negotiations.

Beyond the incentive motives analysed by economists, other literatures highlight  firm-specific characteristics that explain how and why firms act on their internal wage structure. Sociological or socio-economic research -- and in particular recent developments of organisational approaches to stratification that discuss the firm as the central locus of wage inequality creation~\citep{stainback2010,cobb2016} -- stress the crucial role of organisational inertia and the relative balance of power among groups within organisations. Resistance to change favours the continuation of positions of individuals and wage structure within a firm, whereas the resolution of conflicts among groups with different goals and power in the hierarchical, organisational and occupational structure~\citep{blau1967american, goldthorpe1972occupational, wright1980class, erikson2002intergenerational}, may result into either reducing or increasing inequalities within firms, both statically and over time. 

Overall, the effect of firm-level bargaining on within-firm wage inequality remains ultimately an empirical question. The implementation of firm-level bargaining provisions is likely to vary considerably in different firms, with uncertain outcomes on within-firm wage inequality, depending on the relative strength of the factors mentioned above. 

In agreement with the difficulty to provide sharp predictions, the few existing empirical studies find heterogeneous results. Compared to our work, the available studies use similar datasets, essentially exploiting national versions of the SES in its early editions, dating back to the late 1990s or early 2000s.~\cite{dellaringa.lucifora.1994} examine Italian data for the year 1990, and find that within-firm wage dispersion does not differ across firms that only apply centralised bargaining and firms that also apply firm-level agreements. This result is confirmed for Italy, Spain, Belgium and Ireland on data for the year 1995 by~\cite{dellaringa.lucifora.ea.2004}, highlighting the need to adopt residual wages and control variables. Indeed, they show that the wider unconditional within-establishment wage inequality observed for enterprises covered by a single-employer agreement, disappears when a large set of controls is included. Conversely,~\cite{canaldominguez.gutierrez.2004} find that firm-level bargaining
reduces within-firm wage dispersion in Spain, on data again for the year 1995. Finally, Addison et al. (2014) exploits a panel of German establishments over the period 1996-2008. Their findings show a modest widening of within-establishment wage dispersion for establishments that exit from sectoral agreement.



\subsection{Wage setting frameworks in selected countries}

The main features of the national bargaining systems in place in the countries included in our work, over the time-span covered by our data, are summarised in Table~\ref{tab:CBTable}.\footnote{Reported information draws from our own elaboration of the following data-sources and reports: the \textit{Labour Market Reforms}-LABREF database maintained by the European Commission, available at~https://webgate.ec.europa.eu/labref/public; the section “Industrial Relations” of the ILOstat data website~http://www.ilo.org/ilostat; the~\textit{European Company Survey}-ECS, run by the Eurofound Industrial Relations Observatory; the \textit{Institutional Characteristics of Trade Unions, Wage Setting, State Intervention and Social Pacts}-ICTWSS database. We also exploit the broader discussion of legal and institutional aspects featuring the bargaining systems of different countries developed in~\cite{fulton.2013,fulton.2015} and~\cite{visser2013wage}.} The emergence of both similarities and differences in the role and scope of firm-level bargaining across the countries allows to sketch hypotheses about heterogeneous, as opposed to common effects of firm-level bargaining on within-firm inequality that we could expect to observe in the empirical analysis across national contexts.

On the one hand, notwithstanding the general trend towards devolution of bargaining levels in the period under analysis, countries arguably show significant differences in terms of the scope, coverage and extent of derogation of firm-level bargaining vis-à-vis centralised bargaining. This suggests that, although the sign of the relationship between firm-level bargaining and within-firm wage inequality is theoretically uncertain overall, the institutional setting in some countries may favour inequality-enhancing effects of firm-level agreements more than in others. In particular, the national systems of the UK and the Czech Republic are more likely to result in firms bargaining locally having more unequal wage structures, compared to firms that only bargain at more centralised levels. Conversely, more egalitarian outcomes associated to firm-level contracts seem likely to be in place in Germany and Spain, whereas no effect is predicted to emerge in Belgium and outcomes are uncertain in France. Our choice to perform separate analyses by country is exactly intended to verify such potentially heterogeneous effects.

At the same time, on the other hand, we also observe elements of broad similarities, across at least some of the countries, in particular concerning the prevailing locus of collective bargaining. Indeed, in line with~\cite{fulton.2013}, the countries can be assigned to different “bargaining regimes”: Belgium is an emblematic example of the “inter-industry/national regime”; the UK and the Czech Republic represent instances of an opposite “individual-employer regime”; Spain and Germany falls into the intermediate “sectoral regime”; and France is outlying due the specifically complex interaction across all bargaining levels. Accordingly, one could expect that countries sharing the same regime should show similar effects of firm-level agreements on inequality, possibly more similar than country-specific institutional features alone would predict. It is however difficult to predict whether egalitarian or inequality-enhancing pressures should prevail in economies (such as the UK and the Czech Republic) where firm-level bargaining has always been commonplace, or in countries (like Germany, France, Belgium and Spain) that traditionally favour more centralised forms of bargaining. In fact, it may be argued that where more centralised or complex models prevail -- like in Belgium, Spain, Germany and France-- there is stronger resistance (by the laws or due to workers' action) to allow for firm-level contracts to introduce inequalities in the firm internal wage structure. This would reduce the likelihood that in these countries, compared to the UK or the Czech Republic, firms bargaining locally show more unequal wage structures -- if at all -- than across firms bargaining at higher levels. Conversely, one could also argue that in the same countries, firm-level bargaining is used more markedly by firms to differentiate their pay structures, precisely to escape the rigidity and complexities of negotiation typical of more centralised, corporatist regimes where firms have limited margins of manoeuvre to shape internal incentives. If this second tendency prevails, we could expect firm-level bargaining firms to display higher within-firm inequalities than other firms in Spain, Germany, France or Belgium, than in the UK or the Czech Republic.

And of course, the possibility remains that, if the ongoing convergence of wage-setting institutions had completely blurred the institutional differences across countries, or the borders across regimes~\citep[see][]{baccaro2017trajectories}, then any predicted cross-country or cross-regime heterogeneity of the effect of firm-level bargaining may eventually lose significance. If this were the case, our empirical analysis should reveal a complete uniformity in the estimated effect of firm-level bargaining across countries. 


\section{Data and main variables}
\label{sec:data}

\subsection{Data source and sample}
The \emph{Structure of Earnings Survey}~(SES) dataset collected by Eurostat is an established source of information for labour dynamics across Europe, indeed repeatedly used in previous studies. It collects a rich set of earnings-related, personal and jobs-related variables for a vast set of workers, matched with information on some characteristics of the employing firms. 

For this study, we had access to the 2006 and 2010 waves of the SES for Belgium, Germany, Spain, France, the Czech Republic and the United Kingdom. We pool the two waves of the survey in the empirical analysis, but the pooled data must be intended as a repeated cross-section, since the SES does not report any identification code that can be used to match the same firm or the same employee over time.

The structure of the SES data is such that, for each country, a random sample of firms (stratified by size, sector of activity and geographical location) is selected to be representative of the national industrial system. Then, within each selected firm, a representative sample of employees is drawn, and for those employees a large set of personal and job-related characteristics is provided, including wages, age, gender, education, type of contract, tenure, occupation type, and others. As such, the available dataset can be seen as a matched employer-employee dataset, representing a unique source for a consistent comparison across European economies. It has been in fact used several times in the literature, especially within the vast literature on between-firm inequality and its determinants. Early national editions of the survey, covering the 1990s, were used in the few existing studies which examined in our very research question about devolution of bargaining level and within-firm inequality (see Section~2).

Of course, SES data have their own limitations. First, the sample of business units considered in SES is restricted to those with at least 10 employees, which limits the analysis as far as micro firms are concerned. Second, while the surveying procedure provides information on an impressive number of workers across Europe (about 10 million per survey year), for the firms which enter the data the sampling rate of employees varies by firm size and by country. Since measuring within-firm wage inequality requires, by definition, to observe wages for at least two employees of the same firm, we define our working sample as including only firms with at least three sampled employees. Third, the data are very rich concerning employees' personal and work-related characteristics, but the information on firms is limited to five variables: size class, geographical location, sector of activity, public vs.\ private control and -- crucial for our purposes -- the level of wage bargaining adopted in the firm. Fourth, as mentioned, the survey does not allow to identify the same worker across waves. Thus, although the rich set of individual characteristics should cure a great deal of omitted variable bias in estimating residual wages (see Section~3 below for details), one cannot build a panel dataset following employees over time, explicitly controlling for unobserved workers characteristics. 


\subsection{Identification of collective bargaining levels across firms}
\label{sec:decentralisation}
The information provided in SES which is central to our purposes is the type of wage bargaining in place at each firm. Across the countries in our study, the data allow to distinguish three broad cases of collective bargaining coverage. The first case includes firms negotiating under~\emph{centralised bargaining} agreements, meaning those classified by EUROSTAT as “national level or inter-confederal agreements”, “industry agreements” or “agreements for individual industries in individual regions” (also called “multi-employer agreements” or “centralised agreements" in the literature). The second case identifies~\emph{firm-level bargaining} firms, those that apply agreements classified by EUROSTAT as “enterprise or single employer agreements” or “agreements applying only to workers in the local unit”, in addition to -- or departing from -- agreements signed at more centralised levels. The third case concerns the lack of any form of collective bargaining at all, which is the norm in market-oriented countries, and may apply to small firms elsewhere too.

Descriptive statistics of the available data (see Table~\ref{tab:summary_stats} in Appendix~A for details), show that centralised wage bargaining is the dominant form of wage-setting in Belgium, Spain, and France. Firm-level agreements in these countries cover~15--20\% of the employees, and between~6--18\% of firms, while a smaller share of firms and employees are not covered by any collective agreement. In Germany, although a majority of companies~(66--73\%) are not covered by collective agreements, these are mostly small in size, and thus represent less than half of employees sampled. A larger share of employees are covered by centralised agreements, while firm-level agreements are comparatively rare, covering around 6\% or firms, and around 5--7\% of employees. In the Czech Republic and United Kingdom, by contrast, a plurality of firms and employees are not covered by collective bargaining at all, consistent with our description of market-oriented regimes in the previous section. Over time, across all countries except the United Kingdom, firm-level agreements covered a larger share of employees in 2010 than in 2006, consistently with the intuition that devolution of bargaining levels became stronger over the observed period.

The focus of this paper is on bargaining decentralisation, which involves a shift from coordinated central bargaining to decentralised firm bargaining, and is a separate phenomenon from the complete lack of collective wage agreements. Therefore, in our empirical analysis we only consider firms that do apply some form of collective bargaining. We define a variable $\mathit{FLB}$ which compare between the firms that apply firm-level bargaining ($\mathit{FLB}=1$) to those that only apply centralised bargaining ($\mathit{FLB}=0$). This is also the most meaningful comparison to make across countries, considering that in Belgium, Spain, and France, nearly all employees are covered by some form of collective agreement.


\subsection{Definition of the within-firm inequality metric}
To build a measure of within-firm wage inequality, we start from the data on hourly compensation of employees reported in SES. However, using these figures directly to compute a measure of wage variability across the employees of the same firm would provide an incorrect estimate of wage inequality, because it would not account for the different workforce composition and corporate characteristics across firms. In fact, in line with an established practice in the literature on wage inequalities~\citep[dating at least since][]{winter.ebmer.1999}, a meaningful comparison of wages across individuals requires to isolate the component of individual wage that is not directly related to the average market compensation of job, personal and other characteristics of otherwise similar individuals.

As our measure of within-firm inequality for each firm $j$, we take the gap between the 90th and 10th percentile log-wage premia of the employees of the firm
\begin{equation}
  \label{eq:def_disp_90_10}
  \Delta w^{90/10}_j=\hat{w}^{90}_j - \hat{w}^{10}_j
\end{equation}
\noindent where $\hat{w}^p_j$ is the $p$-th percentile of the distribution of the residual wage $\hat{w}_{ij}$ obtained for each employee~$i$ of firm~$j$ from the following Mincer-type regression, estimated separately by country and by year:

\begin{equation}
\label{eq:mincer}
\begin{split}
\log \left(W_{ij} \right)
&= b_0 + b_2\,\mathit{tenure}_{ij} + b_3\,\mathit{tenure}_{ij}^2 + b_1\,\mathit{age}_{ij} + b_4\,\mathit{sex}_{ij} + b_5\,\mathit{educ}_{ij} \\
&+ b_6\,\mathit{contract}_{ij} + b_7\,\mathit{part\_time}_{ij} + b_8\,\mathit{occup}_{ij} + b_9\,\mathit{share}_{ij} + \phi \,\mathit{FE}_j + w_{ij} \;\;\;.
\end{split}
\end{equation}
In this Mincer equation, $\log \left(W_{ij} \right)$ is the logarithm of the hourly wage as reported in SES, which is regressed against a standard set of individual and job characteristics of employees (years of tenure, age, gender, ISCED class of education level, type of contract --permanent, temporary, or apprenticeship--, a dummy for part-time contract, ISCO classes of occupations, and the share of full-time working hours) plus a firm fixed-effect $\mathit{FE}_j$. 

Hence, the residual $w_{ij}$ is a wage-premium that captures the deviation of individual-specific wage from the average wage that could be expected for an employee of firm $j$ based on her characteristics and also controlling for firm-specific average wage premium paid in firm $j$, captured by firm fixed-effect $\mathit{FE}_j$. For example, if a firm had a policy of paying exactly a 10\% premium on average market wages to its employees, this firm-level premium would be accounted for by the coefficient~$\phi$, and this firm's wage policy would have no net effect on within-firm wage inequality in Equation~\eqref{eq:def_disp_90_10}, allowing for meaningful comparison across individuals and firms.\footnote{Table~\ref{tab:summary_cpa_wage} in Appendix~A shows averages of $\Delta w^{90/10}$ by country, year and type of bargaining (firm-level vs. centralised only firms).} 

Taking the interdecile range of (residual) wages allows us to examine if firm-level bargaining is used by firms to adjust wages of low-paid vs. high-paid employees. In fact, as we detail in the next section, we can split the effect of firm-level bargaining on the two deciles. Other measures employed in the literature, such as the standard deviation, do not allow to appreciate where increased or reduced inequality stems from.\footnote{In a robustness check, we took the gap between the 10th and the 50th percentile of residual wages as the dependent variable. Results, not reported but available upon request, were broadly in line with the main findings discussed below.} 


\section{Empirical models and estimation strategy}
\label{sec:empirical}
We here present details of the regression models and the estimation strategy that we apply to examine how firm-level bargaining impact within-firm inequality in the different institutional contexts and over time.

In general, estimating the effect of institutional changes like collective bargaining regimes is a complex problem. The fact that firms are allowed to choose between centralised and decentralised bargaining schemes create endogeneity issues, which raise questions of comparability and selection between firms that choose one regime over the other. Ideally, to draw precise causal inferences on bargaining decentralisation, one would like to observe countries with comparable institutions and conditions enacting discrete reforms to their collective bargaining institutions. Absent this ideal experiment, to derive comparable estimates by country, we control for employee and firm characteristics and account for the propensity of individual firms to adopt decentralised bargaining. 

\subsection{Firm-level bargaining and within-firm inequality}
Our empirical setup is intended to estimate whether, in each country, there is a significant gap in within-firm inequality between firms under firm-level bargaining and those under centralised bargaining, and whether this gap varies by country and over time. We estimate separately by country the following regression model
\begin{equation}
\label{eq:reg_dispersion}
\begin{split}
  \Delta w^{90/10}_j &= \beta_0 + \beta_1\, \mathit{FLB}_j + \beta_2\, \mathit{Y}^{2010}_j + \beta_3\, \mathit{Y}^{2010}_j \times \mathit{FLB}_j + \gamma\, \widehat{\mathit{FLB}}_j \\
                     &+ \zeta\bm{X}_j + \eta\, \mathit{sector}_j + \theta\, \mathit{region}_j + \epsilon_j \;,
 \end{split}
\end{equation}

\noindent where $\Delta w_j^{90/10}$ is the measure of within-firm wage inequality defined in Equation~\eqref{eq:def_disp_90_10}, computed for each firm $j$ which is sampled either in year $t=2006$ or $t=2010$;
$\mathit{FLB}_j$ indicates if firm~$j$ applies only centralised bargaining ($\mathit{FLB}_j=0$), or firm-level bargaining $\mathit{FLB}_j=1$;
$\mathit{Y}^{2010}_j$ is a dummy indicating if the firm $j$ is sampled in the year 2010; $\bm{X}_j$ is a set of firm characteristics and workforce composition (discussed below); $\mathit{sector}_j$ and $\mathit{region}_j$ are fixed-effects, for economic sector (reported in SES at 1-digit NACE) and geographical location (reported in SES at NUTS-1 level) of the firm; $\widehat{\mathit{FLB}}_j$ is a propensity score representing the probability that firm $j$ adopts firm-level bargaining, included to correct for potential endogenous selection effects; $\epsilon_j$ is an idiosyncratic error term. 

There are four parameters of main interest $\beta$. The intercept $\beta_0$ measures the baseline level of within-firm inequality, among firms under centralised bargaining that were sampled in 2006. The coefficient $\beta_1$ captures the gap in within-firm inequality between firm-level bargaining firms and fully centralised bargaining firms in 2006. Then, $\beta_2$ measures the change in baseline inequality that occurred over time between 2006 and 2010 for firms under centralised bargaining, while $\beta_3$ captures the additional growth in inequality between 2006 and 2010 for firms under firm-level bargaining. Separate estimates of the model (and thus of these four key parameters) by country allow to properly account for differences in bargaining systems across countries. In fact, as discussed above in introducing the FLB dummy, the definition of firm-level bargaining firms ($\mathit{FLB}=1$) is relatively homogeneous in SES across all countries, while there is greater variation across countries about the prevailing alternative bargaining system for the control group of firms that do not apply firm-level bargaining ($\mathit{FLB}=0$).

The identification of the key parameters proceed as follows. First of all, the inclusion of sector and regional fixed-effects, together with the firm-level controls in $\bm{X}_j$, accounts for factors that jointly determine inequality and adoption of firm-level bargaining, potentially creating an omitted variables bias if not included in the regression model.\footnote{Controlling for sector fixed-effects, as is especially important, to account for the possibility~\citep[put forward in][]{bechter2012sectors,hassel2014paradox} that industrial relations may be primarily driven by cross-national tendencies specific to industrial sectors, playing a role above and beyond country-specific institutional settings.} The vector $\bm{X}_j$ covers in particular two groups of variables available from SES for each firm $j$. The first are corporate characteristics: a size class (by number of employees), and a dummy for private vs.\ public control on the firm. The expectation is that within-firm wage dispersion is lower in large and publicly owned firms, as the unions tend to be more powerful in these contexts~\citep{canaldominguez.gutierrez.2004}. The second group covers the workforce characteristics of firm $j$, highlighted in previous studies as determinants of wage inequality. For every firm, we measure the share of women employed in the firm; a set of dummies for modal age of the workforce; the share of employees with secondary or tertiary education; the mean tenure of workers in the firm; the share of managers and professionals (according to 1-digit ISCO codes 1 and 2); the share of part-time employees; and the share of employees with a permanent contract. 

Although these controls are relevant in theory, their individual relationship with within-firm inequality is difficult to predict in isolation. Usually, within-firm wage differences are expected to rise with age, tenure and education, because wages tend to increase in all these characteristics, and dispersion is usually higher in firms where average wages are higher~\citep{canaldominguez.gutierrez.2004}. As for gender, the well-documented existence of female wage-gaps would predict wider inequality in firms where the proportion of women is lower. Also, earnings inequalities are expected to be lower in firms having a relatively larger proportion of full-time (vis-à-vis fixed-term), permanent (vis-à-vis part-time), blue-collar (vis-à-vis white-collar) workers, since these types of employees are generally more likely to unionise, and thus their firms to be more affected by unions' efforts to push for equalization of wages among members~\citep{canaldominguez.gutierrez.2004}.\footnote{Basic descriptive statistics on control variables are presented in Tables~A2-A4 in Appendix~A. Notice that some of the controls are not available for the Czech Republic. First, in the data there are no Czech firms with modal employees' age in the range 20-29 years old, so we omit this age category. Second, the Czech Republic defines a single NUTS-1 region, so we cannot further exploit regional dummies in the estimates for this country.}

In addition to including fixed-effects and firm-level controls, in estimating Equation~\ref{eq:reg_dispersion} we also address the potential endogeneity of the FLB dummy that may arise due to non-random selection of firms that apply firm-level bargaining or not. Indeed, there might still be unobserved determinants of the decision to apply firm-level collective agreements that correlate with unobserved determinants of the dependent variable of interest. This may occur despite controlling for employer-specific components of wages and firm-level average wages through the preliminary Mincerian regression, and despite the inclusion of the extensive set of firm-level covariates. Following a solution adopted in the empirical literature~\citep{card.delarica.2006,daouli.demoussis.ea.2013}, we tackle this possible source of bias by augmenting the model with a preliminary Probit estimate of the probability (propensity score) that a given firm adopts firm-level collective bargaining. The overall rationale is that if FLB status is as good as randomly assigned conditional on observed controls, then conditioning also upon the propensity scores allows to clean any further bias due to unobserved firm characteristics (see Appendix~B for details).

\subsection{Firm-level bargaining and top vs.~bottom wage premia}
To better understand the estimates derived from Equation~\eqref{eq:reg_dispersion}, we also explore the relation between firm-level bargaining and the two components of the wage-dispersion measure $\Delta w^{90/10}$. We estimate the following variation of Equation~\eqref{eq:reg_dispersion}:

\begin{equation}
\label{eq:reg_decomposition}
\begin{split}
  \Delta w^{p}_j &= \beta_0 + \beta_1\, \mathit{FLB}_j + \beta_2\, \mathit{Y}^{2010}_j + \beta_3\, \mathit{Y}^{2010}_j \times \mathit{FLB}_j + \gamma\, \widehat{\mathit{FLB}}_j \\
                     &+ \zeta\bm{X}_j + \eta\, \mathit{sector}_j + \theta\, \mathit{region}_j  + \epsilon_j \;,
 \end{split}
\end{equation}
\noindent where as dependent variable $w^p_j$ we employ, alternatively, the 90th or the 10th percentile of the distribution of log-wage residuals in firm $j$, estimated as earlier described in Equation~\eqref{eq:mincer}.

By separating the effects of firm-level bargaining on the components $w^{90}$ and $w^{10}$, this estimation provides hints about where the overall effects observed in Equation~\eqref{eq:reg_dispersion} stem from. In fact, suppose that a positive association between firm-level bargaining and $\Delta w^{90/10}$ is found for a given country from estimates of Equation~\eqref{eq:reg_dispersion}. Then, it is crucial to understand if this stems from wage premia of top-paid employees ($w^{90}$) being higher under firm-level bargaining than in other firms, or from wage premia of bottom-paid workers ($w^{10}$) being lower in FLB firms, or from both effects being in place at the same time. Also, in case no statistical difference in overall inequality $\Delta w^{90/10}$ emerged between firm-level bargaining and other firms, Equation~\eqref{eq:reg_decomposition} could tell us if this is due to the two components being offset in the same direction under firm-level bargaining. 

Estimation of Equation~\eqref{eq:reg_decomposition} follows the same strategy employed to estimate the baseline model in~Equation~\eqref{eq:reg_dispersion}. We perform separate regressions by country, augmented with the same set of firm-level covariates and fixed-effects, plus preliminary first-step Probit estimates of firm-specific FLB propensity scores. The estimates of $\beta_1$ on the $\mathit{FLB}$ dummy give the difference in average outcomes across firm-level bargaining vs.~other firms in 2006, whereas the coefficient $\beta_3$ on the interaction term $\mathit{FLB} \times Y^{2010}$ accounts for changes in the FLB effect over time.

\section{Results}
\label{sec:results}

Table~\ref{tab:disp_90_10} present the estimates of Equation~\eqref{eq:reg_dispersion}, looking at the difference in overall within-firm wage inequality between firms that apply firm-level bargaining and centralised bargaining.

There is a substantial difference in the baseline level of inequality across countries, captured by the intercept $\beta_0$, which estimates the average within-firm difference in between 90th and 10th percentile of log-wage premia for companies under centralised bargaining in 2006. The lowest level of inequality is in Spain, where the baseline difference is estimated at $0.299$, or approximately 30 percentage points. France has nearly double the highest baseline inequality, of around 62 percentage points, while most other countries are around 45--55 percentage points.

In 2006, there was mostly no difference in inequality between firms that adopted firm-level bargaining and those under centralised bargaining. Indeed, the estimated  coefficient $\beta_1$ are not statistically different from zero in all countries, except in the UK. In this case, firm-level bargaining firms show a statistically significant less unequal wage distribution compared to other firms, by about 1.3 percentage points, or around 2\% of baseline inequality.

By 2010, within-firm wage inequality in firms under centralised bargaining dropped compared to 2006 (cf. $\beta_2$), in all countries except Germany. For some countries, the change was quite sizeable, compared to baseline inequality in 2006: around 11\% in Spain, and 13\% in the UK. While this trend of reduction was equally characterising firm-level bargaining and other firms in most countries, in Spain and France this reduction in inequality was instead dampened (or reversed) in firms under firm-level bargaining (cf. $\beta_3$). In both countries, whereas in 2006 there was no significant difference between companies that applied centralised and firm-level bargaining, in 2010 there is a widening gap (around 2.2 percentage points in Spain, 3.6 in France). On balance, while within-firm inequality reduced appreciably in 2010 for Spanish and French firms under centralised bargaining, it narrowed substantially less for companies under firm-level bargaining in Spain (summing $\beta_2$ and $\beta_3$) and even increased for those in France. 

The estimates reveal heterogeneity also with regard to the correlation between the $\Delta w^{90/10}$ wage-gap and control variables. Starting from corporate characteristics, wage dispersion within firms increases with firm size in Germany, Spain and France, but larger firms display lower wage dispersion than the baseline in the UK. Publicly-controlled firms feature lower wage dispersion compared to private firms in Belgium, the Czech Republic and France. Further, moving to workforce characteristics, the modal age of employees shows mostly an insignificant association with wage inequality in Belgium, Germany, Spain and France, while the relation with $\Delta w^{90/10}$ is stronger (positive) in the Czech Republic and the UK. A common result across all countries is that within-firm wage inequality is larger for firms with the most senior workforce (60+ years old). The share of women in the workforce, the average on-the-job tenure of employees and the share of permanent contracts show a negative association with within-firm wage dispersion in most countries, while educational levels, the share of part-time employees and the share of higher professional occupations tend to display a positive association (when significant) with within-firm wage dispersion. Note, lastly, that the significant coefficient on the propensity score $\widehat{\mathit{FLB}}_j$ confirms the need to correct for endogenous selection into $\mathit{FLB}$ in most countries.

Estimates of the separate effects of firm-level bargaining on the 90th and 10th percentiles of wage premia, in Table~\ref{tab:disp_avg_90_10}, are particularly revealing of the dynamics underlying the results observed above for UK, France and Spain. The lower inequality for firm-level bargaining firms emerged in 2006 for the UK appears as due to sensibly higher wage premia paid at the bottom of internal wage distribution in firm-level bargaining firms compared to centralised bargaining firms (see $\beta_1$ estimated for $q_{10}$). In Spain, the reduction of overall wage inequality $\Delta w^{90/10}$ in 2010 observed earlier for firms under centralised bargaining, results from a relative reduction of the 90th percentile wage premia and a corresponding increase of those at the 10th percentile, leading to overall wage reduction by both extremes moving closer.
Crucially, this reduction in inequality was dampened for both $q_{90}$ and $q_{10}$ under firm-level bargaining (see coefficient $\beta_3$), leading to a total 2.17 percentage point difference between the two regimes in 2010. France experienced a more extreme version of the same dynamics: the wage premia at the top increased and those at the bottom decreased under FLB, leading to an overall difference of 3.6 percentage points.

\section{Conclusion}
\label{sec:conclusion}
The impact of collective pay agreements on wage inequality between firms is well-documented. However, there is less evidence on whether wage-setting happening at the level of firms -- as opposed to more centralised bargaining levels -- can explain wage differences emerging within the firm. 

Exploiting matched employer-employee data for six European countries over 2006 and 2010, in this work we contribute to advance the existing literature by addressing three questions. First, is firm-level bargaining generally associated to higher or lower within-firm inequality? Second, have these relations remained stable or changed over the years under study, when a broad process of increasing emphasis on decentralisation of wage bargaining took place and the Great Depression hit? Third, are there patterns common to all or at least to some of the selected countries, mapping into broad bargaining regimes?

A-priori, effects are difficult to predict, because of countervailing trends. On the one hand, by allowing firms more flexibility than higher level negotiations, firm-level agreements may induce an increase in within-firm inequality, if they are used to selectively provide incentives or rewards to specific employees or groups of employees. On the other hand, firm-level agreements may reduce inequalities within firms if they respond to fairness, egalitarian or redistributive motives. The balance between these contrasting forces may, in turn, depend on the institutional context, according to the changing scope of firm-level bargaining in different national frameworks and over time.

Considering the specific features of wage bargaining systems in place in the countries under study over the decade covered by our data, we could have expected higher inequality under firm-level bargaining in the UK and the Czech Republic, consistently with the wage setting in place in these countries, traditionally favouring selective use of firm-level pay schemes. In other countries, the national context made us predict that in Spain and Germany firm-level would be associated with less inequality, but our predictions were overall more uncertain, due to a non-trivial combination of country-specific and regime-specific factors typical of countries like Belgium, Germany, Spain and France. In fact, firm-level bargaining could increase within-firm inequality, to the extent that firm-level negotiations are used to gain flexibility over the complexity of negotiations and standardisation of wages typical of these countries' regimes. But the same egalitarian pressures could be so strong as to justify the hypothesis that firm-level bargaining reduced inequality or had no effect in those countries. Lastly, we also put forward the hypothesis that, if anything, the inequality-enhancing effects of firm-level bargaining could have been stronger in 2010 in all countries, in line with a general trend toward more and more devolution of bargaining levels occurring over time.

Our results only partially match with these predictions, however. First, they confirm that the process of devolution of bargaining level started in all countries in the 90s did not go so far as to completely blur cross-country differences in the use and effect of firm-level negotiations. We indeed find that firms bargaining locally may have similar levels of inequality than those under centralised bargaining, but may also have higher or lower inequality as well. This can change over time, even within the same country. In 2006, firms bargaining locally were less unequal than under centralised bargaining in the UK, while no difference emerged in the other countries. Decomposition of wage premia reveals that the result for the UK is driven by low-paid workers being paid more in firm-level bargaining firms than under centralised bargaining. By 2010, it is only in Spain and France that we observe a divergence in the trend of inequality between firms under centralised and firm-level bargaining firms, with the latter ending up as more unequal in 2010. In both countries, while the top wage premia dropped and bottom wage premia rose in companies under centralised bargaining, this movement either didn't occur or was reversed under firm-level bargaining, suggesting that firm-level negotiations were increasingly used to escape standardisation of wages in these two countries. 

At the same time, the heterogeneity of effects estimated across countries do not map neatly into country-specific features of national bargaining systems, or into broad classification of countries based on prevailing bargaining levels. On the one hand, we could have anticipated the insignificant results obtained for Belgium, and also the significant effects estimated for the UK, Spain and France. However, sets of countries that shared similar institutional characteristics turned out to be different: the UK versus the Czech Republic, or Germany and Belgium versus France and Spain. And, conversely, the same results emerged for countries with a-priori quite diverse collective bargaining institutions and traditions, like Germany and the Czech-Republic. Ultimately, our findings provide an average picture about how countervailing firm-specific drivers of the use of firm-level negotiations, such as incentive motives, inertia and conflicts of power, are resolved. While more detailed data on these factors and perhaps specific case study of single firms could help highlighting the underlying dynamics, our estimates support that these processes do not systematically relate to a peculiar prevailing regime. Actually, a possible avenue for future research, complementary to our study, would be to investigate the existence of common patterns in the use of firm-level bargaining across sectors, instead of across countries. If international sectoral patterns mattered more than country-specific ones~\citep[as suggested in ][]{bechter2012sectors}, we should envisage different uses -- more or less inequality-enhancing -- of firm-level contracts in some sectors than others, across countries. Although we control for this possible trend via sector fixed-effects, direct tackling this hypothesis would lead to clustering the analysis by sectors instead of by countries. 

Further, the availability of matched employer-employee panel data --following the same workers and firms continuously for many years-- would allow to evaluate more precisely if inequality-enhancing vs.~redistributive uses of firm-level agreements are more likely during upswing or downswing of the business cycle, giving firms more flexibility and discretionality to reshape the wage ladder in case of need. This appears to be even more relevant in more recent years in a context of rapidly changing work, where firms are reshaping their organisational structures in relation to processes of digitalization~\citep{OECD2019} and the locus of bargaining is likely to affect how productivity gains will be distributed within-firms.

Overall, our study offers new evidence and methods to inform the renewed debate on the determinants of wage inequality. We highlight the importance of the locus of collective wage bargaining and show that firm-level bargaining can act not only as a driver of wage inequality between firms, but also within them.


% End notes
\clearpage

\clearpage
\singlespacing
\bibliography{biblio_CBWD}

\clearpage

\begin{landscape}
\centering
\begin{table}[htb]
\caption{Collective bargaining across EU countries}
\label{tab:CBTable}
\centering
%!TEX root = ../ILR/revision.tex


  %\resizebox{\textwidth}{!}{%
  \tiny
       \begin{tabular}{p{11.22em}p{12.28em}p{12.28em}p{12.28em}p{14em}p{10em}p{12.28em}}
    \toprule
    & \textbf{Belgium} & \textbf{Germany} & \textbf{Spain} & \textbf{France} & \textbf{The Czech Republic} & \textbf{The United Kingdom} \\
    \midrule
    \textbf{Level of contractual negotiations} & Predominantly at the national, cross-industry level & Industry level between  trade unions and employers' organisations; agreements allow for flexibility at the company level & Predominant role of industry-level but interaction of negotiations at national and province-level, within industries & Peculiarly complex system of industrial relations:  all the levels of collective negotiations (inter-sectoral, industry or Firm-level) are closely intertwined and, in turn, they occur at both national or local level & Uncoordinated wage setting occurring directly between  firms and individuals. Principal Level of Collective Bargaining: company & Wage bargaining is mostly uncoordinated, with most workers bargaining work contracts individually with employers \\
    \midrule
    \textbf{Collective bargaining (\% firms) and Collective bargaining coverage rate (\% employees) } & 66.08\% of companies apply a collective agreement negotiated at higher level than the establishment or the company (in 2009); Collective bargaining coverage rate is 96\% in 2009 (ILO) &  Share of companies covered by forms of collective agreement above firm-level bargaining is around 66.92\% in 2009; Collective bargaining coverage rate is 61.7\% in 2009 (ILO) & Share of firms reporting to negotiate wages outside firms is 66.09\% in 2009; Collective bargaining coverage rate is 80.9\% in 2009 (ILO) & More than 50\% of companies declare to apply centralised bargaining in 2009; Collective bargaining coverage rate is 98\% in 2009 & 80\% of companies  conduct negotiations of wages at the firm or the establishment level; Collective bargaining coverage rate is 35\% in 2009 & 53.4\% of companies sign a firm-level agreement (in 2009); Collective bargaining coverage rate is 32.7\% in 2009 (ILO) \\
    \midrule
    \textbf{Topics covered by collective agreements} & Elements of pay and work conditions including national minimum wage, job creation measures, training and childcare provision set at the national level; industry and company bargaining mostly address non-pay issues & Wide range of issues such as pay, shift-work payments, pay structures, working time, treatment of part-timers and training & The national agreements covering the whole economy deal with non-pay issues such as training, equality and remote working, and since 2002 have, in a series of three-year deals, set broad guidelines on pay increases.\newline{}Lower-level agreements normally cover pay and working time. & Wide range of issues,  industry-level negotiation is obligatory in: pay; equality between women and men and measures to tackle the inequalities identified; working conditions, staffing and career development and exposure to occupational risks; disabled workers; occupational training; job classification; employee saving schemes; and arrangements for organising part-time work & Pay is the main subject of collective bargaining although there are also negotiations on other issues such as working time, work organisation, health and safety, work-life balance and employers’ contributions to pensions & Some negotiations cover all aspects of pay and conditions, but others are limited to only a few areas, principally pay \\
    \midrule
    \textbf{Derogation clauses} & Opening clauses dealing with wages appeared in sector-level agreements. Scarcely used in practice, covering six (sub)sectors: engineering; metal manufacturing; food manufacturing; retail of food products; large retail stores; department stores & Favourability principle prevents firm level agreements to set less favourable terms than those provided in agreements stipulated at higher levels. Wage derogations allowed at company level in times of serious economic difficulties and also in times of more general competitive problems & A company agreement might depart from the wages fixed by a collective agreement negotiated at a higher level, when, as a result of the application of those wages, the economic situation and prospects of the company could be damaged and affect jobs & The inversion of the favourability principle  introduced in 2004, recognising to firm-level agreements the possibility to derogate from any condition settled at more centralised levels, if not explicitly prohibited.  Four major issues are exempted from any derogation at company level: minimum wages; job classifications; supplementary social protection measures; multi-company and cross-sector vocational training funds. & Legal provision of the favourability principle prevents firm level agreements to set less favourable terms than those provided in agreements stipulated at higher levels & Agreements do not establish legally binding norms and, as a rule, they contain no contractual obligations, they are not subject to legal regulation, and pay rates cannot be claimed in court \\
    \bottomrule
    \end{tabular}%
 %  }



\end{table}
\end{landscape}

\section*{Regression Tables}
\label{sec:tables}


\begin{table}[hbt]
\caption{90th-10th percentile within-firm inequality and Firm-level bargaining (FLB)}
\label{tab:disp_90_10}
\centering
\resizebox{\textwidth}{!}{%
\begin{threeparttable}
%\setlength{\tabcolsep}{0pt}
%
\begin{tabular}{l*{6}{S}}
\toprule
%                         &  (1)         & (2)         & (3)        & (4)         & (5)        & (6)          \\
& \multicolumn{1}{c}{BE} & \multicolumn{1}{c}{DE} & \multicolumn{1}{c}{ES} %
& \multicolumn{1}{c}{CZ} & \multicolumn{1}{c}{UK} & \multicolumn{1}{c}{FR} \\
\midrule
$\beta_0$: Intercept                         &       0.455***    & 0.529***    & 0.299***   & 0.463***    & 0.596***   & 0.624***     \\
$\hookrightarrow$ \textit{Base inequality (FLB=0 in 2006)} &       (0.0340)    & (0.0539)    & (0.0259)   & (0.0427)    & (0.122)    & (0.0401)     \\[1ex]
$\beta_1$: FLB                              &       -0.00158    & -0.00285    & 0.00510    & -0.0103     & -0.0129**  & -0.00478     \\
$\hookrightarrow$ \textit{Additional ineq. of FLB=1 in 2006} &       (0.00420)   & (0.00638)   & (0.00458)  & (0.0100)    & (0.00634)  & (0.00990)    \\[1ex]
$\beta_2$: Year 2010                        &       -0.0334***  & 0.00454     & -0.0326*** & -0.0142     & -0.0798*** & -0.0151***   \\
$\hookrightarrow$ \textit{Add. ineq. in 2010}        &       (0.00265)   & (0.00424)   & (0.00273)  & (0.0119)    & (0.00842)  & (0.00348)    \\[1ex]
$\beta_3$: FLB$\times$2010                  &       0.00148     & 0.00323     & 0.0217***  & 0.00496     & -0.00420   & 0.0362***    \\
$\hookrightarrow$ \textit{Add. ineq. of FLB in 2010} &       (0.00592)   & (0.00846)   & (0.00672)  & (0.0129)    & (0.00849)  & (0.0114)     \\[1ex]
$\gamma$: Prob. FLB                         &       0.0953***   & -0.000383   & -0.226***  & 0.117**     & -0.00772   & 0.105***     \\
$\hookrightarrow$ \textit{Add. ineq. of predicted FLB status}                                            &       (0.0355)    & (0.0520)    & (0.0250)   & (0.0458)    & (0.125)    & (0.0359)     \\[1ex]
\cmidrule(lr){1-7}
Modal age workers:                          \\[1ex]
\quad \textit{20-29}                        &       -0.0297     & 0.0330      & 0.0101     &             & 0.00383    & -0.0654*     \\
                                            &       (0.0242)    & (0.0455)    & (0.0242)   &             & (0.0127)   & (0.0369)     \\[1ex]
\quad \textit{30-39}                        &       -0.0133     & 0.0204      & 0.0264     & 0.0363***   & 0.0378***  & -0.0557      \\
                                            &       (0.0244)    & (0.0446)    & (0.0244)   & (0.00985)   & (0.0127)   & (0.0370)     \\[1ex]
\quad \textit{40-49}                        &       -0.00313    & 0.0156      & 0.0177     & 0.0120      & 0.0466***  & -0.0577      \\
                                            &       (0.0242)    & (0.0446)    & (0.0243)   & (0.0119)    & (0.0131)   & (0.0360)     \\[1ex]
\quad \textit{50-59}                        &       0.0197      & 0.0124      & 0.0157     & 0.0152      & 0.0470***  & -0.0389      \\
                                            &       (0.0250)    & (0.0451)    & (0.0248)   & (0.0103)    & (0.0139)   & (0.0366)     \\[1ex]
\quad \textit{60+}                          &       0.0322      & 0.103*      & 0.0785***  & 0.0955**    & 0.0309*    & -0.0315      \\
                                            &       (0.0311)    & (0.0572)    & (0.0284)   & (0.0381)    & (0.0166)   & (0.0404)     \\[1ex]
\% of women empl.                           &       -0.0616***  & -0.0490***  & -0.0392*** & -0.0230     & -0.0386*** & -0.0451***   \\
                                            &       (0.00599)   & (0.0119)    & (0.00467)  & (0.0157)    & (0.00986)  & (0.00603)    \\[1ex]
Mean experience empl.                       &       -0.00216*** & -0.00245*** & 0.00344*** & -0.00468*** & -5.36e-05  & -0.000761*** \\
                                            &       (0.000392)  & (0.000522)  & (0.000360) & (0.000980)  & (0.000576) & (0.000287)   \\[1ex]
\% empl. with tert. educ.                   &       0.116***    & 0.0847***   & 0.165***   & 0.249***    & 0.103***   & 0.0799***    \\
                                            &       (0.00836)   & (0.0272)    & (0.00634)  & (0.0386)    & (0.0159)   & (0.00814)    \\[1ex]
\% empl. with sec. educ.                    &       0.0184***   & 0.0532***   & 0.0730***  & 0.0206      & 0.0592***  & 0.00665      \\
                                            &       (0.00498)   & (0.0185)    & (0.00480)  & (0.0275)    & (0.0134)   & (0.00802)    \\[1ex]
\% managers and profess.                    &       0.0931***   & 0.0688***   & 0.0745***  & 0.121***    & 0.251***   & 0.148***     \\
                                            &       (0.00971)   & (0.0204)    & (0.00949)  & (0.0260)    & (0.0111)   & (0.00853)    \\[1ex]
\% part-time empl.                          &       -0.0108     & 0.140***    & 0.109***   & 0.175***    & 0.0419***  & 0.00672      \\
                                            &       (0.00773)   & (0.0126)    & (0.00664)  & (0.0529)    & (0.0108)   & (0.00832)    \\[1ex]
\% permanent contracts                      &       -0.0781***  & -0.0861***  & -0.00416   & -0.00757    & -0.0464**  & -0.160***    \\
                                            &       (0.0101)    & (0.0184)    & (0.00509)  & (0.0173)    & (0.0190)   & (0.0136)     \\[1ex]

Firm size:                                  \\[1ex]
\quad \textit{50--249 empl.}                &       0.000931    & 0.0361***   & 0.121***   & 0.0163      & -0.0631*** & 0.0378***    \\
                                            &       (0.00562)   & (0.00545)   & (0.00357)  & (0.0108)    & (0.0131)   & (0.00429)    \\[1ex]
\quad \textit{$\geq$ 250 empl.}             &       -0.00323    & 0.0340***   & 0.193***   & 0.0112      & -0.0652*** & 0.0490***    \\
                                            &       (0.00967)   & (0.00513)   & (0.00641)  & (0.0164)    & (0.0117)   & (0.00490)    \\[1ex]
Public firm                                 &       -0.0502***  & -0.00360    & 0.0287     & -0.0797***  & 0.0194     & -0.0694***   \\
                                            &       (0.0119)    & (0.0140)    & (0.00823)  & (0.0116)    & (0.0598)   & (0.00937)    \\[1ex]
\midrule
Observations                                &       13,765      & 12,312      & 37,887     & 3,498       & 14,502     & 30,009       \\
R-squared                                   &       0.187       & 0.064       & 0.197      & 0.230       & 0.123      & 0.118        \\
Region FE                                   &       \checkmark  & \checkmark  & \checkmark & \checkmark  & \checkmark & \checkmark   \\
Sector FE                                   &       \checkmark  & \checkmark  & \checkmark & \checkmark  & \checkmark & \checkmark   \\
\bottomrule%
\end{tabular}
%
\begin{tablenotes}
\item Notes: Bootsrapped standard errors in parentheses (200 repetitions);
\item Asterisks denote significance levels: $^{*}$ p$<$0.05, $^{**}$ p$<$0.01, $^{***}$ p$<$0.001
\end{tablenotes}
%
\setlength{\tabcolsep}{6pt}
%
\end{threeparttable}
}

\end{table}

\clearpage


\begin{landscape}
\centering
\begin{table}[htb]
\caption{Decomposition 90th and 10th percentiles and Firm-level bargaining (FLB)}
\label{tab:disp_avg_90_10}
\centering
\centering
\resizebox{\textwidth}{!}{%
\begin{threeparttable}
% \setlength{\tabcolsep}{0pt}
%
\begin{tabular}{l*{12}{S}}
\toprule
                          % & (1)         & (2)         & (3)          & (4)        & (5)        & (6)         & (7)         & (8)        & (9)        & (10)       & (11)         & (12)       \\
                          & \multicolumn{2}{c}{BE} & \multicolumn{2}{c}{DE} & \multicolumn{2}{c}{ES} & \multicolumn{2}{c}{CZ} & \multicolumn{2}{c}{UK} & \multicolumn{2}{c}{FR} \\
 & \multicolumn{1}{c}{q\_90} & \multicolumn{1}{c}{q\_10} & \multicolumn{1}{c}{q\_90} & \multicolumn{1}{c}{q\_10} & \multicolumn{1}{c}{q\_90} & \multicolumn{1}{c}{q\_10} %
                          & \multicolumn{1}{c}{q\_90} & \multicolumn{1}{c}{q\_10} & \multicolumn{1}{c}{q\_90} & \multicolumn{1}{c}{q\_10} & \multicolumn{1}{c}{q\_90} & \multicolumn{1}{c}{q\_10} \\

\midrule
$\beta_0$: Intercept             &       0.226***    & -0.228***   & 0.270***     & -0.259***  & 0.148***   & -0.151***   & 0.244***    & -0.219***  & 0.307***   & -0.290***  & 0.315***     & -0.309***  \\
                                &       (0.0203)    & (0.0173)    & (0.0289)     & (0.0265)   & (0.0134)   & (0.0112)    & (0.0223)    & (0.0195)   & (0.0722)   & (0.0571)   & (0.0234)     & (0.0173)   \\[1ex]
$\beta_1$: FLB                  &       -0.00106    & 0.000520    & -0.00945***  & -0.00660*  & -0.000250  & -0.00535**  & -0.00479    & 0.00548    & -0.00414   & 0.00877*** & -0.00187     & 0.00292    \\
                                &       (0.00276)   & (0.00244)   & (0.00343)    & (0.00394)  & (0.00260)  & (0.00239)   & (0.00521)   & (0.00535)  & (0.00373)  & (0.00320)  & (0.00500)    & (0.00601)  \\[1ex]
$\beta_2$: Year 2010            &       -0.0189***  & 0.0145***   & 0.00678***   & 0.00223    & -0.0174*** & 0.0151***   & -0.00548    & 0.00871    & -0.0392*** & 0.0406***  & -0.00764***  & 0.00741*** \\
                                &       (0.00158)   & (0.00135)   & (0.00236)    & (0.00262)  & (0.00139)  & (0.00131)   & (0.00623)   & (0.00728)  & (0.00530)  & (0.00419)  & (0.00202)    & (0.00162)  \\[1ex]
$\beta_3$: FLB$\times$2010      &       0.00159     & 0.000108    & 0.00578      & 0.00255    & 0.0148***  & -0.00693**  & 0.00125     & -0.00371   & -0.00431   & -0.000113  & 0.0195***    & -0.0167*** \\
                                &       (0.00332)   & (0.00298)   & (0.00451)    & (0.00452)  & (0.00346)  & (0.00316)   & (0.00693)   & (0.00771)  & (0.00533)  & (0.00443)  & (0.00600)    & (0.00598)  \\[1ex]
$\gamma$: Prob. FLB              &       0.0539***   & -0.0415**   & -0.0153      & -0.0149    & -0.133***  & 0.0928***   & 0.0595**    & -0.0576*** & -0.0112    & -0.00350   & 0.0678***    & -0.0374**  \\
                                &       (0.0173)    & (0.0171)    & (0.0238)     & (0.0296)   & (0.0151)   & (0.0129)    & (0.0259)    & (0.0212)   & (0.0739)   & (0.0573)   & (0.0183)     & (0.0167)   \\[1ex]
\cmidrule(lr){1-13}
Modal age workers:             \\[1ex]
\quad \textit{20-29}            &       -0.0141     & 0.0156      & -0.00923     & -0.0423*   & 0.00651    & -0.00356    &             &            & 0.00736    & 0.00353    & -0.0306      & 0.0348**   \\
                                &       (0.0136)    & (0.0117)    & (0.0251)     & (0.0244)   & (0.0131)   & (0.0109)    &             &            & (0.00635)  & (0.00660)  & (0.0211)     & (0.0165)   \\[1ex]
\quad \textit{30-39}            &       -0.00464    & 0.00867     & -0.0135      & -0.0339    & 0.0153     & -0.0111     & 0.0183***   & -0.0180*** & 0.0267***  & -0.0111*   & -0.0250      & 0.0308*    \\
                                &       (0.0138)    & (0.0115)    & (0.0250)     & (0.0239)   & (0.0129)   & (0.0108)    & (0.00523)   & (0.00603)  & (0.00684)  & (0.00663)  & (0.0209)     & (0.0166)   \\[1ex]
\quad \textit{40-49}            &       0.00105     & 0.00418     & -0.0175      & -0.0331    & 0.0108     & -0.00687    & 0.00668     & -0.00533   & 0.0315***  & -0.0151**  & -0.0249      & 0.0329**   \\
                                &       (0.0139)    & (0.0116)    & (0.0249)     & (0.0238)   & (0.0131)   & (0.0108)    & (0.00628)   & (0.00640)  & (0.00635)  & (0.00674)  & (0.0208)     & (0.0166)   \\[1ex]
\quad \textit{50-59}            &       0.0127      & -0.00702    & -0.0167      & -0.0291    & 0.0101     & -0.00561    & 0.00820     & -0.00701   & 0.0332***  & -0.0138*   & -0.0146      & 0.0243     \\
                                &       (0.0141)    & (0.0118)    & (0.0252)     & (0.0237)   & (0.0133)   & (0.0110)    & (0.00538)   & (0.00583)  & (0.00681)  & (0.00723)  & (0.0210)     & (0.0164)   \\[1ex]
\quad \textit{60+}              &       0.0193      & -0.0129     & 0.0232       & -0.0798**  & 0.0437***  & -0.0348***  & 0.0538***   & -0.0417**  & 0.0233***  & -0.00762   & -0.0108      & 0.0206     \\
                                &       (0.0175)    & (0.0160)    & (0.0317)     & (0.0325)   & (0.0147)   & (0.0128)    & (0.0184)    & (0.0176)   & (0.00895)  & (0.00815)  & (0.0224)     & (0.0186)   \\[1ex]
\% of women empl.               &       -0.0294***  & 0.0323***   & -0.0203***   & 0.0287***  & -0.0176*** & 0.0216***   & -0.00670    & 0.0163**   & -0.0215*** & 0.0171***  & -0.0255***   & 0.0196***  \\
                                &       (0.00340)   & (0.00323)   & (0.00576)    & (0.00646)  & (0.00280)  & (0.00247)   & (0.00815)   & (0.00750)  & (0.00520)  & (0.00427)  & (0.00333)    & (0.00321)  \\[1ex]
Mean experience empl.           &       -0.00125*** & 0.000907*** & -0.000873*** & 0.00158*** & 0.00188*** & -0.00156*** & -0.00276*** & 0.00192*** & -3.41e-05  & 1.95e-05   & -0.000512*** & 0.000249*  \\
                                &       (0.000228)  & (0.000190)  & (0.000266)   & (0.000301) & (0.000215) & (0.000176)  & (0.000548)  & (0.000481) & (0.000359) & (0.000287) & (0.000158)   & (0.000144) \\[1ex]
\% empl. with tert. educ.       &       0.0587***   & -0.0577***  & 0.0597***    & -0.0251    & 0.0844***  & -0.0804***  & 0.130***    & -0.119***  & 0.0494***  & -0.0531*** & 0.0369***    & -0.0430*** \\
                                &       (0.00431)   & (0.00406)   & (0.0134)     & (0.0169)   & (0.00319)  & (0.00306)   & (0.0202)    & (0.0193)   & (0.00939)  & (0.00736)  & (0.00473)    & (0.00388)  \\[1ex]
\% empl. with sec. educ.        &       0.00871***  & -0.00971*** & 0.0322***    & -0.0210*   & 0.0384***  & -0.0346***  & -0.00260    & -0.0232    & 0.0264***  & -0.0328*** & -0.000259    & -0.00690*  \\
                                &       (0.00272)   & (0.00263)   & (0.00874)    & (0.0114)   & (0.00267)  & (0.00239)   & (0.0150)    & (0.0145)   & (0.00822)  & (0.00699)  & (0.00400)    & (0.00370)  \\[1ex]
\% managers and profess.        &       0.0438***   & -0.0493***  & 0.0319***    & -0.0369*** & 0.0382***  & -0.0364***  & 0.0571***   & -0.0637*** & 0.125***   & -0.126***  & 0.0820***    & -0.0660*** \\
                                &       (0.00600)   & (0.00515)   & (0.00957)    & (0.0108)   & (0.00466)  & (0.00488)   & (0.0166)    & (0.0131)   & (0.00592)  & (0.00558)  & (0.00484)    & (0.00412)  \\[1ex]
\% part-time empl.              &       -0.00418    & 0.00662*    & 0.0570***    & -0.0829*** & 0.0551***  & -0.0535***  & 0.0777***   & -0.0969*** & 0.0258***  & -0.0161*** & 0.00355      & -0.00317   \\
                                &       (0.00418)   & (0.00375)   & (0.00675)    & (0.00791)  & (0.00342)  & (0.00332)   & (0.0301)    & (0.0255)   & (0.00612)  & (0.00545)  & (0.00412)    & (0.00395)  \\[1ex]
\% permanent contracts          &       -0.0345***  & 0.0436***   & -0.0242***   & 0.0619***  & 0.000483   & 0.00465*    & -0.00203    & 0.00554    & -0.0277*** & 0.0188*    & -0.0778***   & 0.0826***  \\
                                &       (0.00567)   & (0.00617)   & (0.00906)    & (0.0107)   & (0.00255)  & (0.00238)   & (0.00795)   & (0.00835)  & (0.00961)  & (0.00979)  & (0.00750)    & (0.00651)  \\[1ex]

Firm size:                      \\[1ex]
\quad \textit{50--249 empl.}    &       0.000439    & -0.000491   & 0.0217***    & -0.0144*** & 0.0657***  & -0.0554***  & 0.00895     & -0.00736   & -0.0330*** & 0.0301***  & 0.0201***    & -0.0177*** \\
                                &       (0.00275)   & (0.00287)   & (0.00268)    & (0.00314)  & (0.00235)  & (0.00182)   & (0.00578)   & (0.00566)  & (0.00776)  & (0.00724)  & (0.00258)    & (0.00217)  \\[1ex]
\quad \textit{$\geq$ 250 empl.} &       -0.00293    & 0.000297    & 0.0239***    & -0.0101*** & 0.106***   & -0.0876***  & 0.00620     & -0.00498   & -0.0332*** & 0.0320***  & 0.0226***    & -0.0264*** \\
                                &       (0.00469)   & (0.00472)   & (0.00275)    & (0.00313)  & (0.00397)  & (0.00347)   & (0.00880)   & (0.00795)  & (0.00663)  & (0.00592)  & (0.00256)    & (0.00230)  \\[1ex]
Public firm                     &       -0.0219***  & 0.0283***   & -0.0129*     & -0.00926   & 0.0183***  & -0.0105***  & -0.0448***  & 0.0349***  & 0.00734    & -0.0121    & -0.0402***   & 0.0292***  \\
                                &       (0.00592)   & (0.00583)   & (0.00663)    & (0.00830)  & (0.00436)  & (0.00379)   & (0.00719)   & (0.00558)  & (0.0352)   & (0.0268)   & (0.00458)    & (0.00407)  \\[1ex]
\midrule
Observations                    &       13,765      & 13,765      & 12,312       & 12,312     & 37,887     & 37,887      & 3,498       & 3,498      & 14,502     & 14,502     & 30,009       & 30,009     \\
R-squared                       &       0.138       & 0.199       & 0.059        & 0.059      & 0.174      & 0.191       & 0.226       & 0.191      & 0.110      & 0.124      & 0.105        & 0.115      \\
Region FE                       &       \checkmark  & \checkmark  & \checkmark   & \checkmark & \checkmark & \checkmark  & \checkmark  & \checkmark & \checkmark & \checkmark & \checkmark   & \checkmark \\
Sector FE                       &       \checkmark  & \checkmark  & \checkmark   & \checkmark & \checkmark & \checkmark  & \checkmark  & \checkmark & \checkmark & \checkmark & \checkmark   & \checkmark \\
\bottomrule
\end{tabular}
%
\begin{tablenotes}
\item Notes: Bootsrapped standard errors in parentheses (200 repetitions); asterisks denote significance levels: $^{*}$ p$<$0.05, $^{**}$ p$<$0.01, $^{***}$ p$<$0.001
\end{tablenotes}
%
\setlength{\tabcolsep}{6pt}
\end{threeparttable}
}

\end{table}
\end{landscape}

\appendix


\section*{Appendix~A: Descriptive statistics}
\label{sec:appendix_desc}
% Reset table counter for appendix
\setcounter{table}{0}
\renewcommand{\thetable}{A\arabic{table}}

Table~\ref{tab:summary_stats} shows shares of employees and firms falling in different categories of bargaining in our working sample, by country and by year, also including firms that do not apply any form of collective bargaining (\emph{i.e.,} contract wages separately with each single employee).

\begin{table}[hbp]
\centering
\caption{Share of firms and employees under different bargaining regimes, by country and year}
\label{tab:summary_stats}
\resizebox*{0.9\linewidth}{!}{%
\begin{tabular}{lrrrrrrrrrrrr|rr} %
\toprule
\textbf{Bargaining} & \multicolumn{4}{c}{Centralized} & \multicolumn{4}{c}{Firm-level} & \multicolumn{4}{c}{None}  & \multicolumn{2}{c}{\textbf{Total}}    \\
            & N firm & \% firm & N empl  & \% empl & N firm & \% firm & N empl & \% empl & N firm & \% firm & N empl  & \% empl & N firms & N empl \\
                             \cmidrule(lr){2-5}                \cmidrule(lr){6-9}               \cmidrule(lr){10-13} \cmidrule(lr){14-15} %\
BE          &        &         &         &         &        &         &        &         &        &         &         &         &               &              \\
\qquad 2006 & 7341   & 82,0\%  & 131339  & 79,5\%  & 1606   & 18,0\%  & 33852  & 20,5\%  & 0      & 0,0\%   & 0       & 0,0\%   & 165191        & 8947         \\
\qquad 2010 & 5581   & 81,1\%  & 108109  & 78,8\%  & 1304   & 18,9\%  & 29145  & 21,2\%  & 0      & 0,0\%   & 0       & 0,0\%   & 137254        & 6885         \\[1ex]
CZ          &        &         &         &         &        &         &        &         &        &         &         &         &               &              \\
\qquad 2006 & 466    & 2,6\%   & 122565  & 6,2\%   & 1315   & 7,3\%   & 942397 & 47,8\%  & 16278  & 90,1\%  & 905902  & 46,0\%  & 1970864       & 18059        \\
\qquad 2010 & 517    & 2,9\%   & 91588   & 4,6\%   & 1504   & 8,3\%   & 985320 & 49,4\%  & 16025  & 88,8\%  & 916717  & 46,0\%  & 1993625       & 18046        \\[1ex]
DE          &        &         &         &         &        &         &        &         &        &         &         &         &               &              \\
\qquad 2006 & 7546   & 20,3\%  & 1688535 & 58,4\%  & 2397   & 6,5\%   & 161995 & 5,6\%   & 27189  & 73,2\%  & 1042351 & 36,0\%  & 2892881       & 37132        \\
\qquad 2010 & 8001   & 27,0\%  & 774884  & 45,5\%  & 1787   & 6,0\%   & 120223 & 7,1\%   & 19815  & 66,9\%  & 806251  & 47,4\%  & 1701358       & 29603        \\[1ex]
ES          &        &         &         &         &        &         &        &         &        &         &         &         &               &              \\
\qquad 2006 & 23896  & 87,5\%  & 189372  & 80,5\%  & 3405   & 12,5\%  & 45900  & 19,5\%  & 0      & 0,0\%   & 0       & 0,0\%   & 235272        & 27301        \\
\qquad 2010 & 19294  & 76,9\%  & 141643  & 65,3\%  & 3992   & 15,9\%  & 57641  & 26,6\%  & 1818   & 7,2\%   & 17485   & 8,1\%   & 216769        & 25104        \\[1ex]
FR          &        &         &         &         &        &         &        &         &        &         &         &         &               &              \\
\qquad 2006 & 13583  & 88,3\%  & 86640   & 76,2\%  & 983    & 6,4\%   & 19036  & 16,8\%  & 820    & 5,3\%   & 7965    & 7,0\%   & 113641        & 15386        \\
\qquad 2010 & 27333  & 89,1\%  & 180504  & 81,9\%  & 3089   & 10,1\%  & 37974  & 17,2\%  & 271    & 0,9\%   & 1891    & 0,9\%   & 220369        & 30693        \\[1ex]
UK          &        &         &         &         &        &         &        &         &        &         &         &         &               &              \\
\qquad 2006 & 9645   & 22,4\%  & 32113   & 24,1\%  & 12104  & 28,1\%  & 33509  & 25,1\%  & 21262  & 49,4\%  & 67721   & 50,8\%  & 133343        & 43011        \\
\qquad 2010 & 17838  & 17,1\%  & 43622   & 24,4\%  & 21611  & 20,7\%  & 41709  & 23,3\%  & 64778  & 62,2\%  & 93785   & 52,4\%  & 179116        & 104227       \\
\bottomrule %
\end{tabular}%
}

\end{table}


Table~\ref{tab:summary_cpa_wage} shows mean of the dependent variable (the interdecile range of the within-firm distribution of residual wages), by country and type of collective bargaining.
\begin{table}[tb]
\caption{Average gap between 90th and 10th wage by country and collective agreement}
\label{tab:summary_cpa_wage}
\centering

\begin{tabular}{lrrrrrrr}
%\toprule
%& & \multicolumn{6}{c}{Log-difference \textbf{adjusted} 90th and 10th wages}\\
%Coll. Agr. & Year & BE & CZ & DE & ES & FR & UK\\
%\midrule
%Centralized & 2006 & 0.519 & 0.716 & 0.791 & 0.493 & 0.691 & 0.821\\
%Centralized & 2010 & 0.523 & 0.674 & 0.762 & 0.511 & 0.665 & 0.692\\
%\addlinespace
%Firm-level & 2006 & 0.540 & 0.753 & 0.704 & 0.609 & 0.654 & 0.700\\
%Firm-level & 2010 & 0.550 & 0.761 & 0.746 & 0.622 & 0.623 & 0.499\\
%\addlinespace
%None & 2006 & NA & 0.750 & 0.883 & NA & 0.657 & 0.784\\
%None & 2010 & NA & 0.673 & 0.910 & 0.623 & 0.755 & 0.615\\
%\bottomrule
%\end{tabular}
%\begin{tabular}{lrrrrrrr}
\toprule
Collective Agreement. & Year & BE & CZ & DE & ES & FR & UK\\
\midrule
Centralized & 2006 & 0.383 & 0.529 & 0.496 & 0.409 & 0.472 & 0.529\\
Centralized & 2010 & 0.357 & 0.521 & 0.492 & 0.402 & 0.450 & 0.487\\
\addlinespace
Firm-level & 2006 & 0.395 & 0.506 & 0.487 & 0.486 & 0.440 & 0.510\\
Firm-level & 2010 & 0.366 & 0.518 & 0.500 & 0.485 & 0.404 & 0.416\\
\addlinespace
%None & 2006 & NA & 0.379 & 0.555 & NA & 0.444 & 0.563\\
%None & 2010 & NA & 0.373 & 0.583 & 0.475 & 0.512 & 0.474\\
\bottomrule
 \multicolumn{8}{l}{Note: Mincer-adjusted residual wages}\\
\end{tabular}


\end{table}

\bigskip

Table~\ref{tab:firm_size}, Table~\ref{tab:sum_reg_var_cont} and Table~\ref{tab:sum_reg_var_mod_age} provide basic descriptive statistics for variables entering as controls in the regression analysis.

%CHECK: Tagliare? Inutile?
\begin{table}[ht]
\centering
\small
\caption{Distribution of firms by size by country}
\label{tab:firm_size}
\begin{tabular}{lrrrrrr}
\toprule
Firm size: & BE   & CZ    & DE    & ES    & FR    & UK    \\
\midrule
1--49       & 4,547 & 22,218  & 26,592 & 19,695 &  7,818 &  9,343 \\
50--249     & 4,533 &  6,633  & 19,021 &  8,662 &  9,019 &  2,765 \\
$\geq$250    & 5,766 &  3,251  & 15,177 & 10,400 & 14,058 & 20,332 \\
\bottomrule
\end{tabular}


\end{table}

%CHECK: tagliare? Inutile?
\begin{table}[htp]
\caption{Summary means and standard deviations for continuous variables in regression}
\label{tab:sum_reg_var_cont}
\centering
\small
\begin{tabular}{llrrrrrr}
\toprule
country                     & Stat. & BE    & CZ    & DE    & ES    & FR     & UK    \\
\midrule
Mean experience empl. (yrs) & mean & 9,506 & 9,061 & 8,574 & 7,235 & 10,748 & 7,079 \\
                            & s.d. & 5,810 & 5,229 & 6,051 & 6,202 & 6,361  & 5,278 \\
\% empl. with tert. educ.   & mean & 0,329 & 0,244 & 0,123 & 0,318 & 0,408  & 0,379 \\
                            & s.d. & 0,330 & 0,253 & 0,195 & 0,327 & 0,298  & 0,266 \\
\% empl. with sec. educ.    & mean & 0,419 & 0,654 & 0,706 & 0,190 & 0,420  & 0,523 \\
                            & s.d. & 0,305 & 0,255 & 0,236 & 0,244 & 0,270  & 0,262 \\
\% managers and profess.    & mean & 0,197 & 0,398 & 0,110 & 0,142 & 0,293  & 0,273 \\
                            & s.d. & 0,287 & 0,303 & 0,175 & 0,241 & 0,256  & 0,266 \\
\% part-time empl.          & mean & 0,229 & 0,139 & 0,266 & 0,152 & 0,159  & 0,269 \\
                            & s.d. & 0,260 & 0,186 & 0,248 & 0,247 & 0,231  & 0,268 \\
\% permanent contracts      & mean & 0,925 & 0,785 & 0,885 & 0,750 & 0,921  & 0,938 \\
                            & s.d. & 0,162 & 0,215 & 0,148 & 0,303 & 0,171  & 0,156 \\
\bottomrule                            
\end{tabular}


\end{table}

\begin{table}[ht]
\caption{Distribution of firms by modal age of employees by country}
\label{tab:sum_reg_var_mod_age}
\centering
\small
\begin{tabular}{lrrrrrr}
\toprule
Modal age workers: & BE   & CZ    & DE    & ES    & FR   & UK   \\
\midrule
14-19              & 35   & 19    & 276   & 127   & 68   & 987  \\
20-29              & 2414 & 2288  & 9277  & 8726  & 3664 & 6954 \\
30-39              & 4681 & 6254  & 10543 & 18674 & 9098 & 8052 \\
40-49              & 5671 & 10787 & 28528 & 11421 & 9926 & 8929 \\
50-59              & 2417 & 12166 & 11563 & 5275  & 7648 & 6174 \\
60+                & 54   & 588   & 603   & 564   & 491  & 1371 \\
\bottomrule
\end{tabular}

\end{table}


\clearpage

\section*{Appendix~B: FLB propensity score estimation details}

In order to address possible endogeneity driven by the potential self-selection of firms into a particular bargaining regime ($\mathit{FLB}=1)$, we essentially apply a two-step procedure based on propensity score estimates. We first estimate, separately by country, a preliminary first-step Probit

\begin{equation}
\label{eq:reg_propensity}
  \mathit{FLB}_j = \mathit{P}\left( \alpha_0 + \alpha_1\bm{V}_j \right)
\end{equation}
where $\mathit{FLB}_j$ is the dummy for the \emph{observed} presence of firm-level bargaining in firm $j$, $\mathit{P}$ is the Probit link function, and $\bm{V}$ a set of covariates that affect the choice to bargain at firm-level.

In the second step, the predicted probabilities (propensity scores)
$\widehat{\mathit{FLB}}_j = \mathit{P}\left( \hat{\alpha}_0 +
  \hat{\alpha}_1\bm{V}_j \right)$ obtained for each firm are included
as an additional control variable, as shown in Equation~\ref{eq:reg_dispersion} and Equation~\ref{eq:reg_decomposition}. The idea is that conditioning also upon the p-scores in addition to other controls solves selection due to unobserved factors, if FLB status is assigned as good as random based on observables. Thus, a simple OLS on the “p-score augmented” second step regressions will return correct estimates of the FLB dummy coefficient.

Table~\ref{tab:propensity} reports first-step Probit estimates, that
we use to compute FLB p-scores for eah firm. They show a satisfactory
goodness of fit, in terms of relatively high values of the area under
the ROC curve.

\begin{table}[thb]
\caption{Probit estimates of FLB propensity}
\label{tab:propensity}
\centering
\tiny
\begin{threeparttable}
\begin{tabular}{l*{6}{S}}
\toprule
                          & \multicolumn{1}{c}{(1)} & \multicolumn{1}{c}{(2)} & \multicolumn{1}{c}{(3)} & \multicolumn{1}{c}{(4)} & \multicolumn{1}{c}{(5)} & \multicolumn{1}{c}{(6)} \\
                          & \multicolumn{1}{c}{BE}  & \multicolumn{1}{c}{DE}  & \multicolumn{1}{c}{ES}  & \multicolumn{1}{c}{CZ}  & \multicolumn{1}{c}{UK}  & \multicolumn{1}{c}{FR}  \\
\midrule
Mean experience empl.     &  0.0300*** & 0.0243*** & 0.0409***  & 0.0755*** & -0.0127*** & 0.00221    \\
                          &  (0.00313) & (0.00276) & (0.00184)  & (0.00730) & (0.00277)  & (0.00282)  \\[0.5ex]

Modal age workers:        \\[1ex]
\quad 20-29               &  -0.0844   & -0.890*** & -0.110     &           & 0.115      & 0.125      \\
                          &  (0.252)   & (0.246)   & (0.218)    &           & (0.0936)   & (0.142)    \\[0.5ex]

\quad 30-39               &  0.0366    & -0.564**  & -0.0319    & -0.0328   & 0.141      & 0.285**    \\
                          &  (0.250)   & (0.245)   & (0.218)    & (0.104)   & (0.0938)   & (0.136)    \\[0.5ex]

\quad 40-49               &  0.00410   & -0.498**  & -0.0881    & 0.0855    & 0.124      & 0.395***   \\
                          &  (0.249)   & (0.244)   & (0.218)    & (0.117)   & (0.0934)   & (0.134)    \\[0.5ex]

\quad 50-59               &  -0.124    & -0.559**  & -0.0560    & 0.0737    & 0.152      & 0.420***   \\
                          &  (0.251)   & (0.245)   & (0.219)    & (0.107)   & (0.0956)   & (0.134)    \\[0.5ex]

\quad 60+                 &            & -0.829*** & -0.113     & -0.357    & 0.120      &            \\
                          &            & (0.298)   & (0.232)    & (0.299)   & (0.110)    &            \\[0.5ex]

\% empl. with tert. educ. &  0.109     & 0.662***  & 0.405***   & 0.455     & -0.198**   & 0.108      \\
                          &  (0.0793)  & (0.135)   & (0.0393)   & (0.347)   & (0.0991)   & (0.0759)   \\[0.5ex]

\% empl. with sec. educ.  &  0.137**   & 0.728***  & 0.159***   & 0.117     & -0.133     & 0.522***   \\
                          &  (0.0557)  & (0.0982)  & (0.0382)   & (0.237)   & (0.0945)   & (0.0699)   \\[0.5ex]

\% managers and profess.  &  -0.0209   & 0.0162    & -0.186***  & 0.206     & -0.00602   & -0.581***  \\
                          &  (0.0906)  & (0.110)   & (0.0604)   & (0.290)   & (0.0594)   & (0.0785)   \\[0.5ex]

\% part-time empl.        &  -0.152**  & 0.115*    & -0.0164    & 0.0864    & -0.0706    & 0.386***   \\
                          &  (0.0682)  & (0.0648)  & (0.0430)   & (0.366)   & (0.0560)   & (0.0558)   \\[0.5ex]

\% permanent contracts    &  0.510***  & 0.116     & -0.156***  & -0.279*   & 0.242**    & -0.262***  \\
                          &  (0.115)   & (0.127)   & (0.0385)   & (0.151)   & (0.0978)   & (0.0797)   \\[0.5ex]

Firm size:                \\[1ex]
\quad \textit{50-249 empl.}              &  0.704***  & 0.105***  & 0.624***   & 0.440***  & -0.184**   & 0.473***   \\
                          &  (0.0408)  & (0.0383)  & (0.0235)   & (0.0855)  & (0.0720)   & (0.0385)   \\[0.5ex]

\quad \textit{$\geq$ 250 empl.}         &  1.135***  & 0.179***  & 1.133***   & 1.164***  & -0.290***  & 0.624***   \\
                          &  (0.0425)  & (0.0375)  & (0.0233)   & (0.0916)  & (0.0511)   & (0.0387)   \\[0.5ex]

Public firm               &  -1.230*** & -0.921*** & 0.570***   & 0.202     & -1.346***  & 0.838***   \\
                          &  (0.0964)  & (0.0404)  & (0.0405)   & (0.139)   & (0.0403)   & (0.0428)   \\[0.5ex]

NACE Sector:              \\[1ex]
\quad D                   &  -0.413*   & -0.293**  & -0.380***  & 0.609***  & 0.201      & -1.373***  \\
                          &  (0.243)   & (0.137)   & (0.0667)   & (0.218)   & (0.311)    & (0.180)    \\[0.5ex]

\quad E                   &  -0.809*** & 0.407***  & 0.272***   & 1.075***  & 0.141      & 1.157***   \\
                          &  (0.304)   & (0.141)   & (0.0803)   & (0.305)   & (0.328)    & (0.155)    \\[0.5ex]

\quad F                   &  -1.140*** & -1.096*** & -0.882***  & -0.894*** & -1.032***  & -1.334***  \\
                          &  (0.253)   & (0.203)   & (0.0770)   & (0.225)   & (0.318)    & (0.278)    \\[0.5ex]

\quad G                   &  -0.853*** & -0.451*** & -0.493***  & 0.799***  & 0.405      & -1.270***  \\
                          &  (0.245)   & (0.156)   & (0.0715)   & (0.254)   & (0.311)    & (0.220)    \\[0.5ex]

\quad H                   &  -0.999*** & 0.275     & -0.808***  &           & -0.510     & -0.486***  \\
                          &  (0.268)   & (0.168)   & (0.0832)   &           & (0.334)    & (0.182)    \\[0.5ex]

\quad I                   &  -0.441*   & 1.229***  & -0.112     & -0.168    & -0.248     & 0.573***   \\
                          &  (0.248)   & (0.136)   & (0.0705)   & (0.228)   & (0.310)    & (0.152)    \\[0.5ex]

\quad J                   &  -0.0811   & -0.816*** & -1.105***  & 0.216     & 0.562*     & -0.00925   \\
                          &  (0.255)   & (0.161)   & (0.0818)   & (0.309)   & (0.315)    & (0.158)    \\[0.5ex]

\quad K                   &  -0.744*** & 0.117     & -0.509***  & 1.391***  & 0.0816     & 0.169      \\
                          &  (0.247)   & (0.139)   & (0.0728)   & (0.286)   & (0.314)    & (0.155)    \\[0.5ex]

\quad L                   &            &           & -0.101     & 2.351***  & 0.324      & 1.735***   \\
                          &            &           & (0.111)    & (0.364)   & (0.311)    & (0.159)    \\[0.5ex]

\quad M                   &  -0.966*** & -0.0856   & -0.953***  & 0.142     & -0.676**   & -0.510***  \\
                          &  (0.282)   & (0.157)   & (0.0920)   & (0.286)   & (0.311)    & (0.189)    \\[0.5ex]

\quad N                   &  -0.860*** & 0.973***  & -0.670***  &           & -1.004***  & -0.554***  \\
                          &  (0.248)   & (0.138)   & (0.0834)   &           & (0.312)    & (0.163)    \\[0.5ex]

\quad O                   &  -0.602**  & 0.394***  & 0.126*     & 1.703***  & -0.496     & 0.338**    \\
                          &  (0.254)   & (0.139)   & (0.0744)   & (0.357)   & (0.311)    & (0.155)    \\[0.5ex]

Regional GDP pps          &  0.00120   & 0.000710  & 0.00371    & 0.302***  & 0.00649*** & -0.00184   \\
                          &  (0.00157) & (0.00397) & (0.00231)  & (0.0712)  & (0.00181)  & (0.00168)  \\[0.5ex]

Regional unemp. rate      &  0.00426   & 0.0357*** & 0.00866*** &           & 0.0716***  & -0.0433*** \\
                          &  (0.00366) & (0.00585) & (0.00161)  &           & (0.00832)  & (0.0101)   \\[0.5ex]

Constant                  &  -1.770*** & -1.832*** & -1.679***  & -7.023*** & 0.544      & -2.022***  \\
                          &  (0.374)   & (0.329)   & (0.235)    & (1.437)   & (0.351)    & (0.243)    \\
\midrule
Observations              &  13,730    & 12,312    & 37,887     & 3,498     & 14,502     & 29,943     \\
Area under ROC curve      &  0.781     & 0.811     & 0.825      & 0.870     & 0.875      & 0.935      \\
\bottomrule
\end{tabular}
%
\begin{tablenotes}
\item Notes: Dependent variable is FLB dummy. Standard errors in parentheses; asterisks denote significance levels: $^{*}$ p$<$0.05, $^{**}$ p$<$0.01, $^{***}$ p$<$0.001
\end{tablenotes}

\end{threeparttable}

\end{table}

Notice that the predictors $\bm{V}$ are for the most part the same as
the controls appearing in the set $\bm{X}$ in the main equations.
However, to ease identification, we exclude average tenure of the
workforce, as it is sensible to assume that tenure affects wages and
wage inequalities, but it does not directly impact on the decision to
adopt FLB. Also notice that, in place of the sector and region
fixed-effects included in the controls $\bm{X}$ (likely subject to
incidental parameter problems in Probit estimates), the set of
covariates $\bm{V}$ includes the GDP per capita (at
purchasing power parity, base year 2006) and the unemployment rate in
the region where each firm is located, thus controlling for
macroeconomic-and-regional dynamics that may play a direct influence
on the decision to apply firm-level bargaining.\footnote{These
  additional variables are taken from EUROSTAT-Regional Statistics and
  measured at the level of NUTS-1 regions, since this is the
  precision of the information on firms' geographical location in
  SES.}

\end{document}


% \clearpage

% \section*{Appendix~C: FLB and components of inequality measures}

% \subsection*{C1: baseline OLS estimates XXX La teniamo???XXX}

% In Table~\ref{tab:prelim_ols} we show a basic difference-in-mean test
% obtained by running a simple OLS regression of the two measures of
% (residual) wage inequality $\Delta w^{90/10}$ and $\Delta
% w^\mathit{jobs}$ against the $\mathit{FLB}$ dummy and a constant term,
% separately by country and survey year.

% This exercise provides a descriptive assessment of the unconditional
% relation linking firm-level bargaining and wage inequalities, whereas
% regression results reported in the main text involve more reliable
% estimates, controlling for additional observables that may drive the
% differences in wage inequalities and for potential endogenous
% selection of the $\mathit{FLB}$ dummy.

% \begin{table}[hbp]
% \centering
% \caption{Within-firm wage inequalities: OLS Difference-in-means test across firms under firm-level bargaining and other firms, by country and year.}
% \label{tab:prelim_ols}
% \sisetup{
    parse-numbers = false,
    table-number-alignment  = center,%
    group-digits            = false,%
    table-format            = -2.4,%
    table-auto-round,%
    input-symbols           = {()},    % redefine ( ) as text symbols
    table-space-text-pre    = {$-$},   % allow for proper spacing of -(
    table-space-text-post   = {},%
    table-align-text-post   = false,   % toggle alignment of *** after estimates
}

\begin{threeparttable}
\begin{tabular}{cccSSSSr} %
\toprule %
 & & & \multicolumn{2}{c}{Firm-level bargaining} & \multicolumn{2}{c}{Constant} & Obs. \\
& Country & Year & \multicolumn{1}{r}{Coeff.} & \multicolumn{1}{c}{S.e.} %
                 & \multicolumn{1}{r}{Coeff.} & \multicolumn{1}{c}{S.e.} & \multicolumn{1}{c}{N} \\[1ex]
\cmidrule(lr){2-2} \cmidrule(lr){3-3} \cmidrule(lr){4-5} \cmidrule(lr){6-7} \cmidrule(lr){8-8} 
\multirow{13}[0]{*}{$\Delta w^{90/10}$} \ldelim\{{13}{1pt}%
& BE & 2006 & 0.0120**   & (0.00463) & 0.384***  & (0.00211) & 8639  \\
&    & 2010 & 0.00927*   & (0.00382) & 0.357***  & (0.00179) & 6633  \\[1ex]
& DE & 2006 & -0.00735   & (0.00521) & 0.495***  & (0.00238) & 7462  \\
&    & 2010 & 0.00868    & (0.00532) & 0.494***  & (0.00238) & 9753  \\[1ex]
& ES & 2006 & 0.0784***  & (0.00453) & 0.410***  & (0.00168) & 24278 \\
&    & 2010 & 0.0838***  & (0.00434) & 0.403***  & (0.00193) & 19108 \\[1ex]
& CZ & 2006 & -0.0225*   & (0.0104)  & 0.531***  & (0.00953) & 1780  \\
&    & 2010 & -0.000329  & (0.00960) & 0.521***  & (0.00866) & 2019  \\[1ex]
& UK & 2006 & -0.0196*** & (0.00551) & 0.531***  & (0.00399) & 9178  \\
&    & 2010 & -0.0730*** & (0.00680) & 0.489***  & (0.00524) & 6079  \\[1ex]
& FR & 2006 & -0.0317*** & (0.00963) & 0.472***  & (0.00270) & 10900 \\
&    & 2010 & -0.0475*** & (0.00412) & 0.451***  & (0.00214) & 19109 \\[2ex]
%
\multirow{13}[0]{*}{$\Delta w^\mathrm{jobs}$} \ldelim\lbrace{13}{1pt} %
& BE & 2006 & 0.000427   & (0.0207)  & -0.0166   & (0.0115)  & 1411  \\
&    & 2010 & -0.0179    & (0.0179)  & -0.0108   & (0.00858) & 1164  \\[1ex]
& DE & 2006 & -0.0706*** & (0.0186)  & -0.000797 & (0.00913) & 2706  \\
&    & 2010 & -0.0115    & (0.0170)  & 0.00555   & (0.00747) & 3529  \\[1ex]
& ES & 2006 & -0.0528*   & (0.0212)  & 0.0597*** & (0.0116)  & 2068  \\
&    & 2010 & -0.154***  & (0.0226)  & 0.0855*** & (0.0138)  & 1695  \\[1ex]
& CZ & 2006 & 0.0298     & (0.0213)  & 0.0792*** & (0.0192)  & 1598  \\
&    & 2010 & 0.127***   & (0.0229)  & -0.0490*  & (0.0215)  & 1689  \\[1ex]
& UK & 2006 & -0.0156    & (0.0260)  & -0.0318   & (0.0206)  & 1544  \\
&    & 2010 & -0.124***  & (0.0357)  & -0.0228   & (0.0259)  & 646   \\[1ex]
& FR & 2006 & -0.167***  & (0.0216)  & 0.0479*** & (0.00875) & 2572  \\
&    & 2010 & -0.108***  & (0.0178)  & 0.0338*** & (0.00638) & 4323  \\
\bottomrule%
\end{tabular}
%
\begin{tablenotes}
\item Notes: Robust standard errors in parenthesis; asterisks denote significance levels: $^{*}$ p$<$0.05, $^{**}$ p$<$0.01, $^{***}$ p$<$0.001
\end{tablenotes}
%
\setlength{\tabcolsep}{6pt}
%
\end{threeparttable}

% \end{table}

% Considering the percentile wage-dispersion $\Delta w^{90/10}$, firms
% that adopt firm-level bargaining present, on average, higher
% inequality in Belgium and Spain, while lower dispersion in France, the
% United Kingdom, and the Czech Republic (except in 2010). In Germany,
% instead, the average $\Delta w^{90/10}$ does not differ statistically
% between the two groups of firms, in both 2006 and 2010.

% In terms of the professional wage-gap $\Delta w^\mathit{jobs}$, across
% firms where we can compute this measure, the more common pattern is
% that firms bargaining locally display lower inequalities than other
% firms, although we observe insignificant coefficients on the
% $\mathit{FLB}$ dummy in some country-year combinations, and a positive
% coefficient for the Czech Republic in 2010. 




% XXX Commenti a control variables per eserizio su 90 e 10 percentile XXXX
% Results on the control variables are rather consistent across
% countries, although with some variation in the significance levels.
% Modal age tends to positively correlate with the 90th percentile and
% negatively with the 10th percentile.  A higher proportion of women in
% the workforce associates with lower wages at the 90th percentile, but
% with higher wages at the 10th percentile. A similar pattern is also
% detected for mean in-job tenure (with the exception of Spain) and for
% the share of workers covered by permanent contracts. Conversely, the
% share of educated workforce associates with higher wages at the 90th
% percentile and lower wages at the 10th percentile, and exactly the
% same pattern is detected for the proportion of apical professions in
% the firm and for the share of part-time workers (not in Belgium).
% Moving to firm characteristics, larger firms show higher wages at the
% 90th percentile, but lower wages at the 10th percentile (not in the
% UK), whereas publicly-owned firms tend to pay more than private firms
% their workers at the 10th percentile and to pay less their workers at
% the 90th percentile.XXX



% XXX Commenti su controls coefficients per esercizio con le components
% di occupation wage gap XXXX
% Results on controls display, once again, heterogeneity across
% countries. The modal age of the workforce displays a significant
% association with wages of managers in Spain (positive) and in the UK
% (negative), while a relatively strong and negative association emerges
% with the average wage of low-layers employees in Belgium. The share of
% women in the workforce features a positive relation with managers'
% wages in Spain and Germany, but the relation is negative in France.
% Also, wages of low-layers employees are higher in firms with more
% women in the Czech Republic, while they decrease with the number of
% women in Germany and Spain. Average tenure does not display strong
% associations in most countries, whereas education does, and the share
% of employees with tertiary education, in particular: in all countries
% (but France), firms with relatively more educated workforce pay
% relatively higher wages to managers and relatively lower wages to
% low-layers employees.  The share of managers/professionals and the
% contract types do not show systematic patterns. Among enterprise
% characteristics, we observe that larger firms pay managers more than
% other firms in most countries (not in France). The opposite holds for
% public firms as compared to private firms.


XXXXXXXXXXXXXXXXXXXXXXXXXXXXXXX 
PEZZI DA USARE ALTROVE, e.g. in conclusions ???
XXXXXXXXXXXXXXXXXXXXXXXXXXXXXXXX

XXX Footnote su opening clauses: \footnote{A feature of the bargaining
  system common to Belgium, Germany and Spain that we cannot measure
  in our data, but potentially relevant for inequality connected to
  firm-level bargaining, is the existence in these countries of the
  so-called “opening clauses” (since 1982 in Belgium, since 1993 in
  Germany, and since 2001 in Spain). While firm-level agreements
  cannot worsen conditions settled at higher bargaining levels, these
  clauses allow for that in certain specific circumstances. However,
  even if prima facie they could work as an inequality-enhancing
  driver attached to firm-level bargaining, by allowing to pay at
  least some of the employees less than what stipulated at more
  centralised levels, clear predictions about their potential impact
  on within-firm pay structures are not easy. In fact, their use does
  not necessarily entail pay reductions, but ranges from working-time
  adjustments, to suspension of pay increases or supplementary grants.
  Also, it is not clear if they are applied to some categories
  \textit{only} (e.g., the bottom paid workers), thus impacting on
  within-firm wage structure, or to \textit{all} employees, which
  would instead affect all wages, keeping the rankings unchanged.
  Previous studies do not provide much help, since detailed
  information on their specific use by firms is at best limited, if
  any, exactly as in our data. For Germany, they were mostly used to
  increase wages, as a monetary compensation bargained by unions in
  exchange of more flexibility in covered
  firms~\citep{brandle2013opening}, but their use may have also
  produced significant wage worsening in some firms~\citep{ellguth2012wage}.} 
XXXXX





XXX Pezzo su heterogeneity of bargaining systems...
This is consistent
with the clear-cut cross-country differences in terms of bargaining
coverage, structures and mechanisms of coordination: national,
sectoral and company bargaining do not operate equally in all
countries, and the relative diffusion, scope and content of firm-level
collective bargaining are highly heterogeneous.
XXX

XXXX In this respect, any inter-temporal change that we shall uncover
in the relation between firm-level bargaining and within-firm wage
inequalities should be taken as mirroring changes in the use of
firm-level agreements, and not as a test of what happens when firms in
a country are given the opportunity to move from a fully centralised
system to fully decentralised wage bargaining. XXX


XXXXXX Pezzo su contrasting effects...USARE ALTROVE??? XXX Based on
theory, it is difficult to formulate a sharp a-priori hypothesis on
whether firms which bargaining only at more centralised levels should
be expected to be more or less unequal than firms which also bargain
locally.  Counteracting effects are likely at work. Our analysis shall
be informative about whether inequality enhancing use of firm-level
bargaining prevail over potentially equalizing factors, revealing if
and to what extent different groups of employees (top vs.  bottom
paid, and managers vs. low-layers) eventually benefit or loose. XXXXX
XXXXX


XXX Our study speaks to this literature.  These approaches suggest us
that power struggles within firms are critical to within-firm wage
stratification.XXX

XXX Balance of powers within the firm, closely related to the role and
importance of different employees in the hierarchical and
organizational structure, together with the institutional context,
defined by cross-country and cross-sectoral differences in wage
bargaining practices, and their evolution over time, represent key
elements stressed in other literatures. XXXX


XXXX We use various measures of within-firm inequality that try to
differently capture how groups of employees (between highest and
lowest wage groups, and between managers and low-layers workers) may
benefit from firm-level bargaining.XXXX


XXXX Pezzo su predictions across VoC groups...XXXX Predict that
firm-level agreements are more likely used for inequality enhancing
purposes in countries closer to coordinated-market regimes, where
firms may be more in need to adjust their pay structure employing them
selectively on some groups of employees, since the structure of
industrial relations at more centralised (i.e., national or industry)
levels remains comparatively more rigid and more complex to manage
than in market-oriented systems, despite the process of hybridization
took place everywhere.XXXX

XXXX Altro pezzo su VoC groups...XXX The prediction would be that
firm-level bargaining is more likely to increase inequality in
market-oriented, flexibility-friendly countries like the UK and the
Czech Republic, than in the other countries we study, where a
coordinated-market type of capitalism is reflected in their more
centralised bargaining systems.XXXX

%fact that we observe difference in the relationship between firm-level bargaining and inequality in some countries but not others may have less to do with differences in national institutions, than with firm-level dynamics.
%Whatever the underlying mechanisms, the in the estimated effects suggest that the final outcome 

% Limitations (wage drift)
%As the measure for observed wages, $\hat{w}$, we use hourly wages. In SES, these are recorded as the compensation actually paid to the workers, without distinguishing between the wage components that are set through firm-level bargaining from the components agreed upon at more centralised levels. In particular, as it is often the case in the literature, we do not have information on un-bargained wage drifts. These are parts of compensation granted by firms to specific employees (or group of employees) outside collective bargaining, whatever the level of collective bargaining adopted by the enterprise. \cite{cardoso.portugal.2005} find for Portugal that such unilateral components increase wage inequalities within firms, although the theoretical possibility remains open that wage drifts -- much in line with the mechanisms that may lie behind firm-level collective bargaining -- operate to re-balance the internal pay structure, for instance for fairness reasons. Also, although unilateral wage drifts may affect in principle all types of firms, they are expected to be stronger and more frequent in firms which only bargain at national or industry level, as a way to gain flexibility and adjust the internal wage structure vis-à-vis the centralised agreements, but without going through collective bargaining at the firm-level~\citep{dellaringa.pagani.2007}. If this is the case, then we expect such wage drifts to increase within-firm inequalities less in firms which bargain locally.


%%% PEZZI SU SINGOLI PAESI:
In Belgium, the room for pay bargaining at the enterprise level is limited due to indexation of wages in national agreements. As a result, we do not expect firm-level bargaining to play a major role in this country: the scope for local contracting to affect internal wage structures is limited, with no major changes over time.

In \textbf{Germany}, wages are bargained mostly at the industry level between individual trade unions and employers' organisations, although the agreements allow for flexibility at the company level. During the period considered, the large prevalence of the higher bargaining level remains quite stable over time. Given these features and the central role of workers/unions in the work councils, typically pushing toward wage standardisation, we could conjecture that firm-level agreements signed on top of more centralised bargaining are especially likely to pursue egalitarian purposes in this country, thus compressing wage dispersion within firms that adopt them.

In \textbf{Spain}, much like in Germany, the structure of wage bargaining system shows a predominant role of industry-level but there are features that are quite peculiar to this country. There is indeed a peculiarly complex coexistence and interaction of negotiations at national and province-level within industries, while firms adopting firm-level collective bargaining traditionally feature a higher union density than multi-employer bargaining firms~\citep{plasman.rusinek.ea.2007}. This all suggests that in Spain the union's pressure to compress wage inequalities may be particularly strong in firm-level bargaining firms.

The \textbf{United Kingdom} epitomises the Anglo-Saxon tradition of industrial relations, where wage bargaining is mostly uncoordinated, with most workers bargaining work contracts individually with employers. When a collective agreement occurs, the majority of them are signed at the firm-level but such agreements do not establish legally binding norms claimable in court. Also, collective agreements are very rare in the private sector, while in the public sector employee coverage is more comparable to other countries. Altogether, in view of these features and of the traditionally high flexibility in the use and content of heterogeneous pay schemes at firm-level in this country, one can suppose that in the UK within-firm pay inequalities are especially wide in firms bargaining locally compared to other firms.

The \textbf{Czech Republic} belongs in the Eastern European trend to embrace decentralised, market-oriented institutional settings in the post-Soviet era.
In line with this, uncoordinated wage setting occurring directly between firms and individuals are quite spread. When collective agreements are reached, they mostly occur at firm or establishment level. Collective agreements at higher levels are rare but, when present, they last for at least two years, while those signed at company level may be renegotiated every year, thus allowing for a certain degree of flexibility in reshaping the wage ladder in the enterprise. These features suggest that in the Czech Republic, similarly to the UK, there is room for firm-level bargaining to result into more unequal within-firm pay structures.

The last country that we analyse, \textbf{France}, is characterised by a peculiarly complex system of industrial relations, where all the levels of collective negotiations -- inter-sectoral, industry or firm-level -- are closely intertwined and, in turn, they occur at both national or local level~\citep{fulton.2013,fulton.2015}. A notable feature is that the inversion of the favourability principle was introduced in 2004, recognising to firm-level agreements the possibility to derogate from any condition settled at more centralised levels, if not explicitly prohibited~\citep{keune2011decentralizing}.
This mostly concerned working time, however, and few firms exploited the opportunity to act on pay structures. Altogether, the combination of elements pushing to increase flexibility in the firm with the enduring and complex role of centralised bargaining levels, makes it particularly difficult to predict whether firm-level bargaining firms should show unequal wage structures than other firms in this country.




