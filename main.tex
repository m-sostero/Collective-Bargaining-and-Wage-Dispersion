\documentclass[12pt]{article}
\usepackage[a4paper,left=2cm,right=2cm,top=1.0in,bottom=1.0in]{geometry}
%
\usepackage[english]{babel}
% \usepackage[utf8]{inputenc}
% \usepackage[T1]{fontenc}
% \usepackage{lmodern} 
% Punteggiatura in stile TeX: ``virgolette", -- incisi -- 

\usepackage{rotating}
\usepackage{pdflscape}
\usepackage{graphicx}
% \usepackage{epstopd}
% \usepackage{color}

\usepackage{endnotes} % Move footnote at the end of the document
\let\footnote=\endnote

% \usepackage{calc}
\usepackage{indentfirst}
\usepackage[toc,page]{appendix}
\usepackage{amsmath,amsfonts,bm,amssymb}
% \usepackage{hyperref}

%\usepackage{caption3}
%\usepackage{float}

% \usepackage{longtable}
% \usepackage{colortbl}
% \usepackage[flushleft]{threeparttable}
\usepackage{booktabs}
\usepackage{multirow}
\usepackage{bigdelim}
% \usepackage{array}
% \usepackage{supertabular}

\usepackage{siunitx} % Per le tabelle dei coefficienti
\sisetup{
    detect-all, % current font
    parse-numbers = false,
    table-number-alignment  = center,
    group-digits            = false,
    table-format            = -2.6,
    table-auto-round,
    input-symbols           = {()},  % redefine ( ) as text symbols
    table-space-text-pre    = {$-$},   % allow for proper spacing of -(
    table-space-text-post   = {$^{***}$},
    table-align-text-post = false, % toggle alignment of *** after estimates
  }

\usepackage{threeparttable}

\usepackage{authblk}

\usepackage{setspace}

% Questo il formato che richiede la rivista per la biblio
\usepackage[authoryear]{natbib}
\bibpunct{(}{)}{;}{}{,}{,} % redefine punctuation of natbib output
\bibliographystyle{ecta}

\begin{document}

\title{Firm-level pay agreements and within-firm wage inequalities: Evidence across Europe}
\date{}
%\author{}

\author[a]{Valeria Cirillo}
\author[b]{Matteo Sostero}
\author[b]{Federico Tamagni\footnote{\textit{Corresponding author}: Federico Tamagni, Scuola Superiore Sant'Anna, Pisa, Italy. Postal address: c/o Institute of Economics, Scuola Superiore Sant'Anna, Piazza Martiri 33, 56127, Pisa, Italy, \emph{E-mail}~f.tamagni@sssup.it, \emph{Tel}~+39-050-883343.}}

\affil[a]{\footnotesize{INAPP, Rome, Italy}}
\affil[b]{Institute of Economics, Scuola Superiore Sant'Anna, Pisa, Italy}

% articolo totale 10 000 words

\maketitle




\maketitle

\begin{abstract}
  This article investigates the relation linking single-employer
  bargaining -- increasingly spread in Europe -- and within-firm wage
  dispersion -- a significant driver of overall wage inequality.  The
  study considers six European economies (Belgium, Spain, Germany,
  France, the Czech Republic and the UK), featuring different
  collective bargaining institutions, in 2006 and 2010. We examine two
  different measures of within-firm inequality, allowing to capture
  how different groups of employees may differently benefit or lose
  from firm-level bargaining. Our findings show that firm-level
  bargaining has heterogeneous effects across countries, by inequality
  measures and over time. We interpret our evidence as supporting that
  country-specificities and the heterogeneous balance of power within
  organizations represent key elements to understand the role of the
  bargaining system in shaping inequalities.

\bigskip
 
\noindent \textbf{Keywords}: within-firm wage inequalities, occupational wage-gap, firm-level bargaining, matched employer-employee data

%\noindent \textbf{SER Keywords}: wages, inequality, collective bargaining, firms, occupations, Europe 

\noindent \textbf{JEL classification}: J31, J33, J51, J52

\end{abstract}

\cleardoublepage

\onehalfspacing


\section{Introduction}
\label{sec:intro}



%%%% NEW INTRO
The notable raise of inequality observed since the Great Depression
reopened the debate on the causes of this phenomenon among social
scientists. Together with technological changes, globalization,
financialization and changes in power relations among social groups,
the evolution of wage-setting institutions is among the major
candidates in explaining rising wage inequality~\citep{cobb2016}. In
this article, we focus on the possible role played by devolution of
bargaining levels -- the progressive shift of the central locus of
collective wage setting from more centralised levels (national or
industry) to the level of single firms~\citep{undy1978}. This trend
has affected wage bargaining systems -- particularly in Europe -- with
the intended aim to provide more flexibility and a better match with
the specific needs arising in the firms. The ``corporatist'' system of
industrial relations~\citep{wallerstein1997unions} that characterized
most European countries in the second half of the
20\textsuperscript{th} century, has progressively given way to an
``hybrid'' system~\citep{Braakmann}, where ``multi-employer''
collective bargaining conducted at centralised level still prevails,
but ``single-employer'' collective agreements signed locally at the
firm level are increasingly spread and allowed to derogate to specific
provisions stipulated at centralised levels~\citep{visser2013wage}.

The increased role of firm-level collective agreements has been
connected to two types of wage inequality, that is between or within
firms. The vast majority of the studies focus on between-firm wage
inequality, comparing the dispersion of wages among workers that are
covered only by a centralised agreement against the dispersion
observed among workers who also bargain at the firm-level, on top of
centralised
contracts~\citep{dellaringa.lucifora.1994,hibbs.locking.1996,palenzuela.jimeno.1996,hartog.leuven.ea.2002,rycx.2003,cardoso.portugal.2005,checchi.pagani.2005,plasman.rusinek.ea.2007,card.delarica.2006,dellaringa.pagani.2007,dahl.lemaire.ea.2013,daouli.demoussis.ea.2013}.

In this paper, we explore the effects of the level of collective
bargaining on within-firm wage inequalities, comparing whether firms
that apply firm-level collective bargaining exhibit more unequal wages
than firms that only adopt centralised collective bargaining. That is,
in line with approaches recognizing the key role of firms as a locus
of inequality creation, we place our attention on whether firm-level
bargaining is one of the ways through which employers shape inequality
by deciding to differently pay their different workers in different
jobs. In fact, within-firm wage inequality is not less relevant than
between-firm inequality, as it accounts for around half of overall
wage dispersion in
most~economies~\citep{lazear.shaw.2007,fournier.koske.2013,GlobalWageReport}.

Theoretically, the links between the level of collective bargaining
and within-firm wage inequalities can be framed within several
approaches across different fields of research.

The labour economics literature offers a variety of competing
frameworks that explain why wage setting at the workplace -- as
opposed to market wage-setting -- may be central in the creation of
inequality.  Tournament theory~\citep{lazear.1979} predicts that
firm-level bargaining firms show a more unequal wage structure, due to
performance-related pay or other differentials-in-compensation schemes
being designed to elicit or reward workers' effort in the firm.
Conversely, firms which bargain locally are expected to show lower
wage dispersion in insider-outsider models ``with
unions''~\citep{lindbeck1986wage,lindbeck2001insiders}, since unions
are known to favor wage compression, as well as in theories of ``fair
wages''~\citep{akerlof.1984}, suggesting that firms seek to avoid that
too large pay differences, which may eventually end up detrimental to
overall firm performance if perceived as ``unfair''. Other frameworks
explaining wage determination on the basis of efficiency-wages,
rent-sharing or differential compensations for unmeasured workers'
ability -- although more suited to explain between-firm inequalities
-- offer examples of other wage-setting practices that may affect
within-firm wage dispersion, to the extent that they are used
selectively by employers to reshape the pay-scale within firms. In
fact, we would expect within-firm inequality to be higher in firms
bargaining locally -- as opposed to firms bargaining only at more
centralised levels-- every time contracts collectively bargained in
the firm are used to unequally compensate the contribution of
different employees to the firms'
objectives~\citep{bayo2013diffusion}. High within-firm inequality may
also arise when firms selectively remunerate human capital or
particularly valuable firm-specific resources (according to the
resource-based view of the firm), or to solve transaction costs and
agency problems arising for different occupational groups
\citep{eisenhardt1989agency,o1998structure}. Yet, the actual
implementation of such practices may also end up reducing within-firm
inequalities vis-a-vis firms which only bargain at centralised levels,
if these types of work-place collective agreements respond to
redistributive, fair or egalitarian motives (e.g.~by workers or
unions). Overall, many contrasting mechanisms may operate in different
firms.

A major limitation of economic theories is that they are divorced from
institutional contexts of countries, while cross-country differences
in wage bargaining practices and their inter-temporal evolution
represent key elements stressed in other literatures. Recent
developments in organizational approaches to stratification,
recognizing the central role of firms as drivers of wage-inequality,
offer a broader perspective explaining how and why firms act on their
internal wage structure. Three main driving forces coexist and shape
stratification~\citep{stainback2010}. Inertia and the relative balance
of power among groups represent the two key factors internal to the
firm. Resistance to change favors reproduction of relative wages and
positions of individuals within a firm, whereas the resolution of
conflicts among groups within the firm may result into either reducing
or increasing inequalities within firms, both statically and over
time. External to the firm, the institutions shaping the environment
wherein firms operate constitute the third driving force of
inequalities. Internal and external forces always coexist and
constantly interact. Elaborating along these lines,~\cite{cobb2016}
provides an attempt towards a systematic theory for how firms
contribute to shape inequality. The building blocks are at the
intersection between environment-level characteristics -- epitomized
by the system of corporate governance prevailing in a country --, and
their interactions with internal distribution of power (among firms'
stakeholders). Our study speaks to this literature.  Indeed, examining
the levels of collective wage bargaining allowed for by labour and
industrial regulation in different countries and ``chosen'' by firms,
and how such interactions impact upon internal wage structure,
represents another interesting example to examine how institutional
and environment characteristics interact to determine inequality
outcomes. In particular, we draw from these approaches the notion that
potential conflicts of power exist within firms and are critical to
within-firm wage stratification. As we explain below, in fact, we use
different measures of within-firm inequality that try to differently
capture how different groups of employees (top vs. bottom paid, and
managers vs.~low-layers workers) may differently benefit from
firm-level bargaining.

The empirical literature exploring the relation between within-firm
wage inequalities and the level of collective bargaining, mostly by
economists, is limited and based on fairly old data, dating back to
the 1990s.~\cite{dellaringa.lucifora.1994} find that within-firm wage
dispersion does not differ between firms which only apply centralised
bargaining and firms which also apply firm-level agreements, in a
sample of Italian firms active in 1990. The result is confirmed
in~\cite{dellaringa.lucifora.ea.2004} for Italy, Spain, Belgium and
Ireland on data covering the year 1995: enterprises covered by a
single-employer agreement display greater \textit{unconditional}
within-establishment inequalities than multi-employer bargaining
firms, but such differences become statistically insignificant once
controlling for other factors and potential endogeneity of the choice
to bargain at the firm level.
Conversely,~\cite{canaldominguez.gutierrez.2004} find that firm-level
bargaining reduces within-firm wage dispersion in Spain, on data again
for the year 1995. Overall, these studies reflect the expectation that
multiple contrasting mechanisms may be at work.\footnote{A more recent
  study, posing a related although different question, is
  by~\cite{addison.koelling.ea.2014}, showing a modest widening of
  workplace wage dispersion across establishments that abandoned
  centralised (sector-level) collective bargaining in Germany over the
  period 1996-2008.}


The present work contributes to this relatively underdeveloped
empirical literature. We consider matched employer-employee data on
six European countries -- Belgium, Spain, France, Germany, the Czech
Republic and the UK -- available for the years 2006 and 2010, and
exploit the variation in collective bargaining models in place at
different firms in each country to shed light on three interrelated
research questions.

Our first and central contribution is to examine to what extent, in
the different countries and years spanned in the data, within-firm
wage inequality differs across firms which apply firm-level bargaining
on top of more centralised agreements as compared to firms which only
apply centralised bargaining. We employ two different measures of
within-firm wage dispersion: the inter-decile wage ratio (measured as
the 90\textsuperscript{th}-to-10\textsuperscript{th} percentile
wage-gap) is used to proxy the distance between the top~vs.~the bottom
part of the internal wage structure, while we consider the pay-gap
between managers and low-layers employees (manual workers and
elementary occupations) to capture whether the effect of firm-level
bargaining varies according to the occupation hierarchy within the
firm. Although one can expect some degree of overlapping between the
two measures, they do capture different aspects of wage inequalities
within firms. The occupational wage-gap, in particular, more directly
connects to whether firm-level bargaining works through the degree of
power or control on firms' decisions. In fact, there is a recent
renewed emphasis on the decline in wage premia of low-layers or
low-skilled workers (\citealt{song_etal2015NBER}) and the skyrocketing
wages of professionals and managers (\citealt{piketty,
  sabadish_mishel}). Whether the level of collective wage bargaining
negotiations contributed to these trends has not been yet investigated.

As theories and previous evidence suggest, many counteracting effects
are likely at work, preventing to have a sharp a-priori hypothesis on
whether firms which bargaining only at more centralised levels should
be expected to be more or less unequal than firms which also bargain
locally. Our analysis shall be informative about which forces prevail.
Beyond estimating the effect of firm-level bargaining on the two
measures of within-firm wage inequality, we also explore the effect of
firm-level bargaining on the components of the two wage-gaps. This
shall allow to uncover if an eventual statistically significant effect
of firm-level bargaining on internal wage structures arises from
favoring (or discriminating) certain categories of employees.

The second question we ask is whether the relations linking firm-level
bargaining to within-firm inequality exhibit comparable patterns
across countries. In line with theory, differences in environment and
institutional frameworks are central to establish in which direction
within-firm inequalities may evolve. As we detail below in Section~2
presenting the main features of collective bargaining models in the
selected countries, the room for maneuver warranted by firm-level
collective agreements vis-a-vis more centralised wage bargaining
significantly vary across the national contexts, depending on the
legal and institutional framework of the country where each firm
operates its industrial relations. Accordingly, it would be difficult
to predict that we shall observe exactly the same effect of firm-level
bargaining in all countries. However, despite country-level
specificities, some of the countries selected in our study can be
classified as sharing similar wage bargaining
regimes~\citep{fulton.2013}, in turn mapping into a sort of ``narrow''
version of a Varieties of Capitalism framework, associating countries
where firm-level bargaining is historically and still nowadays more
relevant and spread (the UK and the Czech Republic) to a
market-oriented model of capitalism, while putting under the common
heading of coordinated market economies those countries where
centralised forms of bargaining are historically and still nowadays
prevailing (Germany, France, Belgium and Spain).\footnote{See
  \cite{crouch2005} for a critical review of the Varieties of
  Capitalism literature, also discussing important refinements going
  beyond the original dualistic model distinguishing between
  market-liberal vs.~coordinated-market countries, as developed, e.g.,
  in~\cite{amable_book}.} In this respect, our analysis shall provide
a test for whether such a-priori taxonomies are supported in the
data, or it would otherwise suggest a different taxonomy, depending on
which countries turn out to exhibit comparable patterns in the way
firm-level bargaining shapes within-firm wage gaps.

Lastly, our third goal is to explore whether the (country-specific or
regime-specific) relations linking firm-level bargaining to (our two
measures of) within-firm inequality change over time, in between the
two years covered in our data (2006 and 2010). In the regulatory
framework of all countries we study, the legal provision to stipulate
firm-level collective agreements was already established before the
time span under analysis, but the years under study are those when the
major reforms introduced in late 1990s and beginning of 2000s likely
deployed their consequences. Also, the pressure toward assigning more
relevance and wider scope to firm-level negotiations increased over
time, as part of the broader tendency toward greater flexibility and
de-unionization of labour relations. Our intuition is that, along this
processes, an increasing use of firm-level agreements to differentiate
salaries established over the years spanned in the data, thereby
fueling the potential inequality-enhancing role of firm-level
bargaining. As a result, the likelihood to observe that firms
bargaining locally show more unequal wage structures than firms only
bargaining at more centralised levels, should increase over time. This
may be particularly the case in countries closer to coordinated-market
regimes, where the structure of industrial relations at more
centralised (national or industry) levels remains comparatively more
rigid and more complex to manage.  Whether this is the case may be
particularly interesting to assess also considering that the Great
Depression hit in between the two years available to us. Although we
do not claim to identify any causal effect related to the global
crisis, our results contribute to the discussion whether firm-level
agreements have been a
factor of amplification of inequalities in such a turbulent period.\\


The article is organised as follows. In Section
\ref{sec:bargaining_models} we describe the key features of the
wage-bargaining systems in the selected countries and provide some
hypotheses about the role of firm-level bargaining in the different
national contexts. In Section~\ref{sec:data} we introduce the data and
provide details on the definition of the main variables we use in the
empirical analysis. The empirical models and the estimation strategy
are next described in Section~\ref{sec:empirical}, while the
estimation results are presented in Section~\ref{sec:results}. We
discuss interpretations of results in the concluding
Section~\ref{sec:conclusion}.





\section{Firm-level wage bargaining across selected countries}
\label{sec:bargaining_models}

%\section{Wage setting structures in selected countries}
%\label{sec:bargaining_models}

The countries for which data available for this study -- Belgium, the
Czech Republic, Germany, Spain, France and the United Kingdom --
provide a good representation of the heterogeneity of the bargaining
regimes in Europe. We here present a brief description of the main
characteristics of collective bargaining systems featuring the various
countries.\footnote{We draw here from our own elaboration from a
  number of data-sources and reports, cited in the text.
  See~\cite{fulton.2013,fulton.2015} for a broader discussion of legal
  and institutional aspects featuring the bargaining systems of
  different countries.} This allows to sketch country-specific
hypotheses about whether we can expect differences in within-firm wage
inequality between firms which negotiate their wages only at more
centralised levels, and firms which choose to also apply firm-level
collective agreements. Then, at the end of the section, we reflect on
similarities and heterogeneities across countries and over time.

  
In Belgium, collective wage bargaining is highly structured with a
central level at the top covering the entire private sector, an
industry level covering specific industries, and company level
negotiations at the bottom. Wage bargaining takes place predominantly
at the national, cross-industry level. Notwithstanding two reforms
occurred during the period under analysis (as reported in the
\textit{Labour Market Reforms}-LABREF database maintained by the
European Commission\footnote{The LABREF dataset is available on-line
  at~https://webgate.ec.europa.eu/labref/public.}), the percentage of
employees covered by collective bargaining has remained steady at 96\%
over the period 2006--2010 (source: ILOstat database\footnote{See the
  section ``Industrial Relations'' of the ILOstat
  website~http://www.ilo.org/ilostat.}). According to the data from
the~\textit{European Company Survey}-ECS (run by the Eurofound
Industrial Relations Observatory), in 2009 66.08\% of companies apply
a collective agreement which has been negotiated at a higher level
than the establishment or the company, while 88.2\% of companies
applying national, inter-sectoral or sectoral collective bargaining
declare it was not possible for them to derogate from these
agreements. Elements of pay and work conditions -- including national
minimum wage, job creation measures, training and childcare provision
- are set in binding national agreements, while industry and company
bargaining mostly address non-pay issues, not affected by the ceiling
imposed by the central agreement~\citep{visser2013wage}. Opening
clauses -- which can allow companies to deviate from centralised
agreements -- are present since 1982, but firm-level negotiations
generally only agree on improvements upon what is settled at higher
levels. The room for pay bargaining at the enterprise level is also
limited due to indexation of wages in national agreements. As a
result, we do not expect firm-level bargaining to play a major role in
this country. The scope for local contracting to affect internal wage
structures is limited, with no major changes over time.


Spain and Germany present bargaining systems where wages are
predominantly set at the sector or industry level. The percentage of
employees covered by enterprise-level agreements amounts to less than
9\% in both countries in 2006 and such percentage does not
significantly change in 2010 (source: the \textit{Institutional
  Characteristics of Trade Unions, Wage Setting, State Intervention
  and Social Pacts}-ICTWSS database).

In Germany, wages are bargained mostly at the industry level between
individual trade unions and employers' organisations, although the
agreements allow for flexibility at the company level. Collective
agreements regulate a wide range of issues such as pay, shift-work
payments, pay structures, working time, treatment of part-timers and
training. Work councils play a central role because they can reach
agreements with individual employers on issues not covered by
collective agreements, or negotiate improvements on pay-related and
other issues already covered by collective agreements, under the
favourability principle. Opening clauses are present since 1993, and
they were mostly used to increase wages, as a monetary compensation
bargained by unions in exchange of more flexibility in covered
firms~\citep{brandle2013opening}. During the period considered in our
analysis, some reforms were implemented in the field of wage setting
policies, such as the introduction of binding minimum wages in several
sectors (LABREF data). However, the large prevalence of the higher
bargaining levels remains quite stable over time. According to the ECS
data, in 2009, the share of companies covered by forms of collective
agreement higher than firm-level bargaining was as high as 66.92\%,
and the possibility to derogate from these higher level agreements was
open to only a modest 17\% of the surveyed companies. Given these
features and the central role of workers/unions in the work councils,
typically pushing toward wage standardization, we could conjecture
that firm-level agreements signed on top of more centralised
bargaining are in this country especially likely to pursue egalitarian
purposes, thus aiming at compressing wage dispersion within firms.
However, there is also evidence that the use of opening clauses
produced significant wage worsening.~\cite{ellguth2012wage} estimate
that, on average, firms pay via these clauses a 7\% higher wage than
otherwise, but their use can lead to a wage reduction up to 9 \%. This
suggests that there may be room for firm-level bargaining firms to
display more unequal pay structures than other firms.

In Spain, the majority of firms (66.09\% in 2009 according to the ECS
dataset) report to negotiate their wages outside the firm, and the
structure of wage bargaining system shows a predominant role of
industry-level much like in Germany. But there are features that are
quite peculiar to this country. A first specific characteristic rests
in the complex coexistence and interaction of negotiations at national
and province-level, within industries. On top of this, firms adopting
firm-level collective bargaining in Spain traditionally feature a
higher presence of unions than multi-employer bargaining
firms~\citep{plasman.rusinek.ea.2007}, suggesting that in this country
the union's pressure to compress wage inequalities may be particularly
strong in firm-level bargaining firms. However, opening clauses are
present since 2001, and a significant labour market reform in
September 2010 enlarged the scope of such derogations (source: LABREF
database), allowing firm-level agreements to be more able to at least
potentially be used to increase inequality by lowering salary levels
and/or the amount of working time of some groups of workers, compared
to firms following centralised bargaining practices.

The Czech Republic and the United Kingdom represent two instances of
countries where collective agreements prevalently takes place at the
local -- firm or establishment -- level.

The UK is a paradigmatic case of the Anglo-Saxon tradition of
industrial relations, where wage bargaining is mostly un-coordinated,
with most workers bargaining work contracts individually with
employers. In fact, only about a third of all employees (33.3\% in
2006 and 30\% in 2010, according to ILOstat) is covered by some form
of collective bargaining. When a collective agreement occurs, the
majority of them are signed at the firm-level (53.4\% of companies in
2009, according to the ECS), but such agreements do not establish
legally binding norms and, as a rule, they contain no contractual
obligations such as opening clauses, they are not subject to legal
regulation, and pay rates cannot be claimed in
court~\citep{visser2013wage}. Also, collective agreements are very
rare in the private sector, while in the public sector workers'
coverage is more comparable to other countries~\citep{fulton.2013}.
This warrants public servants some more protection, although in May
2010 an emergency budget was approved freezing wages for high earners
in the public sector for a two-year period as a temporary measure to
face the 2008 global crisis (see LABREF data). Altogether, these
features make the UK a peculiar case in view of our aim at comparing
inequalities across firms which only apply centralised bargaining vis a
vis firms which also adopt firm-level bargaining. Considering the
traditionally high flexibility in the use and content of heterogeneous
pay schemes at firm-level in this country, we conjecture that
within-firm pay structures are particularly more unequal in firm-level
bargaining firms than centralised-bargaining firms in this national
context.


The Czech Republic well represents the tendency, spread across Eastern
Europe countries to embrace decentralised, market-oriented
institutional settings in the post Soviet Union era. In fact,
uncoordinated wage setting directly occurring between individuals firms
and individuals are quite spread, although less than in the UK. The
employees covered by collective wage bargaining are 50.8\% in 2006 and
50.1\% in 2010, according to ILOstat data. When collective agreements
are reached, they occur at firm level: more than 80\% of companies in
Eurofound-ECS dataset declare to have conducted negotiations of wages
at the firm or the establishment level. There exist a legal provision
of the favourability principle, since collective bargaining
regulations exclude opening clauses and derogations that set less
favourable terms than those provided in agreements stipulated at
higher levels. However, collective agreements signed at the industry
level last for at least two years, while those signed at company level
run for one year, thus allowing for a certain degree of flexibility in
reshaping the wage ladder in the enterprise. These features suggest
that in the Czech Republic, similarly to the UK, there is more room
than in other countries for firm-level bargaining to result into more
unequal within-firm pay structures.


The last country that we analyse, France, represents an outlying case,
due to its complexity, since all the levels of collective negotiations
-- intersectoral, industry or company -- are closely intertwined and,
in turn, they occur at both national or local
level~\citep{fulton.2013,fulton.2015}. Industry level bargaining is
the most important in terms of numbers of employees covered (97.3\% in
2006 and 98\% in 2010 according to ILOstat data).  More than 50\% of
companies declare to apply centralised bargaining in 2009 (see ECS
data), but the vast majority apply a combination of different levels.
The inversion of the favourability principle was introduced in 2004,
recognizing to firm-level agreements the possibility to derogate from
any condition settled at more centralised levels, if not explicitly
prohibited~\citep{keune2011decentralizing}.  This mostly concerned
working time, however, and few firms exploited the opportunity to act
on pay structures. All in all, the combination of elements pushing to
increase flexibility in the firm, with the enduring and complex role
of centralised bargaining levels, makes particularly difficult to
provide predictions.\\

Beyond providing some intuitions on the different effect that
firm-level bargaining may exert on within firm inequality, the basic
elements characterizing different bargaining systems suggest further
considerations noteworthy for the interpretation of our empirical
analysis.

A first qualification pertains the implications of country
specificities for the intertemporal changes we could expect to observe
in the estimated effect of firm-level bargaining. As argued in the
introduction, we hypothesize that firm-level bargaining is likely to
have gained an increasing role, and thus its potentially
inequality-enhancing effects to become more likely to manifest over
time in all countries. Yet, as the discussion of bargaining systems
here above shows, major reforms did not take place in the period under
study. In fact, firm-level bargaining was already legally established
in all countries well before the initial year of our analysis. In this
respect, any intertemporal change that we shall uncover in the
relation between firm-level bargaining and within-firm wage
inequalities should be taken as mirroring changes in the use of
firm-level agreements, and not as a test of what happens when firms in
a country are given the opportunity to move from a fully centralised
system to fully decentralised wage bargaining.

An additional issue, important for subsequent interpretation our
empirical analysis, regards the relative balance between the
similarities and the heterogeneities observed comparing
wage-bargaining systems. From the discussion here above, clearcut
differences emerge in terms of bargaining coverage, structures and
mechanisms of coordination: national, sectoral and company bargaining
do not operate equally in all countries, and the relative diffusion,
scope and content of firm-level collective bargaining are highly
heterogeneous. Our choice to perform separate analysis by country
exactly accomplishes the need to recognize that the choice of a firm
to negotiate at firm level on top of more centralised levels has
different meaning in different countries.

However, complementary to the focus on heterogeneities, one could also
argue that interesting hypotheses may be derived from considering that
different sets of countries do share broad common tendencies. In
particular, countries can be grouped based on the prevailing locus
where collective bargaining occurs, with Belgium being a paradigmatic
example of the ``inter-industry/national regime'', the UK and the
Czech Republic representing instances of an opposite
``individual-employer model'', Spain and Germany falling into the
intermediate ``sectoral model'', and France somehow outlying due the
specifically complex interaction across all
levels~\citep{fulton.2013}. Different hypotheses can developed about
the role of firm-level bargaining in shaping within-firm inequalities
in these different regimes or models. On the one hand, it may be
argued that where more centralised or complex models prevail, there is
stronger resistance (by the laws or due to workers' action) to allow
for firm-level contracts to introduce inequalities in the firm
internal wage structure. This would lead firms bargaining locally to
be characterised by an only mildly higher inequality -- if at all
present -- as compared to firms which bargain at higher levels in
countries like Belgium or France, and to a less extent also in Spain
and Germany. However, it may also be the case that firm-level
bargaining is used more markedly by firms to differentiate their pay
structures to escape the rigidities, complexities of negotiations and
the greater pressures toward wage standardization that exactly
characterize the more centralised regimes. Eventually, the process of
progressive decentralization of wage-setting was justified precisely
to allow firms more freedom to shape internal incentives as compared
to the limited margins of maneuver allowed for by corporatist
industrial relations. If this second tendency prevails, we could
expect firm-level bargaining firms to display larger within-firm
inequalities than other firms also in countries like Spain, Germany,
France or Belgium. As mentioned in the introduction, this grouping by
prevailing level of bargaining eventually ends up evoking a Varieties
of Capitalism type of reasoning. The prediction would be that
firm-level bargaining is more likely to increase inequality in
market-oriented, flexibility-friendly countries like the UK and the
Czech Republic, than in the other countries we study, where a
coordinated-market type of capitalism is reflected in their more
centralised bargaining systems.





\section{Data and main variables}
\label{sec:data}

The \emph{Structure of Earnings Survey}~(\textsc{ses}) dataset
collected by Eurostat is a well-known source of information for labour
dynamics across Europe. It collects a rich number of earnings-related,
personal and jobs-related variables for a vast set of workers, matched
with information on some characteristics of the employing firms. For
this study, we had access to the 2006 and 2010 waves of the
\textsc{ses} for Belgium, Germany, Spain, France, the Czech Republic
and the United Kingdom. We pool the two waves of the survey in the
empirical analysis, but the pooled data must be intended as a repeated
cross-section, since the \textsc{ses} does not report any
identification code that can be used to match the same firm or the
same employee over time.

The structure of the \textsc{ses} data is such that, for each country,
a random sample of firms (stratified by size, sector of activity and
geographical location) is selected to be representative of the
national industrial system. Then, within each selected firm, a
representative sample of employees is drawn, and for those employees a
large set of personal and job-related characteristics is provided,
including age, gender, education, wages, type of contract, tenure,
occupation type (according to the 2008 International Standard
Classification of Occupations, \textsc{isco}), and others. As such,
the \textsc{ses} data can be seen as a matched employer-employee
dataset, representing a unique source for a consistent comparison
across European economies, indeed repeatedly used in previous studies.
Of course, the dataset has its own limitations. First, while the
surveying procedure provides information on an impressive number of
workers across Europe (about 10 million per survey year), for the
firms which enter the data the sampling rate of employees varies by
firm size and by country. Second, the sample of business units
considered in the survey is restricted to those with at least 10
employees, which limits the analysis as far as micro firms are
concerned. Third, the data are very rich concerning employees'
personal and work-related characteristics, but the information on
firms is limited to five variables: size class, geographical location,
sector of activity, public vs. private control and -- crucial for our
purposes -- the level of wage bargaining adopted in the firm.

The outcome variables of interest are two measures of
\emph{within-firm} wage inequalities. For each firm $j$, we first
consider the ratio between the 90\textsuperscript{th} and
10\textsuperscript{th} percentile of the wages paid to the
employees of the firm
\begin{equation}
  \label{eq:def_disp_90_10}
  \Delta w^{90/10}_j=\dfrac{w^{90}_j}{w^{10}_j} \;\;,
\end{equation}
yielding a characterization of the wage distance between top and
bottom earners within the firm. This is in line with previous
empirical studies, which in fact discuss a purely statistical
characterization of within-firm wage inequalities.

Second, and departing from the literature, we provide an
occupation-related characterization of inequalities, considering the
ratio between the average wage of managers and average wage of workers employed in
low-layers occupations
\begin{equation} 
\label{eq:def_disp_jobs}
\Delta w^\mathrm{jobs}_j= \dfrac{\mathbb{E}\left(w^\mathrm{Managers}_j \right)}{\mathbb{E}\left(w^\mathrm{Low}_j \right)} \;\ .
\end{equation}
Information on employees' occupation is reported in the
\textsc{ses} data according to the \textsc{isco} categories, at
1-digit level. We take employees with \textsc{isco} code~1
(``managers'') to define apical managerial positions, while low-layers
workers include employees with \textsc{isco} code~8 (``plant and
machine operators, and assemblers'') or~9 (``elementary
occupations'').

The two measures of inequality may correlate to some extent, but they
may reveal different facets of earnings inequality resulting from
firm-level bargaining. The first measure relates to the more standard
question whether company-level agreements are used selectively across
employees differently positioned in the within-firm wage distribution.
The occupational wage-gap, instead, allows us to ask whether
firm-level negotiations favor or reduce inter-occupational wage
differences in relation to the hierarchical jobs structure within the
enterprise.

Following an established practice in the literature~\citep[at least
since][]{winter.ebmer.1999}, in order to compute the two measures of
wage inequality we start from adjusted residual wages. That is, the
wages $w$ that enter the two definitions above are the residuals from
an augmented Mincerian wage-regression

\begin{equation}
\label{eq:mincer}
\log \left( \hat{w}_{ij} \right) = 
    b_0 + b_1 \bm{Z}_{i} + b_2\, \mathrm{Firm}_j + \varepsilon_{ij}
\end{equation}
where the (log-)wage reported in the data for employee $i$ of firm
$j$, $\hat{w}_{ij}$, is regressed against a standard set of individual
characteristics $\bm{Z}_{i}$ (age, tenure and tenure squared, gender,
education, contract duration, part-time status, share of full-timer's
hours, and occupation at 1-digit~\textsc{isco}), plus firm
fixed-effects, $Firm_j$. Separate regressions are estimated
by year (2006 and 2010) and sector (one digit \textsc{nace}), within
each selected country.

As the proxy for observed wages, $\hat{w}$, we use hourly wages. In
\textsc{ses}, these are recorded as the compensation actually paid to
the workers, without distinguishing between the wage components that
are set through firm-level bargaining from the components agreed upon
at more centralised levels. In particular, as it is often the case in
the literature, we do not have information on un-bargained wage
drifts. These are parts of compensation granted by firms to specific
employees (or group of employees) outside collective bargaining,
whatever the level of collective bargaining adopted by the
enterprise.~\cite{cardoso.portugal.2005} find for Portugal that such
unilateral components increase wage inequalities within firms,
although the theoretical possibility remains open that wage drifts --
much in line with the mechanisms that may lie behind firm-level
collective bargaining -- operate to re-balance the internal pay
structure, for instance for fairness reasons. Also, although
unilateral wage drifts may affect in principle all types of firms,
they are expected to be stronger and more frequent in firms which only
bargain at national or industry level (e.g. when allowed for via
opening clauses), as a way to gain flexibility and adjust the internal
wage structure vis-a-vis the centralised agreements, but without going
through collective bargaining at
firm-level~\citep{dellaringa.pagani.2007}. If this is the case, then
we expect such wage drifts to increase within-firm inequalities less
in firms which bargain locally.


The construction of within-firm inequality measures, as well as the
estimation of residual wages, requires by definition that a minimum
number of employees per firm is present in the sample. In particular,
the professional wage-gap in Equation~\ref{eq:def_disp_jobs} implies
that at least one manager and one low-layer employee are sampled
from the same firm. After careful consideration of alternative
restrictions to the data, and sensitivity analysis about robustness of
main results, we define our working sample as including only firms
with at least three sampled employees.\\


% As our main explanatory variable, according to our goal to estimate
% the incremental effect of firm-level collective agreements on top of
% centralised collective negotiations, we build a dummy that
% distinguishes firms which only adopt centralised bargaining vs. firms
% that also apply firm-level bargaining. 

As our main explanatory variable we build a dummy that distinguishes
firms that only adopt centralised bargaining vs. firms that also apply
firm-level bargaining. This allows us to respond our key research goal
to estimate the incremental effect of firm-level collective agreements
on top of centralised collective negotiations. The variable in
\textsc{ses} recording the type of wage bargaining in place at each
firm, precisely reflects the incremental engagement of firms in
different bargaining levels. It distinguishes firms which do not apply
any form of collective bargaining, and then splits those which do
negotiate collective agreements into two groups: firms which only
negotiate at more centralised levels and firms which also apply
firm-level bargaining on top of higher level negotiations.
Accordingly, we created a dummy variable $\mathrm{FLB}$ taking value 1
for the latter group, and zero for the former. More precisely, firms
defined as bargaining only at centralised, multi-employer levels
($\mathrm{FLB}=0$) apply wage agreements classified by Eurostat as
``national level or inter-confederal agreement'', ``industry
agreement'', or ``agreement for individual industries in individual
regions''. Firms which we define as engaging in firm-level bargaining
($\mathrm{FLB}=1$) also subscribe agreements classified as
``enterprise or single employer agreements'' or ``agreements applying
only to workers in the local unit'', on top of one or more of the
above centralised contracts.

Table~\ref{tab:summary_stats} shows the number and percentage shares
of employees and firms falling in different categories of bargaining
in our working sample, by country and by year, also providing
information on firms which do not apply any form of collective
bargaining, (\emph{i.e.,} contract wages separately with each single
employee). In line with discussion of country-specific bargaining
regimes, the percentage of firms or employees covered only by more
centralised levels of bargaining is generally higher in all other
countries than in the UK or in the Czech Republic, whereas in these
two latter cases ``no-collective bargaining at all'' prevail.

Table~\ref{tab:prelim_ols} shows a basic difference-in-means test
obtained by running a simple OLS regression of the two measures of
(residual) wage inequality $\Delta w^{90/10}$ and $\Delta
w^\mathrm{jobs}$ against the $\mathrm{FLB}$ dummy and a constant term.
Considering the percentile wage-dispersion $\Delta w^{90/10}$, firms
that adopt firm-level bargaining present, on average, higher
inequality in Belgium and Spain, while lower dispersion in France, the
United Kingdom, and the Czech Republic (except in 2010). In Germany,
instead, the average $\Delta w^{90/10}$ do not differ statistically
between the two groups of firms, in both 2006 and 2010. In terms of
the professional wage-gap $\Delta w^\mathrm{jobs}$, the more common
pattern is that firms bargaining locally display lower inequalities
than other firms, although we observe insignificant coefficients on
the $\mathrm{FLB}$ dummy in some country-year combinations, and a
positive coefficient for the Czech Republic in 2010. This exercise
just provides a first descriptive assessment of the unconditional
relation linking firm-level bargaining and wage inequalities. In the
next session, we present the empirical framework that we design in
order to obtain more reliable estimates, controlling for additional
observables that may drive the differences in wage inequalities and
for potential endogenous selection of the $\mathrm{FLB}$ dummy.




\section{Empirical models and estimation strategy}
\label{sec:empirical}

% Our main research question pertains the identification of whether
% firms which apply firm-level collective bargaining on top of other
% forms of collective bargaining at more centralised levels, display
% higher or lower within-firm wage inequalities, and the variation of
% this relation across countries and over time. To this purpose, we
% pool the observations over $t$=2006 and $t$=2010, and specify the
% following baseline regression model

To identify the effect of firm-level bargaining across countries and
over time, we pool the observations available for each country over
the years $t$=2006 and $t$=2010, and specify the following baseline
regression model

\begin{equation}
\label{eq:reg_dispersion}
  \Delta w^d_{jt} = \alpha + \beta_1\, \mathrm{FLB}_{jt} + \beta_2\, \mathrm{Y}_{2010} + \beta_3\, \mathrm{Y}_{2010} \times \mathrm{FLB}_{jt} + \gamma\bm{X}_{jt} + \epsilon_{jt} \;.
\end{equation}
The dependent variable $\Delta w^d_{j}$ is, alternatively, one of the
two measures of (residual) wage inequality $\Delta w^{90/10}$ or
$\Delta w^\mathrm{jobs}$, computed as explained above for each firm
$j$ present in each survey year $t$ (2006 or 2010). The set
$\bm{X}_{jt}$ includes control variables (discussed further below).
The regressor of primer interest is the dummy $\mathrm{FLB}$
indicating if firm $j$ applies firm-level collective bargaining in the
year $t$, which we include both as a stand-alone variable and
interacted with the dummy $\mathrm{Y}_{2010}$ set to 1 for the year
2010, accounting for possible time-varying effects of firm-level
collective bargaining over the two survey years. That is, conditional
on the control variables included in the set $\bm{X}_{jt}$, the
coefficient $\beta_1$ accounts for the difference in average wage
inequalities between firms bargaining locally in 2006 compared to
those that do not. The interaction coefficient $\beta_3$ captures
whether the effect of collectively bargaining at the firm level
in~2010 changes as compared to 2006.

% To shed light on this underlying dynamics, we explore the relation
% between firm-level bargaining and the components (numerator and
% denominator) of the wage-dispersion measures $\Delta w^{90/10}$ and
% $\Delta w^{jobs}$. 

As a further contribution, we provide a dissection of the effects of
firm-level bargaining on the wages of the groups of employees that we
implicitly compare in (numerator and denominator of) the wage-gaps
$\Delta w^{90/10}$ and $\Delta w^\mathrm{jobs}$. We estimate the
following variation of the specification in
Equation~\eqref{eq:reg_dispersion}

\begin{equation}
\label{eq:reg_decomposition}
\ w^d_{jt} = \alpha + \beta_1\, \mathrm{FLB}_{jt} + \beta_2\, \mathrm{Y}_{2010} + \beta_3\, \mathrm{Y}_{2010} \times \mathrm{FLB}_{jt} + \gamma\bm{X}_{jt} + \epsilon_{jt} \;,
\end{equation}
where as dependent variable $w^d_{jt}$ we employ, alternatively, the
90\textsuperscript{th} or the 10\textsuperscript{th} percentile of the
within-firm distribution of (residual) log-wages, or take the average
(residual) log-wages of managers and of low-layers employees. In line
with Equation~\eqref{eq:reg_dispersion}, the identification works
across firms with different wage-bargaining. Thus, the estimates of
the coefficient $\beta_1$ on the $\mathrm{FLB}$ dummy gives the
difference in average outcomes across firm-level bargaining vs.~other
firms in 2006, whereas the coefficient $\beta_3$ on the interaction
term $\mathrm{FLB} \times Y_{2010}$ accounts for changes in the FLB
effect over time. Notice that these separate regressions on the
components of the two wage-gaps $\Delta w^{90/10}$ and $\Delta
w^{jobs}$ do not correspond to an exact split of the overall effects
estimated from Equation~\eqref{eq:reg_dispersion} regressions.
Nonetheless, the results are revealing of the underlying driving
forces, telling which group of employees gains or loses from
firm-level collective bargaining. Indeed, it matters whether a
hypothetical increase in $\Delta w^{90/10}$ in firms bargaining
locally comes from a choice to to pay highest earners even more,
rather than the same increase resulting from paying low earners even
less. In the two cases, the diverging interests between different
groups of employees within the firm are clearly solved in opposite
ways.  Similarly, were firm-level bargaining to have any effect on the
professional wage-gap $\Delta w^{jobs}$, then it would be relevant to
understand who benefits or
loses between managers and low-layers workers.\\

A common empirical strategy is followed in estimating the regression
models in Equations~\ref{eq:reg_dispersion}
and~\ref{eq:reg_decomposition}.

First, as mentioned, all the models are estimated separately country
by country. This strategy, as opposed to estimating a pooled model
with country fixed effects, allows to properly account for differences
in bargaining systems across countries. In fact, while the definition
of firm-level bargaining firms (FLB=1) is homogeneous in \textsc{ses}
across all countries, there is great variation across countries about
what alternative type of bargaining level is likely to prevail in
the control group of firms which do not apply firm-level bargaining
(FLB=0). By allowing coefficient estimates to vary by country, we
avoid any assumption of homogeneity across national institutional
settings.

Second, we include the same set of controls $\bm{X}_{jt}$ in both
regression~\ref{eq:reg_dispersion} and~\ref{eq:reg_decomposition},
accounting for a large number of other determinants of wage inequalities,
beyond firm-level bargaining. Building on previous literature, wage
dispersion within firms depends on firm characteristics as well as on
personal and professional characteristics of the workforce. The
\textsc{ses} data allow to control for a variety of these confounding
factors suggested by previous studies. As far as firm attributes are
concerned, in~\textsc{ses} we have information on firm size (as
size-class by number of employees), and a dummy for private vs. public
control on the firm. In general, the expectation is that within-firm
wage dispersion is lower in large and publicly owned firms, as the
unions tend to be more powerful in these
contexts~\citep{canaldominguez.gutierrez.2004}. Moreover, thanks to
information on sector of main activity and geographical location of
each firm, we can also control for the well-known variation of both
wages and incidence of bargaining level across sectors and regions,
via a full set of sector (reported in SES at 1-digit \textsc{nace})
and regional (reported at \textsc{nuts}-1 level) fixed-effects.

Concerning personal characteristics of the workforce, previous studies
stress the relevance of gender, age, education, and experience. We
capture all these features, by including in the empirical model the
share of women employed in the firm, the share of employees with
secondary or tertiary education, the mean tenure of workers in the
firm, and a set of dummies for modal age of the workforce. Usually,
wage dispersion is expected to rise with age, tenure and education,
because wages tend to increase in all these characteristics, and
dispersion is usually higher in firms where average wages are
higher~\citep{canaldominguez.gutierrez.2004}. As for gender, the
well-documented existence of wage-gaps favourable to males would
suggest that larger inequalities are to be expected in firms where the
proportion of females is lower.

The type of jobs and contracts present in the firm are also known to
play a role. Unions' efforts to push for equalization of wages among
their members is usually identified as the channel trough which
within-firm wage differences are influenced by factors like having a
permanent vs.~a fixed-term contracts, a full-time vs. a part-time job,
or the relative weight of blue-collars vs.~more professionalised
occupations in the firm. Since full-time, permanent, blue-collar
workers are generally more likely to unionize, earnings inequalities
are expected to be lower in firms with a larger proportion of these
job and contract types~\citep{canaldominguez.gutierrez.2004}. We
control for these factors by including, for each firm, the share of
managers and professionals (according to 1-digit \textsc{isco} codes 1
and 2), the share of part-time employees and the share of employees
with a permanent contract.

Notice, however, that the correlations between the workforce
characteristics and the measures of within-firm inequality may be
complicated by unobservable compositional effects. In fact, employees
with different characteristics may fall more or less frequently into
the wage groups that we compare (percentile or occupation-related
wages). For instance, notwithstanding the gender pay-gap, a firm with
a 100\% share of males can be more equal than a firm with a single
female employee, to the extent that all the males employees earn the
same wage (or quite similar wages) in the former firm. An equal
reasoning may replicate for the other controls measuring features like
age, tenure, job and contract types, and so on, in turn suggesting
predictions in contrast with findings in the
literature.\footnote{Basic descriptive statistics on control variables
  are presented in Appendix~A. Notice that some of the controls are
  not available for the Czech Republic. First, in the data there are
  no Czech firms with modal employees' age in the range 20-29 years
  old, so we omit this age category. Second, the Czech Republic
  defines a single \textsc{nuts}-1 region, so we cannot further
  exploit regional dummies in the estimates for this country.}
 

Finally, and perhaps most important, in estimating both
regression~\ref{eq:reg_dispersion} and~\ref{eq:reg_decomposition}, we
address the potential endogeneity of the FLB dummy, due to non-random
endogenous selection of firms between FLB and ``non-FLB'' status.
Indeed, despite we (i) control for employer-specific components of
wages and firm-level average wages trough the preliminary Mincerian
regression, and (ii) include a rich set of covariates, still there
might be unobserved determinants of the decision to apply firm-level
collective agreements that correlate with unobserved determinants of
the dependent variables of interest in each regression equation. 

In order to tackle this source of bias, we follow a solution commonly
adopted in the empirics of within-firm wage
inequality~\citep{card.delarica.2006,daouli.demoussis.ea.2013}. That
is, we augment the model with a preliminary estimate of the
probability (propensity score) that a given firm adopts firm-level
collective bargaining. This is obtained from a preliminary first-step
Probit

\begin{equation}
\label{eq:reg_propensity}
  \mathrm{FLB}_j = \mathrm{P}\left( \alpha_0 + \alpha_1\bm{V}_j \right)
\end{equation}
where $\mathrm{FLB}_j$ is the dummy for the \emph{observed} presence
of firm-level bargaining in firm $j$, $\mathrm{P}$ is the Probit link
function, and $\bm{V}$ a set of covariates that affect the choice to
bargain at firm-level. Separate first-step Probit regressions are
estimated country by country, and the corresponding predicted
probabilities $\widehat{\mathrm{FLB}}_j = \mathrm{P}\left(
  \hat{\alpha}_0 + \hat{\alpha}_1\bm{V}_j \right)$ obtained for each
firm are then included as an additional control variable in a
second-step estimation of the main regressions in
Equations~\ref{eq:reg_dispersion} and~\ref{eq:reg_decomposition}. The
overall logic is that if FLB status is as good as randomly assigned
conditional on observed controls, then conditioning also upon the
propensity scores allows to clean any further bias due to unobserved
firm characteristics, and, thus, a simple OLS on the second step will
return correct estimates of the FLB dummy coefficient. The predictors
$\bm{V}$ are for the most part the same as the controls appearing in
the set $\bm{X}$ in the main equations. However, to ease
identification, we exclude average tenure of the workforce, as it is
sensible to assume that tenure affects wages and wage inequalities,
but it does not directly impact the decision to adopt FLB. Also notice
that, in place of sector and regional fixed-effects included in the
set $\bm{X}$ (likely subject to incidental parameter problems in
Probit estimates), the set of covariates $\bm{V}$ includes the
\textsc{gdp} per capita (at purchasing power parity, base year 2006)
and the unemployment rate in the region where each firm is located,
thus controlling for macroeconomic-and-regional dynamics that may play
a direct influence on the decision to apply firm-level
bargaining.\footnote{These additional variables are taken from
  EUROSTAT-Regional Statistics and measured at the level of
  \textsc{nuts-1} regions, since this is the precision of the
  information on firms' geographical location in \textsc{ses}. The
  results of the first-step Probit regressions are reported in
  Table~\ref{tab:propensity} in Appendix~B. They show a satisfactory
  goodness of fit, in terms of relatively high values of the area
  under the ROC curve.}




\section{Results}
\label{sec:results}

\subsection{Firm-level bargaining and the 90\textsuperscript{th}-to-10\textsuperscript{th} percentile wage-gap}

% We start by presenting the analysis of the effect of firm-level
% bargaining on the wage differences between top and bottom extremes of
% the wage distribution within firms.

Table~\ref{tab:disp_90_10} shows the estimates of the specification of
Equation~\ref{eq:reg_dispersion} where we take the
90\textsuperscript{th}-to-10\textsuperscript{th} percentile wage-gap
$\Delta w^{90/10}$ as the dependent variable. In general, they suggest
that firm-level bargaining has heterogeneous effects on wage
dispersion, both across countries and over time. In 2006 (cf.~the
coefficients on the $\mathrm{FLB}$ dummy) we do not observe
statistically significant differences between firms which adopt
firm-level bargaining as compared to other firms in any country but
the UK, where firm-level bargaining firms are less unequal. This
initial picture observed for 2006 does not change in 2010 in four
countries (Belgium, Germany, the Czech Republic and the UK, cf.~the
insignificant interaction coefficients).  Conversely, we detect a
common inter-temporal pattern in France and Spain, whereby the
distribution of wages becomes more unequal over time in firm-level
bargaining firms (positive estimated interaction coefficients).

The estimates reveal heterogeneities also with regard to the
correlation between the $\Delta w^{90/10}$ wage-gap and control
variables. Starting from workforce characteristics, the modal age of
employees shows a largely insignificant association with wage
inequality in Belgium, Germany, Spain and France, while the relation
with $\Delta w^{90/10}$ is stronger (positive) in the Czech Republic
and the UK. A common result across all countries is that within-firm
wage inequality is larger for firms with the most senior workforce
(60+ years old). The share of women in the workforce, the average
on-the-job tenure of employees and the share of permanent contracts
show a negative association with within-firm wage dispersion in most
countries, while educational levels, the share of part-time employees
and the share of higher professional occupations tend to display a
positive association (when significant) with within-firm wage
dispersion. Further, moving to firm-level characteristics, wage
dispersion within firms increases with firm size in Germany, Spain and
France, but larger firms display lower wage dispersion than the
baseline in the UK. Publicly-controlled firms feature lower wage
dispersion compared to private firms in Belgium, the Czech Republic
and France. Note, lastly, that the significant coefficient on the
propensity score $\widehat{\mathrm{FLB}}_j$ confirms the need to
correct for endogenous selection into $\mathrm{FLB}$ in most
countries. This latter result holds throughout all the estimates of
this article, although we do not stress it in the following.\\


Next, we dissect the separate effect of firm-level bargaining on the
90\textsuperscript{th} and the 10\textsuperscript{th} percentile of
the within-firm distribution of wages. Table~\ref{tab:disp_avg_90_10}
reports the estimates of the corresponding specifications of
Equation~\ref{eq:reg_decomposition}.  

We highlight three main patterns. First, consider the three countries
(Belgium, the Czech Republic and Germany) where firm-level bargaining did
not have any significant effect in the analysis of $\Delta w^{90/10}$
shown above in~Table~\ref{tab:disp_90_10}. In Belgium and the Czech
Republic, the analysis by percentiles confirms that firm-level
bargaining does not have any statistically significant effect. In
Germany, conversely, firm-level bargaining firms feature both a lower
90\textsuperscript{th} percentile and a lower 10\textsuperscript{th}
percentile than other firms in 2006 (cf.~the coefficient on FLB
dummy), and these differences do not change over time (insignificant
interaction coefficients). The magnitudes of the FLB effect on the two
percentiles are comparable, hinting at why we do not see, in
Table~\ref{tab:disp_90_10}, an overall statistically significant
difference in the $\Delta w^{90/10}$ wage-gap between FLB and other
firms. Yet, the underlying dynamics seem to be that the adoption of
firm-level agreements in Germany reduces wages at both the top and the
bottom extreme of the within-firm wage distribution.

Second, the analysis by percentiles suggests that the
equality-enhancing effect of firm-level collective bargaining on
$\Delta w^{90/10}$ observed in Table~\ref{tab:disp_90_10} for the UK
mainly works through FLB practices favouring low-paid workers. Indeed,
we find that, in the UK, firm-level bargaining firms show an higher
10\textsuperscript{th} percentile wage than other firms in both 2006
and 2010.

Third, and finally, the results also add to the understanding of the
increasingly detrimental effect of firm-level bargaining over time, as
detected in~Table~\ref{tab:disp_90_10} for Spain and France. In both
countries, indeed, we observe a clear divergence in the patterns of
high-paid and low-paid employees: over time, the employees in the
90\textsuperscript{th} percentile are paid significantly more in
firm-level bargaining firms than in other firms, whereas the opposite
holds for employees in the 10\textsuperscript{th} of within-firm wage
distribution. In Spain, in particular, the 10\textsuperscript{th}
percentile wages in firms adopting firm-level agreements are lower
than 10\textsuperscript{th} percentile wages paid by other firms
already in 2006.



% \begin{landscape}
% \centering
% \begin{table}[htb]
% \caption{Decomposition of the effects on the 90\textsuperscript{th} and 10\textsuperscript{th} percentile wages}
% \label{tab:disp_avg_90_10}
% \centering
% \centering
\resizebox{\textwidth}{!}{%
\begin{threeparttable}
% \setlength{\tabcolsep}{0pt}
%
\begin{tabular}{l*{12}{S}}
\toprule
                          % & (1)         & (2)         & (3)          & (4)        & (5)        & (6)         & (7)         & (8)        & (9)        & (10)       & (11)         & (12)       \\
                          & \multicolumn{2}{c}{BE} & \multicolumn{2}{c}{DE} & \multicolumn{2}{c}{ES} & \multicolumn{2}{c}{CZ} & \multicolumn{2}{c}{UK} & \multicolumn{2}{c}{FR} \\
 & \multicolumn{1}{c}{q\_90} & \multicolumn{1}{c}{q\_10} & \multicolumn{1}{c}{q\_90} & \multicolumn{1}{c}{q\_10} & \multicolumn{1}{c}{q\_90} & \multicolumn{1}{c}{q\_10} %
                          & \multicolumn{1}{c}{q\_90} & \multicolumn{1}{c}{q\_10} & \multicolumn{1}{c}{q\_90} & \multicolumn{1}{c}{q\_10} & \multicolumn{1}{c}{q\_90} & \multicolumn{1}{c}{q\_10} \\

\midrule
$\beta_0$: Intercept             &       0.226***    & -0.228***   & 0.270***     & -0.259***  & 0.148***   & -0.151***   & 0.244***    & -0.219***  & 0.307***   & -0.290***  & 0.315***     & -0.309***  \\
                                &       (0.0203)    & (0.0173)    & (0.0289)     & (0.0265)   & (0.0134)   & (0.0112)    & (0.0223)    & (0.0195)   & (0.0722)   & (0.0571)   & (0.0234)     & (0.0173)   \\[1ex]
$\beta_1$: FLB                  &       -0.00106    & 0.000520    & -0.00945***  & -0.00660*  & -0.000250  & -0.00535**  & -0.00479    & 0.00548    & -0.00414   & 0.00877*** & -0.00187     & 0.00292    \\
                                &       (0.00276)   & (0.00244)   & (0.00343)    & (0.00394)  & (0.00260)  & (0.00239)   & (0.00521)   & (0.00535)  & (0.00373)  & (0.00320)  & (0.00500)    & (0.00601)  \\[1ex]
$\beta_2$: Year 2010            &       -0.0189***  & 0.0145***   & 0.00678***   & 0.00223    & -0.0174*** & 0.0151***   & -0.00548    & 0.00871    & -0.0392*** & 0.0406***  & -0.00764***  & 0.00741*** \\
                                &       (0.00158)   & (0.00135)   & (0.00236)    & (0.00262)  & (0.00139)  & (0.00131)   & (0.00623)   & (0.00728)  & (0.00530)  & (0.00419)  & (0.00202)    & (0.00162)  \\[1ex]
$\beta_3$: FLB$\times$2010      &       0.00159     & 0.000108    & 0.00578      & 0.00255    & 0.0148***  & -0.00693**  & 0.00125     & -0.00371   & -0.00431   & -0.000113  & 0.0195***    & -0.0167*** \\
                                &       (0.00332)   & (0.00298)   & (0.00451)    & (0.00452)  & (0.00346)  & (0.00316)   & (0.00693)   & (0.00771)  & (0.00533)  & (0.00443)  & (0.00600)    & (0.00598)  \\[1ex]
$\gamma$: Prob. FLB              &       0.0539***   & -0.0415**   & -0.0153      & -0.0149    & -0.133***  & 0.0928***   & 0.0595**    & -0.0576*** & -0.0112    & -0.00350   & 0.0678***    & -0.0374**  \\
                                &       (0.0173)    & (0.0171)    & (0.0238)     & (0.0296)   & (0.0151)   & (0.0129)    & (0.0259)    & (0.0212)   & (0.0739)   & (0.0573)   & (0.0183)     & (0.0167)   \\[1ex]
\cmidrule(lr){1-13}
Modal age workers:             \\[1ex]
\quad \textit{20-29}            &       -0.0141     & 0.0156      & -0.00923     & -0.0423*   & 0.00651    & -0.00356    &             &            & 0.00736    & 0.00353    & -0.0306      & 0.0348**   \\
                                &       (0.0136)    & (0.0117)    & (0.0251)     & (0.0244)   & (0.0131)   & (0.0109)    &             &            & (0.00635)  & (0.00660)  & (0.0211)     & (0.0165)   \\[1ex]
\quad \textit{30-39}            &       -0.00464    & 0.00867     & -0.0135      & -0.0339    & 0.0153     & -0.0111     & 0.0183***   & -0.0180*** & 0.0267***  & -0.0111*   & -0.0250      & 0.0308*    \\
                                &       (0.0138)    & (0.0115)    & (0.0250)     & (0.0239)   & (0.0129)   & (0.0108)    & (0.00523)   & (0.00603)  & (0.00684)  & (0.00663)  & (0.0209)     & (0.0166)   \\[1ex]
\quad \textit{40-49}            &       0.00105     & 0.00418     & -0.0175      & -0.0331    & 0.0108     & -0.00687    & 0.00668     & -0.00533   & 0.0315***  & -0.0151**  & -0.0249      & 0.0329**   \\
                                &       (0.0139)    & (0.0116)    & (0.0249)     & (0.0238)   & (0.0131)   & (0.0108)    & (0.00628)   & (0.00640)  & (0.00635)  & (0.00674)  & (0.0208)     & (0.0166)   \\[1ex]
\quad \textit{50-59}            &       0.0127      & -0.00702    & -0.0167      & -0.0291    & 0.0101     & -0.00561    & 0.00820     & -0.00701   & 0.0332***  & -0.0138*   & -0.0146      & 0.0243     \\
                                &       (0.0141)    & (0.0118)    & (0.0252)     & (0.0237)   & (0.0133)   & (0.0110)    & (0.00538)   & (0.00583)  & (0.00681)  & (0.00723)  & (0.0210)     & (0.0164)   \\[1ex]
\quad \textit{60+}              &       0.0193      & -0.0129     & 0.0232       & -0.0798**  & 0.0437***  & -0.0348***  & 0.0538***   & -0.0417**  & 0.0233***  & -0.00762   & -0.0108      & 0.0206     \\
                                &       (0.0175)    & (0.0160)    & (0.0317)     & (0.0325)   & (0.0147)   & (0.0128)    & (0.0184)    & (0.0176)   & (0.00895)  & (0.00815)  & (0.0224)     & (0.0186)   \\[1ex]
\% of women empl.               &       -0.0294***  & 0.0323***   & -0.0203***   & 0.0287***  & -0.0176*** & 0.0216***   & -0.00670    & 0.0163**   & -0.0215*** & 0.0171***  & -0.0255***   & 0.0196***  \\
                                &       (0.00340)   & (0.00323)   & (0.00576)    & (0.00646)  & (0.00280)  & (0.00247)   & (0.00815)   & (0.00750)  & (0.00520)  & (0.00427)  & (0.00333)    & (0.00321)  \\[1ex]
Mean experience empl.           &       -0.00125*** & 0.000907*** & -0.000873*** & 0.00158*** & 0.00188*** & -0.00156*** & -0.00276*** & 0.00192*** & -3.41e-05  & 1.95e-05   & -0.000512*** & 0.000249*  \\
                                &       (0.000228)  & (0.000190)  & (0.000266)   & (0.000301) & (0.000215) & (0.000176)  & (0.000548)  & (0.000481) & (0.000359) & (0.000287) & (0.000158)   & (0.000144) \\[1ex]
\% empl. with tert. educ.       &       0.0587***   & -0.0577***  & 0.0597***    & -0.0251    & 0.0844***  & -0.0804***  & 0.130***    & -0.119***  & 0.0494***  & -0.0531*** & 0.0369***    & -0.0430*** \\
                                &       (0.00431)   & (0.00406)   & (0.0134)     & (0.0169)   & (0.00319)  & (0.00306)   & (0.0202)    & (0.0193)   & (0.00939)  & (0.00736)  & (0.00473)    & (0.00388)  \\[1ex]
\% empl. with sec. educ.        &       0.00871***  & -0.00971*** & 0.0322***    & -0.0210*   & 0.0384***  & -0.0346***  & -0.00260    & -0.0232    & 0.0264***  & -0.0328*** & -0.000259    & -0.00690*  \\
                                &       (0.00272)   & (0.00263)   & (0.00874)    & (0.0114)   & (0.00267)  & (0.00239)   & (0.0150)    & (0.0145)   & (0.00822)  & (0.00699)  & (0.00400)    & (0.00370)  \\[1ex]
\% managers and profess.        &       0.0438***   & -0.0493***  & 0.0319***    & -0.0369*** & 0.0382***  & -0.0364***  & 0.0571***   & -0.0637*** & 0.125***   & -0.126***  & 0.0820***    & -0.0660*** \\
                                &       (0.00600)   & (0.00515)   & (0.00957)    & (0.0108)   & (0.00466)  & (0.00488)   & (0.0166)    & (0.0131)   & (0.00592)  & (0.00558)  & (0.00484)    & (0.00412)  \\[1ex]
\% part-time empl.              &       -0.00418    & 0.00662*    & 0.0570***    & -0.0829*** & 0.0551***  & -0.0535***  & 0.0777***   & -0.0969*** & 0.0258***  & -0.0161*** & 0.00355      & -0.00317   \\
                                &       (0.00418)   & (0.00375)   & (0.00675)    & (0.00791)  & (0.00342)  & (0.00332)   & (0.0301)    & (0.0255)   & (0.00612)  & (0.00545)  & (0.00412)    & (0.00395)  \\[1ex]
\% permanent contracts          &       -0.0345***  & 0.0436***   & -0.0242***   & 0.0619***  & 0.000483   & 0.00465*    & -0.00203    & 0.00554    & -0.0277*** & 0.0188*    & -0.0778***   & 0.0826***  \\
                                &       (0.00567)   & (0.00617)   & (0.00906)    & (0.0107)   & (0.00255)  & (0.00238)   & (0.00795)   & (0.00835)  & (0.00961)  & (0.00979)  & (0.00750)    & (0.00651)  \\[1ex]

Firm size:                      \\[1ex]
\quad \textit{50--249 empl.}    &       0.000439    & -0.000491   & 0.0217***    & -0.0144*** & 0.0657***  & -0.0554***  & 0.00895     & -0.00736   & -0.0330*** & 0.0301***  & 0.0201***    & -0.0177*** \\
                                &       (0.00275)   & (0.00287)   & (0.00268)    & (0.00314)  & (0.00235)  & (0.00182)   & (0.00578)   & (0.00566)  & (0.00776)  & (0.00724)  & (0.00258)    & (0.00217)  \\[1ex]
\quad \textit{$\geq$ 250 empl.} &       -0.00293    & 0.000297    & 0.0239***    & -0.0101*** & 0.106***   & -0.0876***  & 0.00620     & -0.00498   & -0.0332*** & 0.0320***  & 0.0226***    & -0.0264*** \\
                                &       (0.00469)   & (0.00472)   & (0.00275)    & (0.00313)  & (0.00397)  & (0.00347)   & (0.00880)   & (0.00795)  & (0.00663)  & (0.00592)  & (0.00256)    & (0.00230)  \\[1ex]
Public firm                     &       -0.0219***  & 0.0283***   & -0.0129*     & -0.00926   & 0.0183***  & -0.0105***  & -0.0448***  & 0.0349***  & 0.00734    & -0.0121    & -0.0402***   & 0.0292***  \\
                                &       (0.00592)   & (0.00583)   & (0.00663)    & (0.00830)  & (0.00436)  & (0.00379)   & (0.00719)   & (0.00558)  & (0.0352)   & (0.0268)   & (0.00458)    & (0.00407)  \\[1ex]
\midrule
Observations                    &       13,765      & 13,765      & 12,312       & 12,312     & 37,887     & 37,887      & 3,498       & 3,498      & 14,502     & 14,502     & 30,009       & 30,009     \\
R-squared                       &       0.138       & 0.199       & 0.059        & 0.059      & 0.174      & 0.191       & 0.226       & 0.191      & 0.110      & 0.124      & 0.105        & 0.115      \\
Region FE                       &       \checkmark  & \checkmark  & \checkmark   & \checkmark & \checkmark & \checkmark  & \checkmark  & \checkmark & \checkmark & \checkmark & \checkmark   & \checkmark \\
Sector FE                       &       \checkmark  & \checkmark  & \checkmark   & \checkmark & \checkmark & \checkmark  & \checkmark  & \checkmark & \checkmark & \checkmark & \checkmark   & \checkmark \\
\bottomrule
\end{tabular}
%
\begin{tablenotes}
\item Notes: Bootsrapped standard errors in parentheses (200 repetitions); asterisks denote significance levels: $^{*}$ p$<$0.05, $^{**}$ p$<$0.01, $^{***}$ p$<$0.001
\end{tablenotes}
%
\setlength{\tabcolsep}{6pt}
\end{threeparttable}
}

% \end{table}
% \end{landscape}

Results on the control variables are rather consistent across
countries, although with some variation in the significance levels.
Modal age tends to positively correlate with the
90\textsuperscript{th} percentile and negatively with the
10\textsuperscript{th} percentile.  A higher proportion of women in
the workforce associates with lower wages at the
90\textsuperscript{th} percentile, but with higher wages at the
10\textsuperscript{th} percentile. A similar pattern is also detected
for mean in-job tenure (with the exception of Spain) and for the share
of workers covered by permanent contracts. Conversely, the share of
educated workforce associates with higher wages at the
90\textsuperscript{th} percentile and lower wages at the
10\textsuperscript{th} percentile, and exactly the same pattern is
detected for the proportion of apical professions in the firm and for
the share of part-time workers (not in Belgium). Next, concerning firm
characteristics, we find that larger firms show higher wages at the
90\textsuperscript{th} percentile, but lower wages at the
10\textsuperscript{th} percentile (not in the UK), whereas
publicly-owned firms tend to pay more than private firms their workers
at the 10\textsuperscript{th} percentile and to pay less their workers
at the 90\textsuperscript{th} percentile.





\subsection{Firm-level bargaining and inequalities between managers and low-layers workers}

% \begin{table}[htb]
% \caption{Within-firm wage-gap between managers and low-layers workers}
% \label{tab:disp_man_ls}
% \centering
% \centering
\tiny
\begin{threeparttable}
%\setlength{\tabcolsep}{0pt}
%
\begin{tabular}{l*{6}{S}}
\toprule
%                         &  (1)       & (2)        & (3)        & (4)        & (5)       & (6)         \\
                          & \multicolumn{1}{c}{BE} & \multicolumn{1}{c}{DE} & \multicolumn{1}{c}{ES} %
                          & \multicolumn{1}{c}{CZ} & \multicolumn{1}{c}{UK} & \multicolumn{1}{c}{FR} \\
 \midrule
FLB                       &  0.00463   & -0.0607*** & -0.00730   & 0.0156     & 0.0307    & -0.0862***  \\
                          &  (0.0227)  & (0.0231)   & (0.0224)   & (0.0221)   & (0.0315)  & (0.0248)    \\[1ex]
Year 2010                 &  0.00909   & 0.0167     & 0.0415**   & -0.106***  & -0.106**  & -0.0116     \\
                          &  (0.0163)  & (0.0157)   & (0.0199)   & (0.0296)   & (0.0537)  & (0.0119)    \\[1ex]
FLB$\times$2010           &  -0.0251   & 0.0555**   & -0.0839*** & 0.0352     & -0.0258   & 0.0190      \\
                          &  (0.0295)  & (0.0278)   & (0.0323)   & (0.0318)   & (0.0554)  & (0.0302)    \\[1ex]
Prob. FLB                 &  -0.311*   & -0.756***  & -0.710***  & 0.367***   & 0.493     & 0.145       \\
                          &  (0.161)   & (0.168)    & (0.182)    & (0.104)    & (0.676)   & (0.137)     \\[1ex]

Modal age workers:        \\[1ex]
\quad \textit{20-29}      &  0.0630    & 0.149      & 0.144***   &            & -0.218*   & 0.176       \\
                          &  (0.0696)  & (0.142)    & (0.0441)   &            & (0.121)   & (0.169)     \\[1ex]
\quad \textit{30-39}      &  0.0957    & 0.243*     & 0.143***   & 0.0335     & -0.227*   & 0.201       \\
                          &  (0.0690)  & (0.141)    & (0.0342)   & (0.0273)   & (0.124)   & (0.168)     \\[1ex]
\quad \textit{40-49}      &  0.118*    & 0.237*     & 0.129***   & -0.0599**  & -0.180    & 0.231       \\
                          &  (0.0687)  & (0.140)    & (0.0342)   & (0.0287)   & (0.123)   & (0.168)     \\[1ex]
\quad \textit{50-59}      &  0.130*    & 0.220      & 0.145***   & -0.0348    & -0.164    & 0.270       \\
                          &  (0.0698)  & (0.139)    & (0.0374)   & (0.0279)   & (0.121)   & (0.168)     \\[1ex]
\quad \textit{60+}        &  0.118     & 0.279*     & 0.241***   & 0.167***   & -0.243*   & 0.283*      \\
                          &  (0.156)   & (0.156)    & (0.0917)   & (0.0600)   & (0.136)   & (0.172)     \\[1ex]
\% of women empl.         &  0.0555*   & 0.130***   & 0.197***   & 0.00441    & 0.0411    & -0.0418     \\
                          &  (0.0305)  & (0.0380)   & (0.0389)   & (0.0371)   & (0.0539)  & (0.0255)    \\[1ex]
Mean experience empl.     &  0.00489** & -0.00157   & 0.00386    & -0.00457** & -0.00435  & -0.00292*** \\
                          &  (0.00199) & (0.00164)  & (0.00266)  & (0.00226)  & (0.00302) & (0.00106)   \\[1ex]
\% empl. with tert. educ. &  0.110**   & 0.0636     & 0.233***   & 0.358***   & 0.147     & -0.190***   \\
                          &  (0.0473)  & (0.0759)   & (0.0505)   & (0.0820)   & (0.100)   & (0.0290)    \\[1ex]
\% empl. with sec. educ.  &  0.0672**  & 0.203***   & 0.0760*    & -0.0734    & -0.00143  & -0.0784***  \\
                          &  (0.0305)  & (0.0607)   & (0.0408)   & (0.0531)   & (0.0930)  & (0.0288)    \\[1ex]
\% managers and profess.  &  0.0408    & -0.157***  & -0.215**   & -0.399***  & 0.139*    & -0.0199     \\
                          &  (0.0565)  & (0.0533)   & (0.0859)   & (0.0600)   & (0.0751)  & (0.0353)    \\[1ex]
\% part-time empl.        &  -0.0238   & -0.145***  & -0.162***  & -0.337***  & -0.0800   & 0.0610      \\
                          &  (0.0493)  & (0.0376)   & (0.0612)   & (0.0917)   & (0.0775)  & (0.0382)    \\[1ex]
\% permanent contracts    &  0.196***  & 0.0388     & 0.212***   & 0.0339     & 0.0994    & 0.0180      \\
                          &  (0.0522)  & (0.0560)   & (0.0460)   & (0.0392)   & (0.158)   & (0.0532)    \\[1ex]

Firm size:                \\[1ex]
\quad \textit{50--249 empl.}    &  0.107***  & 0.0991***  & 0.179***   & 0.0602**   & 0.0570    & -0.0878***  \\
                          &  (0.0302)  & (0.0241)   & (0.0298)   & (0.0248)   & (0.0580)  & (0.0159)    \\[1ex]
\quad \textit{$\geq$ 250 empl.} &  0.146***  & 0.0469**   & 0.215***   & 0.0447     & -0.0330   & -0.177***   \\
                          &  (0.0473)  & (0.0230)   & (0.0565)   & (0.0337)   & (0.0579)  & (0.0156)    \\[1ex]
Public firm               &  -0.0764   & -0.268***  & 0.0199     & -0.131***  & 0.184     & -0.119***   \\
                          &  (0.0487)  & (0.0458)   & (0.0504)   & (0.0254)   & (0.325)   & (0.0383)    \\[1ex]
Constant                  &  -0.322**  & -0.375**   & -0.352***  & 0.0358     & -0.564    & 0.134       \\
                          &  (0.154)   & (0.176)    & (0.0921)   & (0.102)    & (0.625)   & (0.181)     \\[1ex]
\midrule
Observations              &  2,416     & 4,396      & 3,443      & 3,006      & 2,059     & 6,895       \\
R-squared                 &  0.087     & 0.083      & 0.091      & 0.158      & 0.078     & 0.069       \\
Region FE               &  \checkmark& \checkmark & \checkmark & \checkmark & \checkmark& \checkmark  \\
Sector FE                 &  \checkmark& \checkmark & \checkmark & \checkmark & \checkmark& \checkmark  \\
\bottomrule%
\end{tabular}
%
\begin{tablenotes}
\item Notes: Bootsrapped standard errors in parentheses (200 repetitions); asterisks denote significance levels: $^{*}$ p$<$0.05, $^{**}$ p$<$0.01, $^{***}$ p$<$0.001
\end{tablenotes}
%
\setlength{\tabcolsep}{6pt}
%
\end{threeparttable}

% \end{table}

We then present the findings concerning whether firm-level bargaining
affects wage inequality across occupations.

Table~\ref{tab:disp_man_ls} reports the estimates of
Equation~\ref{eq:reg_dispersion}, taking the within-firm professional
wage-gap $\Delta w^{jobs}$ as the dependent variable. In close
similarity with the analysis of the the
90\textsuperscript{th}-to-10\textsuperscript{th} percentiles wage-gap,
our general finding is that the effects of bargaining at firm-level on
top of more centralised levels are widely heterogeneous, both across
countries and over time.  However, the estimated effects do not
exactly replicate the patterns emerged above for $\Delta w^{90/10}$.
This supports that accounting for the professional content of wage
inequalities does convey relevant additional information.

In three countries, namely Belgium, the Czech Republic and the UK,
firm-level agreements do not display any relations with the
professional wage-gap, neither in 2006 nor in 2010, whereas in
Germany, Spain and France we observe statistically significant
effects, varying by country and over time variation. In Germany, the
$\Delta w^{jobs}$ pay-gap is less unequal in firms bargaining locally
than in other firms in 2006, but we observe a reversal in the effects
of FLB over time, such that the inequality-reducing effect of
firm-level contracts vanishes by 2010. Indeed, the estimated
interaction coefficient is of similar magnitude, but of opposite sign
compared to the coefficient on the FLB dummy. In France and Spain,
firm-level agreements show a somewhat opposite effect, more favourable
to compressing the professional wage-gap. In France, FLB firms feature
a lower $\Delta w^{jobs}$ than other firms in both 2006 and 2010. In
Spain, firms which bargain at firm-level and other firms do not differ
significantly in their professional wage-gaps in 2006, while
firm-level bargaining firms become less unequal than the other firms
in 2010.

Concerning the estimates on the set of control variables, a higher
share of women (when significant, i.e., Belgium, Germany and Spain)
associates with a higher professional wage-gap across professional
groups, while average tenure shows cross-country heterogeneity: it is
negatively related to the professional wage-gap in France and the
Czech Republic, while we observe a positive correlation in Belgium,
and insignificant estimates are obtained for the other countries.  In
general, a higher proportion of educated workers associates with a
higher professional wage-gap in most countries (not in France). A
larger share of part-time contracts negatively relates with the
professional wage-gap in Germany, Spain and the Czech Republic. The
opposite relation holds in the case of the share of workers with
permanent contracts, at least in Germany and Spain. Among firm-level
characteristics, firm size seems to play a consistent role, as larger
firms experience greater professional wage-gaps in all countries but
France. Public control associate with reduced wage differences across
occupations as compared to private firms, in most countries (not in
Spain and the United Kingdom).\\


% \begin{landscape}
% \centering
% \begin{table}[htb]
% \caption{Decomposition of the effects on the wages of managers and low-layers workers}
% \label{tab:disp_avg_man_ls}
% \centering
\tiny
\begin{threeparttable}
% \setlength{\tabcolsep}{0pt}
%
\begin{tabular}{l*{12}{S}}
\toprule
%                         & (1)       & (2)         & (3)        & (4)        & (5)       & (6)        & (7)        & (8)        & (9)       & (10)      & (11)        & (12)       \\

& \multicolumn{2}{c}{BE} & \multicolumn{2}{c}{DE} & \multicolumn{2}{c}{ES} & \multicolumn{2}{c}{CZ} & \multicolumn{2}{c}{UK} & \multicolumn{2}{c}{FR} \\

& \multicolumn{1}{c}{Manag} & \multicolumn{1}{c}{Low} & \multicolumn{1}{c}{Manag} & \multicolumn{1}{c}{Low} & \multicolumn{1}{c}{Manag} & \multicolumn{1}{c}{Low} 
                          & \multicolumn{1}{c}{Manag} & \multicolumn{1}{c}{Low} & \multicolumn{1}{c}{Manag} & \multicolumn{1}{c}{Low} & \multicolumn{1}{c}{Manag} & \multicolumn{1}{c}{Low} \\

\midrule
FLB                       &  0.00605   & 0.00142     & -0.0466**  & 0.0141     & -0.0128   & -0.00548   & 0.0195     & 0.00385    & 0.0296    & -0.00107  & -0.0412**   & 0.0451***  \\
                          &  (0.0177)  & (0.00735)   & (0.0207)   & (0.0117)   & (0.0179)  & (0.00888)  & (0.0176)   & (0.00671)  & (0.0224)  & (0.0150)  & (0.0176)    & (0.0148)   \\[1ex]
Year 2010                 &  9.62e-05  & -0.00899    & 0.00862    & -0.00812   & 0.0229    & -0.0186**  & -0.0790*** & 0.0268**   & -0.0702*  & 0.0353    & -0.00862    & 0.00299    \\
                          &  (0.0125)  & (0.00581)   & (0.0143)   & (0.00686)  & (0.0158)  & (0.00749)  & (0.0221)   & (0.0105)   & (0.0369)  & (0.0267)  & (0.00838)   & (0.00565)  \\[1ex]
FLB$\times$2010           &  -0.0198   & 0.00531     & 0.0390*     & -0.0165    & -0.0485*  & 0.0354***  & 0.0117     & -0.0236  & -0.0413   & -0.0155   & 0.00421     & -0.0148    \\
                          &  (0.0261)  & (0.00856)   & (0.0249)   & (0.0157)   & (0.0276)  & (0.0111)   & (0.0245)   & (0.0105)   & (0.0390)  & (0.0256)  & (0.0214)    & (0.0190)   \\[1ex]
Prob. FLB    &  0.0164    & 0.327***    & -0.540***  & 0.216***   & -0.498*** & 0.212***   & 0.284***   & -0.0827**  & 0.564     & 0.0711    & 0.0362      & -0.109*    \\
                          &  (0.139)   & (0.0717)    & (0.130)    & (0.0743)   & (0.153)   & (0.0772)   & (0.0939)   & (0.0402)   & (0.455)   & (0.308)   & (0.104)     & (0.0576)   \\[1ex]

Modal age workers:        \\[1ex]
\quad \textit{20-29}      &  0.0241    & -0.0389     & 0.134      & -0.0147    & 0.139***  & -0.00526   &            &            & -0.211**  & 0.00735   & 0.109       & -0.0674    \\
                          &  (0.0999)  & (0.0238)    & (0.145)    & (0.0569)   & (0.0328)  & (0.0144)   &            &            & (0.0969)  & (0.0544)  & (0.100)     & (0.0727)   \\[1ex]
\quad \textit{30-39}      &  0.0402    & -0.0555**   & 0.200      & -0.0427    & 0.144***  & 0.00146    & 0.0197     & -0.0137    & -0.211**  & 0.0165    & 0.130       & -0.0713    \\
                          &  (0.0998)  & (0.0238)    & (0.144)    & (0.0611)   & (0.0273)  & (0.0124)   & (0.0245)   & (0.0106)   & (0.0976)  & (0.0556)  & (0.0988)    & (0.0722)   \\[1ex]
\quad \textit{40-49}      &  0.0583    & -0.0593**   & 0.194      & -0.0424    & 0.129***  & 0.000315   & -0.0568**  & 0.00309    & -0.178*   & 0.00148   & 0.143       & -0.0876    \\
                          &  (0.0984)  & (0.0234)    & (0.145)    & (0.0590)   & (0.0273)  & (0.0109)   & (0.0271)   & (0.0107)   & (0.0985)  & (0.0543)  & (0.0993)    & (0.0719)   \\[1ex]
\quad \textit{50-59}      &  0.0733    & -0.0566**   & 0.179      & -0.0406    & 0.134***  & -0.0104    & -0.0332    & 0.00163    & -0.195**  & -0.0313   & 0.167*      & -0.103     \\
                          &  (0.100)   & (0.0246)    & (0.146)    & (0.0590)   & (0.0285)  & (0.0113)   & (0.0253)   & (0.0104)   & (0.0988)  & (0.0539)  & (0.0987)    & (0.0723)   \\[1ex]
\quad \textit{60+}        &  0.0541    & -0.0636     & 0.209      & -0.0704    & 0.211***  & -0.0297    & 0.0964*    & -0.0706**  & -0.232**  & 0.0105    & 0.178*      & -0.105     \\
                          &  (0.133)   & (0.0582)    & (0.157)    & (0.0701)   & (0.0673)  & (0.0324)   & (0.0512)   & (0.0293)   & (0.107)   & (0.0573)  & (0.102)     & (0.0779)   \\[1ex]
\% of women empl.         &  0.0427    & -0.0128     & 0.0672**   & -0.0626*** & 0.133***  & -0.0634*** & 0.0469     & 0.0425***  & -0.000410 & -0.0415   & -0.0389**   & 0.00291    \\
                          &  (0.0319)  & (0.0140)    & (0.0312)   & (0.0195)   & (0.0331)  & (0.0149)   & (0.0289)   & (0.0123)   & (0.0418)  & (0.0268)  & (0.0155)    & (0.0109)   \\[1ex]
Mean experience empl.     &  0.00110   & -0.00379*** & -0.000924  & 0.000641   & 0.00212   & -0.00173*  & -0.00354*  & 0.00103    & -0.00236  & 0.00199   & -0.00254*** & 0.000382   \\
                          &  (0.00187) & (0.000921)  & (0.00150)  & (0.000740) & (0.00218) & (0.000979) & (0.00206)  & (0.000815) & (0.00218) & (0.00146) & (0.000755)  & (0.000562) \\[1ex]
\% empl. with tert. educ. &  0.0552    & -0.0550***  & -0.0876    & -0.151***  & 0.158***  & -0.0743*** & 0.261***   & -0.0971*** & 0.0715    & -0.0757*  & -0.102***   & 0.0886***  \\
                          &  (0.0344)  & (0.0197)    & (0.0683)   & (0.0385)   & (0.0364)  & (0.0181)   & (0.0714)   & (0.0325)   & (0.0726)  & (0.0449)  & (0.0218)    & (0.0153)   \\[1ex]
\% empl. with sec. educ.  &  0.0471*   & -0.0202*    & 0.0750     & -0.128***  & 0.0543    & -0.0218    & -0.0544    & 0.0190     & -0.0254   & -0.0239   & -0.0588***  & 0.0196*    \\
                          &  (0.0254)  & (0.0108)    & (0.0590)   & (0.0212)   & (0.0350)  & (0.0138)   & (0.0540)   & (0.0178)   & (0.0630)  & (0.0423)  & (0.0196)    & (0.0112)   \\[1ex]
\% managers and profess.  &  0.0930**  & 0.0522*     & -0.0328    & 0.124***   & -0.123*   & 0.0927**   & -0.205***  & 0.194***   & 0.115**   & -0.0236   & 0.000121    & 0.0201     \\
                          &  (0.0470)  & (0.0293)    & (0.0433)   & (0.0376)   & (0.0630)  & (0.0366)   & (0.0475)   & (0.0321)   & (0.0457)  & (0.0421)  & (0.0236)    & (0.0225)   \\[1ex]
\% part-time empl.        &  -0.0382   & -0.0144     & -0.0842*** & 0.0609***  & -0.100**  & 0.0610***  & -0.260***  & 0.0766***  & -0.0671   & 0.0128    & 0.0511**    & -0.00988   \\
                          &  (0.0461)  & (0.0183)    & (0.0294)   & (0.0206)   & (0.0505)  & (0.0191)   & (0.0898)   & (0.0290)   & (0.0528)  & (0.0336)  & (0.0236)    & (0.0157)   \\[1ex]
\% permanent contracts    &  0.0518    & -0.144***   & 0.0115     & -0.0273    & 0.164***  & -0.0479*** & 0.0377     & 0.00382    & 0.131     & 0.0311    & 0.0670**    & 0.0490*    \\
                          &  (0.0567)  & (0.0231)    & (0.0462)   & (0.0295)   & (0.0402)  & (0.0154)   & (0.0344)   & (0.0124)   & (0.114)   & (0.0577)  & (0.0338)    & (0.0294)   \\[1ex]

Firm size:                \\[1ex]
\quad \textit{50--249 empl.}    &  0.0613*** & -0.0455***  & 0.109***   & 0.00998    & 0.157***  & -0.0221**  & 0.0643***  & 0.00412    & 0.0667*   & 0.00975   & -0.0532***  & 0.0346***  \\
                          &  (0.0238)  & (0.0122)    & (0.0178)   & (0.00996)  & (0.0250)  & (0.0109)   & (0.0213)   & (0.0106)   & (0.0398)  & (0.0311)  & (0.0115)    & (0.00792)  \\[1ex]
\quad \textit{$\geq$ 250 empl.}&  0.0662    & -0.0801***  & 0.0595***  & 0.0126     & 0.183***  & -0.0324    & 0.0640**   & 0.0193     & 0.0349    & 0.0679*** & -0.104***   & 0.0730***  \\
                          &  (0.0409)  & (0.0216)    & (0.0199)   & (0.0109)   & (0.0451)  & (0.0221)   & (0.0308)   & (0.0149)   & (0.0379)  & (0.0262)  & (0.0108)    & (0.00859)  \\[1ex]
Public firm               &  -0.0209   & 0.0555**    & -0.182***  & 0.0860***  & 0.00974   & -0.0101    & -0.127***  & 0.00489    & 0.247     & 0.0632    & -0.0541*    & 0.0653***  \\
                          &  (0.0444)  & (0.0266)    & (0.0366)   & (0.0224)   & (0.0411)  & (0.0204)   & (0.0209)   & (0.0118)   & (0.220)   & (0.149)   & (0.0300)    & (0.0168)   \\[1ex]
Constant                  &  -0.163    & 0.159***    & -0.212     & 0.163**    & -0.295*** & 0.0568     & 0.0237     & -0.0121    & -0.556    & 0.00833   & 0.0460      & -0.0877    \\
                          &  (0.134)   & (0.0472)    & (0.180)    & (0.0654)   & (0.0748)  & (0.0351)   & (0.0857)   & (0.0274)   & (0.421)   & (0.286)   & (0.111)     & (0.0828)   \\
\midrule
Observations              &  2,416     & 2,416       & 4,396      & 4,396      & 3,443     & 3,443      & 3,006      & 3,006      & 2,059     & 2,059     & 6,895       & 6,895      \\
R-squared                 &  0.046     & 0.347       & 0.063      & 0.054      & 0.079     & 0.061      & 0.169      & 0.098      & 0.054     & 0.113     & 0.052       & 0.053      \\
Region FE               &  \checkmark& \checkmark  & \checkmark & \checkmark & \checkmark& \checkmark & \checkmark & \checkmark & \checkmark& \checkmark& \checkmark  & \checkmark \\
Sector FE                 &  \checkmark& \checkmark  & \checkmark & \checkmark & \checkmark& \checkmark & \checkmark & \checkmark & \checkmark& \checkmark& \checkmark  & \checkmark \\
\bottomrule
\end{tabular}
 
\begin{tablenotes}
\item Notes: Bootsrapped standard errors in parentheses (200 repetitions); asterisks denote significance levels: $^{*}$ p$<$0.05, $^{**}$ p$<$0.01, $^{***}$ p$<$0.001
\end{tablenotes}
%
\setlength{\tabcolsep}{6pt}
%
\end{threeparttable}

% \end{table}
% \end{landscape}

As the last step our analysis, we examine the relations between
firm-level bargaining and the two components of the professional
wage-gap. Table~\ref{tab:disp_avg_man_ls} presents the estimates of
Equation~\ref{eq:reg_decomposition} taking the average wage of either
the managers or the low-layers employees as the dependent variable. A
first notable finding regards Belgium, the Czech Republic and the UK,
namely the three countries for which firm-level bargaining did not
show any statistical association with the $\Delta w^{jobs}$ wage-gap
in Table~\ref{tab:disp_man_ls} above. Dissecting by wages of different
professional groups confirms the same picture: firms which bargain
locally do not show significant differences compared to other firms,
neither in terms of average wages of low-layers employees, nor insofar
as wages of managers are concerned.

Second, as for Germany, the over time reversal in the effect of FLB on
the professional wage-gap emerged in Table~\ref{tab:disp_man_ls} seems
to be driven by a change in the FLB practices towards managers.
Indeed, FLB firms and other firms do not show differences in the
average wages of their low-layers employees. Conversely, managers are
paid on average less in FLB firms than in other firms in 2006, but
they see an increase in their wages in 2010 in firms adopting FLB.

Finally, we find that a common underlying dynamics characterizes the overall equality-enhancing effect of FLB on $\Delta w^{jobs}$ emerged in Table~\ref{tab:disp_man_ls} above for France and Spain. In both countries, indeed, FLB firms are more equal than other firms due to both lower average wages paid to managers and higher average wages paid to low-layers workers. There is a different timing in the two countries, however. In France, this differential treatment of managers and low-layer employees across FLB and other firms is already in place in 2006, and remains unchanged in 2010. In Spain, it is only in 2010 that the average wages of the two professional categories become statistically different across FLB and other firms.

Results on controls display, once again, heterogeneity across
countries. The modal age of the workforce displays a significant
association with wages of managers in Spain (positive) and in the UK
(negative), while a relatively strong and negative association emerges
with the average wage of low-layers employees in Belgium. The share of
women in the workforce features a positive relation with managers'
wages in Spain and Germany, but the relation is negative in France.
Also, wages of low-layers employees are higher in firms with more
women in the Czech Republic, while they decrease with the number of
women in Germany and Spain. Average tenure does not display strong
associations in most countries, whereas education does, and the share
of employees with tertiary education, in particular: in all countries
(but France), firms with relatively more educated workforce pay
relatively higher wages to managers and relatively lower wages to
low-layers employees. The share of managers/professionals and the
contract types do not show systematic patterns. Among enterprise
characteristics, we observe that larger firms pay managers more than
other firms in most countries (not in France). The opposite holds for
public firms as compared to private firms.

% France is a special case where most of the previous features follow
% a \textit{sui generis} trend, reverting the patterns detected for
% other countries in the analysis.



\section{Conclusions}
\label{sec:conclusion}

The impact of collective pay agreements on inter-firm wage inequality
is well-documented. However, there is less evidence on whether
wage-setting happening at the level of firms -- on top of more
centralized bargaining levels -- can explain wage differences emerging
within the firm. A-priori, effects are uncertain. By allowing to firms
more flexibility and more discretionality than higher level
negotiations, firm-level agreements may induce an increase in
within-firm inequality, if they are used to selectively provide
incentives or rewards to specific employees or groups of employees.
Conversely, firm-level agreements may reduce inequalities within firms
if they respond to fairness, egalitarian or redistributive motives.
The balancing between contrasting outcomes may, in turn, depends from
the institutional context, according to the variation of the
discretionality margins left to firm-level negotiations in different
bargaining systems in place across countries and over time.\\

Exploiting matched employer-employee data for six European nations
over 2006 and 2010, in this work we contribute to advance the existing
literature by addressing three questions. First, does firm-level
bargaining increase or decrease within-firm inequality and, if any,
are the emerging patterns robust across measures of inequality that
differently address possible conflicts of power across different
groups (high vs. low paid, and managers vs. low-layer) of employees?
Second, have these relations remained stable or changed over the years
under study, when a broad process of increasing emphasis on
decentralization of wage bargaining took place and the Great
Depression hit? Third, in case firm-level bargaining emerges as
significantly shaping -- either statically or over time -- the
internal wage structures of firms, are there patterns common to all or
at least to some of the selected countries, mapping into broad
bargaining regimes or models of capitalism?\\

Our empirical results, in summary, reveal ample heterogeneities by
wage-inequality measure, by country and over time.

First, by country, we find no effect of the use of firm-level
bargaining on within-firm inequality in Belgium and the Czech
Republic, while statistically significant results emerge for Germany,
Spain, France and the UK. What is noteworthy is that Belgium and the
Czech Republic are examples of opposite models, one highly centralised
and featuring coordinated industrial relations, and the other
characterised by markedly decentralised and market-oriented
institutional set-up. Also, quite diverse bargaining and institutional
models coexist in the group of countries where we estimate a
significant effect of firm-level negotiations. This observation
suggest that country-specific heterogeneities deliver stronger
explanatory power than taxonomies predicting consistent patterns
across countries a-priori classified as sharing homogeneous bargaining
systems or varieties of capitalism.

Second, if we focus on those country-cases where we estimate a
significant effect of firm-level bargaining, we do not find evidence
of a single, precise direction in the effect. To some extent, one
could have put forward explanations justifying the emergence of
consistent results across France, Germany, Spain and the UK, in spite
of the observed wide country-level specificities featuring these
countries. For instance, arguing that firm-level bargaining is equally
expected to increase within-firm inequalities in the UK --where
workplace bargaining is historically designed to incentivize specific
groups of workers-- as well as in France, Germany, and Spain, because
firm-level negotiations there primarily pursue to introduce
discretionality and flexibility against the standardization of wages
and the complexity of negotiations usually featuring the more
centralised bargaining levels in these countries. Instead, we find
that firm-level bargaining can either enhance or reduce within-firm
pay inequality, and the effects significantly vary even within the
same country, over time and also depending on the measure of wage-gap.
In the UK, firms bargaining locally are more unequal than other firms
in terms of the wage distance between high~vs.~low paid employees, but
not in terms of the professional wage-gap between managers and
low-layers workers. In Spain and France, firms bargaining locally
become over time more unequal than other firms in terms of the
high-paid~vs.~low-paid employees wage-gap, while the professional
wage-gap shows an opposite inequality-reducing effect of firm-level
bargaining over time. This pattern in the occupational wage-gaps
replicates also for Germany, whereas in this country firms bargaining
locally are not more unequal in terms of the wage distance between
their high and low paid employees.

Whatever the inner underlying mechanisms, we show that bargaining at
the firm-level (on top of more centralised levels) allows for some
discretionality in the power relations within firms. As the variation
of results by inequality measure suggests, corroborated by
decomposition analysis of the effects by groups of employees,
firm-level bargaining enhances or reduces within-firm pay inequality
according to how relative power is distributed among groups of
workers. The final outcome heavily reflects the management and the
resolution of potential conflicts of power within organizations. These
processes do not seem related to a peculiar prevailing regime,
however.

Overall, our study offers new evidence and methods to inform the
renewed debate on the determinants of increasing inequalities. We
highlight the importance of the locus of collective wage bargaining as
a central driver of wage inequality within firms, as it is deeply
linked to how the discretionality allowed for by firm-level
negotiations interacts with stakeholders' power and executive decision
making.




% End notes
\clearpage
\theendnotes


\clearpage
\singlespacing
\bibliography{biblio_CBWD}

\clearpage

\appendix

\section*{Tables}

\begin{table}[hbp]
\centering
\caption{Total number of firms and employees by country, year, and type of collective agreement in the sample.}
\label{tab:summary_stats}
\resizebox*{0.9\linewidth}{!}{%
\begin{tabular}{lrrrrrrrrrrrr|rr} %
\toprule
\textbf{Bargaining} & \multicolumn{4}{c}{Centralized} & \multicolumn{4}{c}{Firm-level} & \multicolumn{4}{c}{None}  & \multicolumn{2}{c}{\textbf{Total}}    \\
            & N firm & \% firm & N empl  & \% empl & N firm & \% firm & N empl & \% empl & N firm & \% firm & N empl  & \% empl & N firms & N empl \\
                             \cmidrule(lr){2-5}                \cmidrule(lr){6-9}               \cmidrule(lr){10-13} \cmidrule(lr){14-15} %\
BE          &        &         &         &         &        &         &        &         &        &         &         &         &               &              \\
\qquad 2006 & 7341   & 82,0\%  & 131339  & 79,5\%  & 1606   & 18,0\%  & 33852  & 20,5\%  & 0      & 0,0\%   & 0       & 0,0\%   & 165191        & 8947         \\
\qquad 2010 & 5581   & 81,1\%  & 108109  & 78,8\%  & 1304   & 18,9\%  & 29145  & 21,2\%  & 0      & 0,0\%   & 0       & 0,0\%   & 137254        & 6885         \\[1ex]
CZ          &        &         &         &         &        &         &        &         &        &         &         &         &               &              \\
\qquad 2006 & 466    & 2,6\%   & 122565  & 6,2\%   & 1315   & 7,3\%   & 942397 & 47,8\%  & 16278  & 90,1\%  & 905902  & 46,0\%  & 1970864       & 18059        \\
\qquad 2010 & 517    & 2,9\%   & 91588   & 4,6\%   & 1504   & 8,3\%   & 985320 & 49,4\%  & 16025  & 88,8\%  & 916717  & 46,0\%  & 1993625       & 18046        \\[1ex]
DE          &        &         &         &         &        &         &        &         &        &         &         &         &               &              \\
\qquad 2006 & 7546   & 20,3\%  & 1688535 & 58,4\%  & 2397   & 6,5\%   & 161995 & 5,6\%   & 27189  & 73,2\%  & 1042351 & 36,0\%  & 2892881       & 37132        \\
\qquad 2010 & 8001   & 27,0\%  & 774884  & 45,5\%  & 1787   & 6,0\%   & 120223 & 7,1\%   & 19815  & 66,9\%  & 806251  & 47,4\%  & 1701358       & 29603        \\[1ex]
ES          &        &         &         &         &        &         &        &         &        &         &         &         &               &              \\
\qquad 2006 & 23896  & 87,5\%  & 189372  & 80,5\%  & 3405   & 12,5\%  & 45900  & 19,5\%  & 0      & 0,0\%   & 0       & 0,0\%   & 235272        & 27301        \\
\qquad 2010 & 19294  & 76,9\%  & 141643  & 65,3\%  & 3992   & 15,9\%  & 57641  & 26,6\%  & 1818   & 7,2\%   & 17485   & 8,1\%   & 216769        & 25104        \\[1ex]
FR          &        &         &         &         &        &         &        &         &        &         &         &         &               &              \\
\qquad 2006 & 13583  & 88,3\%  & 86640   & 76,2\%  & 983    & 6,4\%   & 19036  & 16,8\%  & 820    & 5,3\%   & 7965    & 7,0\%   & 113641        & 15386        \\
\qquad 2010 & 27333  & 89,1\%  & 180504  & 81,9\%  & 3089   & 10,1\%  & 37974  & 17,2\%  & 271    & 0,9\%   & 1891    & 0,9\%   & 220369        & 30693        \\[1ex]
UK          &        &         &         &         &        &         &        &         &        &         &         &         &               &              \\
\qquad 2006 & 9645   & 22,4\%  & 32113   & 24,1\%  & 12104  & 28,1\%  & 33509  & 25,1\%  & 21262  & 49,4\%  & 67721   & 50,8\%  & 133343        & 43011        \\
\qquad 2010 & 17838  & 17,1\%  & 43622   & 24,4\%  & 21611  & 20,7\%  & 41709  & 23,3\%  & 64778  & 62,2\%  & 93785   & 52,4\%  & 179116        & 104227       \\
\bottomrule %
\end{tabular}%
}

\end{table}

%\clearpage
%\section*{Regression results}

\begin{table}[hbp]
\centering
\caption{Within-firm wage inequalities: OLS Difference-in-means test across firms under firm-level bargaining and other firms, by country and year.}
\label{tab:prelim_ols}
\sisetup{
    parse-numbers = false,
    table-number-alignment  = center,%
    group-digits            = false,%
    table-format            = -2.4,%
    table-auto-round,%
    input-symbols           = {()},    % redefine ( ) as text symbols
    table-space-text-pre    = {$-$},   % allow for proper spacing of -(
    table-space-text-post   = {},%
    table-align-text-post   = false,   % toggle alignment of *** after estimates
}

\begin{threeparttable}
\begin{tabular}{cccSSSSr} %
\toprule %
 & & & \multicolumn{2}{c}{Firm-level bargaining} & \multicolumn{2}{c}{Constant} & Obs. \\
& Country & Year & \multicolumn{1}{r}{Coeff.} & \multicolumn{1}{c}{S.e.} %
                 & \multicolumn{1}{r}{Coeff.} & \multicolumn{1}{c}{S.e.} & \multicolumn{1}{c}{N} \\[1ex]
\cmidrule(lr){2-2} \cmidrule(lr){3-3} \cmidrule(lr){4-5} \cmidrule(lr){6-7} \cmidrule(lr){8-8} 
\multirow{13}[0]{*}{$\Delta w^{90/10}$} \ldelim\{{13}{1pt}%
& BE & 2006 & 0.0120**   & (0.00463) & 0.384***  & (0.00211) & 8639  \\
&    & 2010 & 0.00927*   & (0.00382) & 0.357***  & (0.00179) & 6633  \\[1ex]
& DE & 2006 & -0.00735   & (0.00521) & 0.495***  & (0.00238) & 7462  \\
&    & 2010 & 0.00868    & (0.00532) & 0.494***  & (0.00238) & 9753  \\[1ex]
& ES & 2006 & 0.0784***  & (0.00453) & 0.410***  & (0.00168) & 24278 \\
&    & 2010 & 0.0838***  & (0.00434) & 0.403***  & (0.00193) & 19108 \\[1ex]
& CZ & 2006 & -0.0225*   & (0.0104)  & 0.531***  & (0.00953) & 1780  \\
&    & 2010 & -0.000329  & (0.00960) & 0.521***  & (0.00866) & 2019  \\[1ex]
& UK & 2006 & -0.0196*** & (0.00551) & 0.531***  & (0.00399) & 9178  \\
&    & 2010 & -0.0730*** & (0.00680) & 0.489***  & (0.00524) & 6079  \\[1ex]
& FR & 2006 & -0.0317*** & (0.00963) & 0.472***  & (0.00270) & 10900 \\
&    & 2010 & -0.0475*** & (0.00412) & 0.451***  & (0.00214) & 19109 \\[2ex]
%
\multirow{13}[0]{*}{$\Delta w^\mathrm{jobs}$} \ldelim\lbrace{13}{1pt} %
& BE & 2006 & 0.000427   & (0.0207)  & -0.0166   & (0.0115)  & 1411  \\
&    & 2010 & -0.0179    & (0.0179)  & -0.0108   & (0.00858) & 1164  \\[1ex]
& DE & 2006 & -0.0706*** & (0.0186)  & -0.000797 & (0.00913) & 2706  \\
&    & 2010 & -0.0115    & (0.0170)  & 0.00555   & (0.00747) & 3529  \\[1ex]
& ES & 2006 & -0.0528*   & (0.0212)  & 0.0597*** & (0.0116)  & 2068  \\
&    & 2010 & -0.154***  & (0.0226)  & 0.0855*** & (0.0138)  & 1695  \\[1ex]
& CZ & 2006 & 0.0298     & (0.0213)  & 0.0792*** & (0.0192)  & 1598  \\
&    & 2010 & 0.127***   & (0.0229)  & -0.0490*  & (0.0215)  & 1689  \\[1ex]
& UK & 2006 & -0.0156    & (0.0260)  & -0.0318   & (0.0206)  & 1544  \\
&    & 2010 & -0.124***  & (0.0357)  & -0.0228   & (0.0259)  & 646   \\[1ex]
& FR & 2006 & -0.167***  & (0.0216)  & 0.0479*** & (0.00875) & 2572  \\
&    & 2010 & -0.108***  & (0.0178)  & 0.0338*** & (0.00638) & 4323  \\
\bottomrule%
\end{tabular}
%
\begin{tablenotes}
\item Notes: Robust standard errors in parenthesis; asterisks denote significance levels: $^{*}$ p$<$0.05, $^{**}$ p$<$0.01, $^{***}$ p$<$0.001
\end{tablenotes}
%
\setlength{\tabcolsep}{6pt}
%
\end{threeparttable}

\end{table}

\clearpage

\begin{table}[hbt]
\caption{FLB and 90th-10th percentile wage inequality}
\label{tab:disp_90_10}
\centering
\resizebox{\textwidth}{!}{%
\begin{threeparttable}
%\setlength{\tabcolsep}{0pt}
%
\begin{tabular}{l*{6}{S}}
\toprule
%                         &  (1)         & (2)         & (3)        & (4)         & (5)        & (6)          \\
& \multicolumn{1}{c}{BE} & \multicolumn{1}{c}{DE} & \multicolumn{1}{c}{ES} %
& \multicolumn{1}{c}{CZ} & \multicolumn{1}{c}{UK} & \multicolumn{1}{c}{FR} \\
\midrule
$\beta_0$: Intercept                         &       0.455***    & 0.529***    & 0.299***   & 0.463***    & 0.596***   & 0.624***     \\
$\hookrightarrow$ \textit{Base inequality (FLB=0 in 2006)} &       (0.0340)    & (0.0539)    & (0.0259)   & (0.0427)    & (0.122)    & (0.0401)     \\[1ex]
$\beta_1$: FLB                              &       -0.00158    & -0.00285    & 0.00510    & -0.0103     & -0.0129**  & -0.00478     \\
$\hookrightarrow$ \textit{Additional ineq. of FLB=1 in 2006} &       (0.00420)   & (0.00638)   & (0.00458)  & (0.0100)    & (0.00634)  & (0.00990)    \\[1ex]
$\beta_2$: Year 2010                        &       -0.0334***  & 0.00454     & -0.0326*** & -0.0142     & -0.0798*** & -0.0151***   \\
$\hookrightarrow$ \textit{Add. ineq. in 2010}        &       (0.00265)   & (0.00424)   & (0.00273)  & (0.0119)    & (0.00842)  & (0.00348)    \\[1ex]
$\beta_3$: FLB$\times$2010                  &       0.00148     & 0.00323     & 0.0217***  & 0.00496     & -0.00420   & 0.0362***    \\
$\hookrightarrow$ \textit{Add. ineq. of FLB in 2010} &       (0.00592)   & (0.00846)   & (0.00672)  & (0.0129)    & (0.00849)  & (0.0114)     \\[1ex]
$\gamma$: Prob. FLB                         &       0.0953***   & -0.000383   & -0.226***  & 0.117**     & -0.00772   & 0.105***     \\
$\hookrightarrow$ \textit{Add. ineq. of predicted FLB status}                                            &       (0.0355)    & (0.0520)    & (0.0250)   & (0.0458)    & (0.125)    & (0.0359)     \\[1ex]
\cmidrule(lr){1-7}
Modal age workers:                          \\[1ex]
\quad \textit{20-29}                        &       -0.0297     & 0.0330      & 0.0101     &             & 0.00383    & -0.0654*     \\
                                            &       (0.0242)    & (0.0455)    & (0.0242)   &             & (0.0127)   & (0.0369)     \\[1ex]
\quad \textit{30-39}                        &       -0.0133     & 0.0204      & 0.0264     & 0.0363***   & 0.0378***  & -0.0557      \\
                                            &       (0.0244)    & (0.0446)    & (0.0244)   & (0.00985)   & (0.0127)   & (0.0370)     \\[1ex]
\quad \textit{40-49}                        &       -0.00313    & 0.0156      & 0.0177     & 0.0120      & 0.0466***  & -0.0577      \\
                                            &       (0.0242)    & (0.0446)    & (0.0243)   & (0.0119)    & (0.0131)   & (0.0360)     \\[1ex]
\quad \textit{50-59}                        &       0.0197      & 0.0124      & 0.0157     & 0.0152      & 0.0470***  & -0.0389      \\
                                            &       (0.0250)    & (0.0451)    & (0.0248)   & (0.0103)    & (0.0139)   & (0.0366)     \\[1ex]
\quad \textit{60+}                          &       0.0322      & 0.103*      & 0.0785***  & 0.0955**    & 0.0309*    & -0.0315      \\
                                            &       (0.0311)    & (0.0572)    & (0.0284)   & (0.0381)    & (0.0166)   & (0.0404)     \\[1ex]
\% of women empl.                           &       -0.0616***  & -0.0490***  & -0.0392*** & -0.0230     & -0.0386*** & -0.0451***   \\
                                            &       (0.00599)   & (0.0119)    & (0.00467)  & (0.0157)    & (0.00986)  & (0.00603)    \\[1ex]
Mean experience empl.                       &       -0.00216*** & -0.00245*** & 0.00344*** & -0.00468*** & -5.36e-05  & -0.000761*** \\
                                            &       (0.000392)  & (0.000522)  & (0.000360) & (0.000980)  & (0.000576) & (0.000287)   \\[1ex]
\% empl. with tert. educ.                   &       0.116***    & 0.0847***   & 0.165***   & 0.249***    & 0.103***   & 0.0799***    \\
                                            &       (0.00836)   & (0.0272)    & (0.00634)  & (0.0386)    & (0.0159)   & (0.00814)    \\[1ex]
\% empl. with sec. educ.                    &       0.0184***   & 0.0532***   & 0.0730***  & 0.0206      & 0.0592***  & 0.00665      \\
                                            &       (0.00498)   & (0.0185)    & (0.00480)  & (0.0275)    & (0.0134)   & (0.00802)    \\[1ex]
\% managers and profess.                    &       0.0931***   & 0.0688***   & 0.0745***  & 0.121***    & 0.251***   & 0.148***     \\
                                            &       (0.00971)   & (0.0204)    & (0.00949)  & (0.0260)    & (0.0111)   & (0.00853)    \\[1ex]
\% part-time empl.                          &       -0.0108     & 0.140***    & 0.109***   & 0.175***    & 0.0419***  & 0.00672      \\
                                            &       (0.00773)   & (0.0126)    & (0.00664)  & (0.0529)    & (0.0108)   & (0.00832)    \\[1ex]
\% permanent contracts                      &       -0.0781***  & -0.0861***  & -0.00416   & -0.00757    & -0.0464**  & -0.160***    \\
                                            &       (0.0101)    & (0.0184)    & (0.00509)  & (0.0173)    & (0.0190)   & (0.0136)     \\[1ex]

Firm size:                                  \\[1ex]
\quad \textit{50--249 empl.}                &       0.000931    & 0.0361***   & 0.121***   & 0.0163      & -0.0631*** & 0.0378***    \\
                                            &       (0.00562)   & (0.00545)   & (0.00357)  & (0.0108)    & (0.0131)   & (0.00429)    \\[1ex]
\quad \textit{$\geq$ 250 empl.}             &       -0.00323    & 0.0340***   & 0.193***   & 0.0112      & -0.0652*** & 0.0490***    \\
                                            &       (0.00967)   & (0.00513)   & (0.00641)  & (0.0164)    & (0.0117)   & (0.00490)    \\[1ex]
Public firm                                 &       -0.0502***  & -0.00360    & 0.0287     & -0.0797***  & 0.0194     & -0.0694***   \\
                                            &       (0.0119)    & (0.0140)    & (0.00823)  & (0.0116)    & (0.0598)   & (0.00937)    \\[1ex]
\midrule
Observations                                &       13,765      & 12,312      & 37,887     & 3,498       & 14,502     & 30,009       \\
R-squared                                   &       0.187       & 0.064       & 0.197      & 0.230       & 0.123      & 0.118        \\
Region FE                                   &       \checkmark  & \checkmark  & \checkmark & \checkmark  & \checkmark & \checkmark   \\
Sector FE                                   &       \checkmark  & \checkmark  & \checkmark & \checkmark  & \checkmark & \checkmark   \\
\bottomrule%
\end{tabular}
%
\begin{tablenotes}
\item Notes: Bootsrapped standard errors in parentheses (200 repetitions);
\item Asterisks denote significance levels: $^{*}$ p$<$0.05, $^{**}$ p$<$0.01, $^{***}$ p$<$0.001
\end{tablenotes}
%
\setlength{\tabcolsep}{6pt}
%
\end{threeparttable}
}

\end{table}

\clearpage

\begin{landscape}
\centering
\begin{table}[htb]
\caption{Decomposition of FLB effects on the 90\textsuperscript{th} and 10\textsuperscript{th} wage percentiles}
\label{tab:disp_avg_90_10}
\centering
\centering
\resizebox{\textwidth}{!}{%
\begin{threeparttable}
% \setlength{\tabcolsep}{0pt}
%
\begin{tabular}{l*{12}{S}}
\toprule
                          % & (1)         & (2)         & (3)          & (4)        & (5)        & (6)         & (7)         & (8)        & (9)        & (10)       & (11)         & (12)       \\
                          & \multicolumn{2}{c}{BE} & \multicolumn{2}{c}{DE} & \multicolumn{2}{c}{ES} & \multicolumn{2}{c}{CZ} & \multicolumn{2}{c}{UK} & \multicolumn{2}{c}{FR} \\
 & \multicolumn{1}{c}{q\_90} & \multicolumn{1}{c}{q\_10} & \multicolumn{1}{c}{q\_90} & \multicolumn{1}{c}{q\_10} & \multicolumn{1}{c}{q\_90} & \multicolumn{1}{c}{q\_10} %
                          & \multicolumn{1}{c}{q\_90} & \multicolumn{1}{c}{q\_10} & \multicolumn{1}{c}{q\_90} & \multicolumn{1}{c}{q\_10} & \multicolumn{1}{c}{q\_90} & \multicolumn{1}{c}{q\_10} \\

\midrule
$\beta_0$: Intercept             &       0.226***    & -0.228***   & 0.270***     & -0.259***  & 0.148***   & -0.151***   & 0.244***    & -0.219***  & 0.307***   & -0.290***  & 0.315***     & -0.309***  \\
                                &       (0.0203)    & (0.0173)    & (0.0289)     & (0.0265)   & (0.0134)   & (0.0112)    & (0.0223)    & (0.0195)   & (0.0722)   & (0.0571)   & (0.0234)     & (0.0173)   \\[1ex]
$\beta_1$: FLB                  &       -0.00106    & 0.000520    & -0.00945***  & -0.00660*  & -0.000250  & -0.00535**  & -0.00479    & 0.00548    & -0.00414   & 0.00877*** & -0.00187     & 0.00292    \\
                                &       (0.00276)   & (0.00244)   & (0.00343)    & (0.00394)  & (0.00260)  & (0.00239)   & (0.00521)   & (0.00535)  & (0.00373)  & (0.00320)  & (0.00500)    & (0.00601)  \\[1ex]
$\beta_2$: Year 2010            &       -0.0189***  & 0.0145***   & 0.00678***   & 0.00223    & -0.0174*** & 0.0151***   & -0.00548    & 0.00871    & -0.0392*** & 0.0406***  & -0.00764***  & 0.00741*** \\
                                &       (0.00158)   & (0.00135)   & (0.00236)    & (0.00262)  & (0.00139)  & (0.00131)   & (0.00623)   & (0.00728)  & (0.00530)  & (0.00419)  & (0.00202)    & (0.00162)  \\[1ex]
$\beta_3$: FLB$\times$2010      &       0.00159     & 0.000108    & 0.00578      & 0.00255    & 0.0148***  & -0.00693**  & 0.00125     & -0.00371   & -0.00431   & -0.000113  & 0.0195***    & -0.0167*** \\
                                &       (0.00332)   & (0.00298)   & (0.00451)    & (0.00452)  & (0.00346)  & (0.00316)   & (0.00693)   & (0.00771)  & (0.00533)  & (0.00443)  & (0.00600)    & (0.00598)  \\[1ex]
$\gamma$: Prob. FLB              &       0.0539***   & -0.0415**   & -0.0153      & -0.0149    & -0.133***  & 0.0928***   & 0.0595**    & -0.0576*** & -0.0112    & -0.00350   & 0.0678***    & -0.0374**  \\
                                &       (0.0173)    & (0.0171)    & (0.0238)     & (0.0296)   & (0.0151)   & (0.0129)    & (0.0259)    & (0.0212)   & (0.0739)   & (0.0573)   & (0.0183)     & (0.0167)   \\[1ex]
\cmidrule(lr){1-13}
Modal age workers:             \\[1ex]
\quad \textit{20-29}            &       -0.0141     & 0.0156      & -0.00923     & -0.0423*   & 0.00651    & -0.00356    &             &            & 0.00736    & 0.00353    & -0.0306      & 0.0348**   \\
                                &       (0.0136)    & (0.0117)    & (0.0251)     & (0.0244)   & (0.0131)   & (0.0109)    &             &            & (0.00635)  & (0.00660)  & (0.0211)     & (0.0165)   \\[1ex]
\quad \textit{30-39}            &       -0.00464    & 0.00867     & -0.0135      & -0.0339    & 0.0153     & -0.0111     & 0.0183***   & -0.0180*** & 0.0267***  & -0.0111*   & -0.0250      & 0.0308*    \\
                                &       (0.0138)    & (0.0115)    & (0.0250)     & (0.0239)   & (0.0129)   & (0.0108)    & (0.00523)   & (0.00603)  & (0.00684)  & (0.00663)  & (0.0209)     & (0.0166)   \\[1ex]
\quad \textit{40-49}            &       0.00105     & 0.00418     & -0.0175      & -0.0331    & 0.0108     & -0.00687    & 0.00668     & -0.00533   & 0.0315***  & -0.0151**  & -0.0249      & 0.0329**   \\
                                &       (0.0139)    & (0.0116)    & (0.0249)     & (0.0238)   & (0.0131)   & (0.0108)    & (0.00628)   & (0.00640)  & (0.00635)  & (0.00674)  & (0.0208)     & (0.0166)   \\[1ex]
\quad \textit{50-59}            &       0.0127      & -0.00702    & -0.0167      & -0.0291    & 0.0101     & -0.00561    & 0.00820     & -0.00701   & 0.0332***  & -0.0138*   & -0.0146      & 0.0243     \\
                                &       (0.0141)    & (0.0118)    & (0.0252)     & (0.0237)   & (0.0133)   & (0.0110)    & (0.00538)   & (0.00583)  & (0.00681)  & (0.00723)  & (0.0210)     & (0.0164)   \\[1ex]
\quad \textit{60+}              &       0.0193      & -0.0129     & 0.0232       & -0.0798**  & 0.0437***  & -0.0348***  & 0.0538***   & -0.0417**  & 0.0233***  & -0.00762   & -0.0108      & 0.0206     \\
                                &       (0.0175)    & (0.0160)    & (0.0317)     & (0.0325)   & (0.0147)   & (0.0128)    & (0.0184)    & (0.0176)   & (0.00895)  & (0.00815)  & (0.0224)     & (0.0186)   \\[1ex]
\% of women empl.               &       -0.0294***  & 0.0323***   & -0.0203***   & 0.0287***  & -0.0176*** & 0.0216***   & -0.00670    & 0.0163**   & -0.0215*** & 0.0171***  & -0.0255***   & 0.0196***  \\
                                &       (0.00340)   & (0.00323)   & (0.00576)    & (0.00646)  & (0.00280)  & (0.00247)   & (0.00815)   & (0.00750)  & (0.00520)  & (0.00427)  & (0.00333)    & (0.00321)  \\[1ex]
Mean experience empl.           &       -0.00125*** & 0.000907*** & -0.000873*** & 0.00158*** & 0.00188*** & -0.00156*** & -0.00276*** & 0.00192*** & -3.41e-05  & 1.95e-05   & -0.000512*** & 0.000249*  \\
                                &       (0.000228)  & (0.000190)  & (0.000266)   & (0.000301) & (0.000215) & (0.000176)  & (0.000548)  & (0.000481) & (0.000359) & (0.000287) & (0.000158)   & (0.000144) \\[1ex]
\% empl. with tert. educ.       &       0.0587***   & -0.0577***  & 0.0597***    & -0.0251    & 0.0844***  & -0.0804***  & 0.130***    & -0.119***  & 0.0494***  & -0.0531*** & 0.0369***    & -0.0430*** \\
                                &       (0.00431)   & (0.00406)   & (0.0134)     & (0.0169)   & (0.00319)  & (0.00306)   & (0.0202)    & (0.0193)   & (0.00939)  & (0.00736)  & (0.00473)    & (0.00388)  \\[1ex]
\% empl. with sec. educ.        &       0.00871***  & -0.00971*** & 0.0322***    & -0.0210*   & 0.0384***  & -0.0346***  & -0.00260    & -0.0232    & 0.0264***  & -0.0328*** & -0.000259    & -0.00690*  \\
                                &       (0.00272)   & (0.00263)   & (0.00874)    & (0.0114)   & (0.00267)  & (0.00239)   & (0.0150)    & (0.0145)   & (0.00822)  & (0.00699)  & (0.00400)    & (0.00370)  \\[1ex]
\% managers and profess.        &       0.0438***   & -0.0493***  & 0.0319***    & -0.0369*** & 0.0382***  & -0.0364***  & 0.0571***   & -0.0637*** & 0.125***   & -0.126***  & 0.0820***    & -0.0660*** \\
                                &       (0.00600)   & (0.00515)   & (0.00957)    & (0.0108)   & (0.00466)  & (0.00488)   & (0.0166)    & (0.0131)   & (0.00592)  & (0.00558)  & (0.00484)    & (0.00412)  \\[1ex]
\% part-time empl.              &       -0.00418    & 0.00662*    & 0.0570***    & -0.0829*** & 0.0551***  & -0.0535***  & 0.0777***   & -0.0969*** & 0.0258***  & -0.0161*** & 0.00355      & -0.00317   \\
                                &       (0.00418)   & (0.00375)   & (0.00675)    & (0.00791)  & (0.00342)  & (0.00332)   & (0.0301)    & (0.0255)   & (0.00612)  & (0.00545)  & (0.00412)    & (0.00395)  \\[1ex]
\% permanent contracts          &       -0.0345***  & 0.0436***   & -0.0242***   & 0.0619***  & 0.000483   & 0.00465*    & -0.00203    & 0.00554    & -0.0277*** & 0.0188*    & -0.0778***   & 0.0826***  \\
                                &       (0.00567)   & (0.00617)   & (0.00906)    & (0.0107)   & (0.00255)  & (0.00238)   & (0.00795)   & (0.00835)  & (0.00961)  & (0.00979)  & (0.00750)    & (0.00651)  \\[1ex]

Firm size:                      \\[1ex]
\quad \textit{50--249 empl.}    &       0.000439    & -0.000491   & 0.0217***    & -0.0144*** & 0.0657***  & -0.0554***  & 0.00895     & -0.00736   & -0.0330*** & 0.0301***  & 0.0201***    & -0.0177*** \\
                                &       (0.00275)   & (0.00287)   & (0.00268)    & (0.00314)  & (0.00235)  & (0.00182)   & (0.00578)   & (0.00566)  & (0.00776)  & (0.00724)  & (0.00258)    & (0.00217)  \\[1ex]
\quad \textit{$\geq$ 250 empl.} &       -0.00293    & 0.000297    & 0.0239***    & -0.0101*** & 0.106***   & -0.0876***  & 0.00620     & -0.00498   & -0.0332*** & 0.0320***  & 0.0226***    & -0.0264*** \\
                                &       (0.00469)   & (0.00472)   & (0.00275)    & (0.00313)  & (0.00397)  & (0.00347)   & (0.00880)   & (0.00795)  & (0.00663)  & (0.00592)  & (0.00256)    & (0.00230)  \\[1ex]
Public firm                     &       -0.0219***  & 0.0283***   & -0.0129*     & -0.00926   & 0.0183***  & -0.0105***  & -0.0448***  & 0.0349***  & 0.00734    & -0.0121    & -0.0402***   & 0.0292***  \\
                                &       (0.00592)   & (0.00583)   & (0.00663)    & (0.00830)  & (0.00436)  & (0.00379)   & (0.00719)   & (0.00558)  & (0.0352)   & (0.0268)   & (0.00458)    & (0.00407)  \\[1ex]
\midrule
Observations                    &       13,765      & 13,765      & 12,312       & 12,312     & 37,887     & 37,887      & 3,498       & 3,498      & 14,502     & 14,502     & 30,009       & 30,009     \\
R-squared                       &       0.138       & 0.199       & 0.059        & 0.059      & 0.174      & 0.191       & 0.226       & 0.191      & 0.110      & 0.124      & 0.105        & 0.115      \\
Region FE                       &       \checkmark  & \checkmark  & \checkmark   & \checkmark & \checkmark & \checkmark  & \checkmark  & \checkmark & \checkmark & \checkmark & \checkmark   & \checkmark \\
Sector FE                       &       \checkmark  & \checkmark  & \checkmark   & \checkmark & \checkmark & \checkmark  & \checkmark  & \checkmark & \checkmark & \checkmark & \checkmark   & \checkmark \\
\bottomrule
\end{tabular}
%
\begin{tablenotes}
\item Notes: Bootsrapped standard errors in parentheses (200 repetitions); asterisks denote significance levels: $^{*}$ p$<$0.05, $^{**}$ p$<$0.01, $^{***}$ p$<$0.001
\end{tablenotes}
%
\setlength{\tabcolsep}{6pt}
\end{threeparttable}
}

\end{table}
\end{landscape}

\clearpage

\begin{table}[htb]
\caption{FLB and the wage-gap between managers and low-layers workers}
\label{tab:disp_man_ls}
\centering
\centering
\tiny
\begin{threeparttable}
%\setlength{\tabcolsep}{0pt}
%
\begin{tabular}{l*{6}{S}}
\toprule
%                         &  (1)       & (2)        & (3)        & (4)        & (5)       & (6)         \\
                          & \multicolumn{1}{c}{BE} & \multicolumn{1}{c}{DE} & \multicolumn{1}{c}{ES} %
                          & \multicolumn{1}{c}{CZ} & \multicolumn{1}{c}{UK} & \multicolumn{1}{c}{FR} \\
 \midrule
FLB                       &  0.00463   & -0.0607*** & -0.00730   & 0.0156     & 0.0307    & -0.0862***  \\
                          &  (0.0227)  & (0.0231)   & (0.0224)   & (0.0221)   & (0.0315)  & (0.0248)    \\[1ex]
Year 2010                 &  0.00909   & 0.0167     & 0.0415**   & -0.106***  & -0.106**  & -0.0116     \\
                          &  (0.0163)  & (0.0157)   & (0.0199)   & (0.0296)   & (0.0537)  & (0.0119)    \\[1ex]
FLB$\times$2010           &  -0.0251   & 0.0555**   & -0.0839*** & 0.0352     & -0.0258   & 0.0190      \\
                          &  (0.0295)  & (0.0278)   & (0.0323)   & (0.0318)   & (0.0554)  & (0.0302)    \\[1ex]
Prob. FLB                 &  -0.311*   & -0.756***  & -0.710***  & 0.367***   & 0.493     & 0.145       \\
                          &  (0.161)   & (0.168)    & (0.182)    & (0.104)    & (0.676)   & (0.137)     \\[1ex]

Modal age workers:        \\[1ex]
\quad \textit{20-29}      &  0.0630    & 0.149      & 0.144***   &            & -0.218*   & 0.176       \\
                          &  (0.0696)  & (0.142)    & (0.0441)   &            & (0.121)   & (0.169)     \\[1ex]
\quad \textit{30-39}      &  0.0957    & 0.243*     & 0.143***   & 0.0335     & -0.227*   & 0.201       \\
                          &  (0.0690)  & (0.141)    & (0.0342)   & (0.0273)   & (0.124)   & (0.168)     \\[1ex]
\quad \textit{40-49}      &  0.118*    & 0.237*     & 0.129***   & -0.0599**  & -0.180    & 0.231       \\
                          &  (0.0687)  & (0.140)    & (0.0342)   & (0.0287)   & (0.123)   & (0.168)     \\[1ex]
\quad \textit{50-59}      &  0.130*    & 0.220      & 0.145***   & -0.0348    & -0.164    & 0.270       \\
                          &  (0.0698)  & (0.139)    & (0.0374)   & (0.0279)   & (0.121)   & (0.168)     \\[1ex]
\quad \textit{60+}        &  0.118     & 0.279*     & 0.241***   & 0.167***   & -0.243*   & 0.283*      \\
                          &  (0.156)   & (0.156)    & (0.0917)   & (0.0600)   & (0.136)   & (0.172)     \\[1ex]
\% of women empl.         &  0.0555*   & 0.130***   & 0.197***   & 0.00441    & 0.0411    & -0.0418     \\
                          &  (0.0305)  & (0.0380)   & (0.0389)   & (0.0371)   & (0.0539)  & (0.0255)    \\[1ex]
Mean experience empl.     &  0.00489** & -0.00157   & 0.00386    & -0.00457** & -0.00435  & -0.00292*** \\
                          &  (0.00199) & (0.00164)  & (0.00266)  & (0.00226)  & (0.00302) & (0.00106)   \\[1ex]
\% empl. with tert. educ. &  0.110**   & 0.0636     & 0.233***   & 0.358***   & 0.147     & -0.190***   \\
                          &  (0.0473)  & (0.0759)   & (0.0505)   & (0.0820)   & (0.100)   & (0.0290)    \\[1ex]
\% empl. with sec. educ.  &  0.0672**  & 0.203***   & 0.0760*    & -0.0734    & -0.00143  & -0.0784***  \\
                          &  (0.0305)  & (0.0607)   & (0.0408)   & (0.0531)   & (0.0930)  & (0.0288)    \\[1ex]
\% managers and profess.  &  0.0408    & -0.157***  & -0.215**   & -0.399***  & 0.139*    & -0.0199     \\
                          &  (0.0565)  & (0.0533)   & (0.0859)   & (0.0600)   & (0.0751)  & (0.0353)    \\[1ex]
\% part-time empl.        &  -0.0238   & -0.145***  & -0.162***  & -0.337***  & -0.0800   & 0.0610      \\
                          &  (0.0493)  & (0.0376)   & (0.0612)   & (0.0917)   & (0.0775)  & (0.0382)    \\[1ex]
\% permanent contracts    &  0.196***  & 0.0388     & 0.212***   & 0.0339     & 0.0994    & 0.0180      \\
                          &  (0.0522)  & (0.0560)   & (0.0460)   & (0.0392)   & (0.158)   & (0.0532)    \\[1ex]

Firm size:                \\[1ex]
\quad \textit{50--249 empl.}    &  0.107***  & 0.0991***  & 0.179***   & 0.0602**   & 0.0570    & -0.0878***  \\
                          &  (0.0302)  & (0.0241)   & (0.0298)   & (0.0248)   & (0.0580)  & (0.0159)    \\[1ex]
\quad \textit{$\geq$ 250 empl.} &  0.146***  & 0.0469**   & 0.215***   & 0.0447     & -0.0330   & -0.177***   \\
                          &  (0.0473)  & (0.0230)   & (0.0565)   & (0.0337)   & (0.0579)  & (0.0156)    \\[1ex]
Public firm               &  -0.0764   & -0.268***  & 0.0199     & -0.131***  & 0.184     & -0.119***   \\
                          &  (0.0487)  & (0.0458)   & (0.0504)   & (0.0254)   & (0.325)   & (0.0383)    \\[1ex]
Constant                  &  -0.322**  & -0.375**   & -0.352***  & 0.0358     & -0.564    & 0.134       \\
                          &  (0.154)   & (0.176)    & (0.0921)   & (0.102)    & (0.625)   & (0.181)     \\[1ex]
\midrule
Observations              &  2,416     & 4,396      & 3,443      & 3,006      & 2,059     & 6,895       \\
R-squared                 &  0.087     & 0.083      & 0.091      & 0.158      & 0.078     & 0.069       \\
Region FE               &  \checkmark& \checkmark & \checkmark & \checkmark & \checkmark& \checkmark  \\
Sector FE                 &  \checkmark& \checkmark & \checkmark & \checkmark & \checkmark& \checkmark  \\
\bottomrule%
\end{tabular}
%
\begin{tablenotes}
\item Notes: Bootsrapped standard errors in parentheses (200 repetitions); asterisks denote significance levels: $^{*}$ p$<$0.05, $^{**}$ p$<$0.01, $^{***}$ p$<$0.001
\end{tablenotes}
%
\setlength{\tabcolsep}{6pt}
%
\end{threeparttable}

\end{table}

\clearpage

\begin{landscape}
\centering
\begin{table}[htb]
\caption{Decomposition of FLB effects across managers and low-layers workers}
\label{tab:disp_avg_man_ls}
\centering
\tiny
\begin{threeparttable}
% \setlength{\tabcolsep}{0pt}
%
\begin{tabular}{l*{12}{S}}
\toprule
%                         & (1)       & (2)         & (3)        & (4)        & (5)       & (6)        & (7)        & (8)        & (9)       & (10)      & (11)        & (12)       \\

& \multicolumn{2}{c}{BE} & \multicolumn{2}{c}{DE} & \multicolumn{2}{c}{ES} & \multicolumn{2}{c}{CZ} & \multicolumn{2}{c}{UK} & \multicolumn{2}{c}{FR} \\

& \multicolumn{1}{c}{Manag} & \multicolumn{1}{c}{Low} & \multicolumn{1}{c}{Manag} & \multicolumn{1}{c}{Low} & \multicolumn{1}{c}{Manag} & \multicolumn{1}{c}{Low} 
                          & \multicolumn{1}{c}{Manag} & \multicolumn{1}{c}{Low} & \multicolumn{1}{c}{Manag} & \multicolumn{1}{c}{Low} & \multicolumn{1}{c}{Manag} & \multicolumn{1}{c}{Low} \\

\midrule
FLB                       &  0.00605   & 0.00142     & -0.0466**  & 0.0141     & -0.0128   & -0.00548   & 0.0195     & 0.00385    & 0.0296    & -0.00107  & -0.0412**   & 0.0451***  \\
                          &  (0.0177)  & (0.00735)   & (0.0207)   & (0.0117)   & (0.0179)  & (0.00888)  & (0.0176)   & (0.00671)  & (0.0224)  & (0.0150)  & (0.0176)    & (0.0148)   \\[1ex]
Year 2010                 &  9.62e-05  & -0.00899    & 0.00862    & -0.00812   & 0.0229    & -0.0186**  & -0.0790*** & 0.0268**   & -0.0702*  & 0.0353    & -0.00862    & 0.00299    \\
                          &  (0.0125)  & (0.00581)   & (0.0143)   & (0.00686)  & (0.0158)  & (0.00749)  & (0.0221)   & (0.0105)   & (0.0369)  & (0.0267)  & (0.00838)   & (0.00565)  \\[1ex]
FLB$\times$2010           &  -0.0198   & 0.00531     & 0.0390*     & -0.0165    & -0.0485*  & 0.0354***  & 0.0117     & -0.0236  & -0.0413   & -0.0155   & 0.00421     & -0.0148    \\
                          &  (0.0261)  & (0.00856)   & (0.0249)   & (0.0157)   & (0.0276)  & (0.0111)   & (0.0245)   & (0.0105)   & (0.0390)  & (0.0256)  & (0.0214)    & (0.0190)   \\[1ex]
Prob. FLB    &  0.0164    & 0.327***    & -0.540***  & 0.216***   & -0.498*** & 0.212***   & 0.284***   & -0.0827**  & 0.564     & 0.0711    & 0.0362      & -0.109*    \\
                          &  (0.139)   & (0.0717)    & (0.130)    & (0.0743)   & (0.153)   & (0.0772)   & (0.0939)   & (0.0402)   & (0.455)   & (0.308)   & (0.104)     & (0.0576)   \\[1ex]

Modal age workers:        \\[1ex]
\quad \textit{20-29}      &  0.0241    & -0.0389     & 0.134      & -0.0147    & 0.139***  & -0.00526   &            &            & -0.211**  & 0.00735   & 0.109       & -0.0674    \\
                          &  (0.0999)  & (0.0238)    & (0.145)    & (0.0569)   & (0.0328)  & (0.0144)   &            &            & (0.0969)  & (0.0544)  & (0.100)     & (0.0727)   \\[1ex]
\quad \textit{30-39}      &  0.0402    & -0.0555**   & 0.200      & -0.0427    & 0.144***  & 0.00146    & 0.0197     & -0.0137    & -0.211**  & 0.0165    & 0.130       & -0.0713    \\
                          &  (0.0998)  & (0.0238)    & (0.144)    & (0.0611)   & (0.0273)  & (0.0124)   & (0.0245)   & (0.0106)   & (0.0976)  & (0.0556)  & (0.0988)    & (0.0722)   \\[1ex]
\quad \textit{40-49}      &  0.0583    & -0.0593**   & 0.194      & -0.0424    & 0.129***  & 0.000315   & -0.0568**  & 0.00309    & -0.178*   & 0.00148   & 0.143       & -0.0876    \\
                          &  (0.0984)  & (0.0234)    & (0.145)    & (0.0590)   & (0.0273)  & (0.0109)   & (0.0271)   & (0.0107)   & (0.0985)  & (0.0543)  & (0.0993)    & (0.0719)   \\[1ex]
\quad \textit{50-59}      &  0.0733    & -0.0566**   & 0.179      & -0.0406    & 0.134***  & -0.0104    & -0.0332    & 0.00163    & -0.195**  & -0.0313   & 0.167*      & -0.103     \\
                          &  (0.100)   & (0.0246)    & (0.146)    & (0.0590)   & (0.0285)  & (0.0113)   & (0.0253)   & (0.0104)   & (0.0988)  & (0.0539)  & (0.0987)    & (0.0723)   \\[1ex]
\quad \textit{60+}        &  0.0541    & -0.0636     & 0.209      & -0.0704    & 0.211***  & -0.0297    & 0.0964*    & -0.0706**  & -0.232**  & 0.0105    & 0.178*      & -0.105     \\
                          &  (0.133)   & (0.0582)    & (0.157)    & (0.0701)   & (0.0673)  & (0.0324)   & (0.0512)   & (0.0293)   & (0.107)   & (0.0573)  & (0.102)     & (0.0779)   \\[1ex]
\% of women empl.         &  0.0427    & -0.0128     & 0.0672**   & -0.0626*** & 0.133***  & -0.0634*** & 0.0469     & 0.0425***  & -0.000410 & -0.0415   & -0.0389**   & 0.00291    \\
                          &  (0.0319)  & (0.0140)    & (0.0312)   & (0.0195)   & (0.0331)  & (0.0149)   & (0.0289)   & (0.0123)   & (0.0418)  & (0.0268)  & (0.0155)    & (0.0109)   \\[1ex]
Mean experience empl.     &  0.00110   & -0.00379*** & -0.000924  & 0.000641   & 0.00212   & -0.00173*  & -0.00354*  & 0.00103    & -0.00236  & 0.00199   & -0.00254*** & 0.000382   \\
                          &  (0.00187) & (0.000921)  & (0.00150)  & (0.000740) & (0.00218) & (0.000979) & (0.00206)  & (0.000815) & (0.00218) & (0.00146) & (0.000755)  & (0.000562) \\[1ex]
\% empl. with tert. educ. &  0.0552    & -0.0550***  & -0.0876    & -0.151***  & 0.158***  & -0.0743*** & 0.261***   & -0.0971*** & 0.0715    & -0.0757*  & -0.102***   & 0.0886***  \\
                          &  (0.0344)  & (0.0197)    & (0.0683)   & (0.0385)   & (0.0364)  & (0.0181)   & (0.0714)   & (0.0325)   & (0.0726)  & (0.0449)  & (0.0218)    & (0.0153)   \\[1ex]
\% empl. with sec. educ.  &  0.0471*   & -0.0202*    & 0.0750     & -0.128***  & 0.0543    & -0.0218    & -0.0544    & 0.0190     & -0.0254   & -0.0239   & -0.0588***  & 0.0196*    \\
                          &  (0.0254)  & (0.0108)    & (0.0590)   & (0.0212)   & (0.0350)  & (0.0138)   & (0.0540)   & (0.0178)   & (0.0630)  & (0.0423)  & (0.0196)    & (0.0112)   \\[1ex]
\% managers and profess.  &  0.0930**  & 0.0522*     & -0.0328    & 0.124***   & -0.123*   & 0.0927**   & -0.205***  & 0.194***   & 0.115**   & -0.0236   & 0.000121    & 0.0201     \\
                          &  (0.0470)  & (0.0293)    & (0.0433)   & (0.0376)   & (0.0630)  & (0.0366)   & (0.0475)   & (0.0321)   & (0.0457)  & (0.0421)  & (0.0236)    & (0.0225)   \\[1ex]
\% part-time empl.        &  -0.0382   & -0.0144     & -0.0842*** & 0.0609***  & -0.100**  & 0.0610***  & -0.260***  & 0.0766***  & -0.0671   & 0.0128    & 0.0511**    & -0.00988   \\
                          &  (0.0461)  & (0.0183)    & (0.0294)   & (0.0206)   & (0.0505)  & (0.0191)   & (0.0898)   & (0.0290)   & (0.0528)  & (0.0336)  & (0.0236)    & (0.0157)   \\[1ex]
\% permanent contracts    &  0.0518    & -0.144***   & 0.0115     & -0.0273    & 0.164***  & -0.0479*** & 0.0377     & 0.00382    & 0.131     & 0.0311    & 0.0670**    & 0.0490*    \\
                          &  (0.0567)  & (0.0231)    & (0.0462)   & (0.0295)   & (0.0402)  & (0.0154)   & (0.0344)   & (0.0124)   & (0.114)   & (0.0577)  & (0.0338)    & (0.0294)   \\[1ex]

Firm size:                \\[1ex]
\quad \textit{50--249 empl.}    &  0.0613*** & -0.0455***  & 0.109***   & 0.00998    & 0.157***  & -0.0221**  & 0.0643***  & 0.00412    & 0.0667*   & 0.00975   & -0.0532***  & 0.0346***  \\
                          &  (0.0238)  & (0.0122)    & (0.0178)   & (0.00996)  & (0.0250)  & (0.0109)   & (0.0213)   & (0.0106)   & (0.0398)  & (0.0311)  & (0.0115)    & (0.00792)  \\[1ex]
\quad \textit{$\geq$ 250 empl.}&  0.0662    & -0.0801***  & 0.0595***  & 0.0126     & 0.183***  & -0.0324    & 0.0640**   & 0.0193     & 0.0349    & 0.0679*** & -0.104***   & 0.0730***  \\
                          &  (0.0409)  & (0.0216)    & (0.0199)   & (0.0109)   & (0.0451)  & (0.0221)   & (0.0308)   & (0.0149)   & (0.0379)  & (0.0262)  & (0.0108)    & (0.00859)  \\[1ex]
Public firm               &  -0.0209   & 0.0555**    & -0.182***  & 0.0860***  & 0.00974   & -0.0101    & -0.127***  & 0.00489    & 0.247     & 0.0632    & -0.0541*    & 0.0653***  \\
                          &  (0.0444)  & (0.0266)    & (0.0366)   & (0.0224)   & (0.0411)  & (0.0204)   & (0.0209)   & (0.0118)   & (0.220)   & (0.149)   & (0.0300)    & (0.0168)   \\[1ex]
Constant                  &  -0.163    & 0.159***    & -0.212     & 0.163**    & -0.295*** & 0.0568     & 0.0237     & -0.0121    & -0.556    & 0.00833   & 0.0460      & -0.0877    \\
                          &  (0.134)   & (0.0472)    & (0.180)    & (0.0654)   & (0.0748)  & (0.0351)   & (0.0857)   & (0.0274)   & (0.421)   & (0.286)   & (0.111)     & (0.0828)   \\
\midrule
Observations              &  2,416     & 2,416       & 4,396      & 4,396      & 3,443     & 3,443      & 3,006      & 3,006      & 2,059     & 2,059     & 6,895       & 6,895      \\
R-squared                 &  0.046     & 0.347       & 0.063      & 0.054      & 0.079     & 0.061      & 0.169      & 0.098      & 0.054     & 0.113     & 0.052       & 0.053      \\
Region FE               &  \checkmark& \checkmark  & \checkmark & \checkmark & \checkmark& \checkmark & \checkmark & \checkmark & \checkmark& \checkmark& \checkmark  & \checkmark \\
Sector FE                 &  \checkmark& \checkmark  & \checkmark & \checkmark & \checkmark& \checkmark & \checkmark & \checkmark & \checkmark& \checkmark& \checkmark  & \checkmark \\
\bottomrule
\end{tabular}
 
\begin{tablenotes}
\item Notes: Bootsrapped standard errors in parentheses (200 repetitions); asterisks denote significance levels: $^{*}$ p$<$0.05, $^{**}$ p$<$0.01, $^{***}$ p$<$0.001
\end{tablenotes}
%
\setlength{\tabcolsep}{6pt}
%
\end{threeparttable}

\end{table}
\end{landscape}



\clearpage
\section*{Appendix~A: Summary statistics of explanatory variables}

\begin{table}[ht]
\centering
\small
\label{tab:sum_reg_var_size}
\caption{Distribution of firms by size by country}
\begin{tabular}{lrrrrrr}
\toprule
Firm size: & BE   & CZ    & DE    & ES    & FR    & UK    \\
\midrule
1--49       & 4,547 & 22,218  & 26,592 & 19,695 &  7,818 &  9,343 \\
50--249     & 4,533 &  6,633  & 19,021 &  8,662 &  9,019 &  2,765 \\
$\geq$250    & 5,766 &  3,251  & 15,177 & 10,400 & 14,058 & 20,332 \\
\bottomrule
\end{tabular}


\end{table}

\begin{table}[htp]
\label{tab:sum_reg_var_cont}
\caption{Summary means and standard deviations for continuous variables in regression}
\centering
\small
\begin{tabular}{llrrrrrr}
\toprule
country                     & Stat. & BE    & CZ    & DE    & ES    & FR     & UK    \\
\midrule
Mean experience empl. (yrs) & mean & 9,506 & 9,061 & 8,574 & 7,235 & 10,748 & 7,079 \\
                            & s.d. & 5,810 & 5,229 & 6,051 & 6,202 & 6,361  & 5,278 \\
\% empl. with tert. educ.   & mean & 0,329 & 0,244 & 0,123 & 0,318 & 0,408  & 0,379 \\
                            & s.d. & 0,330 & 0,253 & 0,195 & 0,327 & 0,298  & 0,266 \\
\% empl. with sec. educ.    & mean & 0,419 & 0,654 & 0,706 & 0,190 & 0,420  & 0,523 \\
                            & s.d. & 0,305 & 0,255 & 0,236 & 0,244 & 0,270  & 0,262 \\
\% managers and profess.    & mean & 0,197 & 0,398 & 0,110 & 0,142 & 0,293  & 0,273 \\
                            & s.d. & 0,287 & 0,303 & 0,175 & 0,241 & 0,256  & 0,266 \\
\% part-time empl.          & mean & 0,229 & 0,139 & 0,266 & 0,152 & 0,159  & 0,269 \\
                            & s.d. & 0,260 & 0,186 & 0,248 & 0,247 & 0,231  & 0,268 \\
\% permanent contracts      & mean & 0,925 & 0,785 & 0,885 & 0,750 & 0,921  & 0,938 \\
                            & s.d. & 0,162 & 0,215 & 0,148 & 0,303 & 0,171  & 0,156 \\
\bottomrule                            
\end{tabular}


\end{table}

\begin{table}[ht]
\label{tab:sum_reg_var_mod_age}
\caption{Distribution of firms by modal age of employees by country}
\centering
\small
\begin{tabular}{lrrrrrr}
\toprule
Modal age workers: & BE   & CZ    & DE    & ES    & FR   & UK   \\
\midrule
14-19              & 35   & 19    & 276   & 127   & 68   & 987  \\
20-29              & 2414 & 2288  & 9277  & 8726  & 3664 & 6954 \\
30-39              & 4681 & 6254  & 10543 & 18674 & 9098 & 8052 \\
40-49              & 5671 & 10787 & 28528 & 11421 & 9926 & 8929 \\
50-59              & 2417 & 12166 & 11563 & 5275  & 7648 & 6174 \\
60+                & 54   & 588   & 603   & 564   & 491  & 1371 \\
\bottomrule
\end{tabular}

\end{table}


\clearpage
\section*{Appendix~B: FLB propensity scores}

\begin{table}[thb]
\caption{Probit estimates of FLB propensity}
\label{tab:propensity}
\centering
\tiny
\begin{threeparttable}
\begin{tabular}{l*{6}{S}}
\toprule
                          & \multicolumn{1}{c}{(1)} & \multicolumn{1}{c}{(2)} & \multicolumn{1}{c}{(3)} & \multicolumn{1}{c}{(4)} & \multicolumn{1}{c}{(5)} & \multicolumn{1}{c}{(6)} \\
                          & \multicolumn{1}{c}{BE}  & \multicolumn{1}{c}{DE}  & \multicolumn{1}{c}{ES}  & \multicolumn{1}{c}{CZ}  & \multicolumn{1}{c}{UK}  & \multicolumn{1}{c}{FR}  \\
\midrule
Mean experience empl.     &  0.0300*** & 0.0243*** & 0.0409***  & 0.0755*** & -0.0127*** & 0.00221    \\
                          &  (0.00313) & (0.00276) & (0.00184)  & (0.00730) & (0.00277)  & (0.00282)  \\[0.5ex]

Modal age workers:        \\[1ex]
\quad 20-29               &  -0.0844   & -0.890*** & -0.110     &           & 0.115      & 0.125      \\
                          &  (0.252)   & (0.246)   & (0.218)    &           & (0.0936)   & (0.142)    \\[0.5ex]

\quad 30-39               &  0.0366    & -0.564**  & -0.0319    & -0.0328   & 0.141      & 0.285**    \\
                          &  (0.250)   & (0.245)   & (0.218)    & (0.104)   & (0.0938)   & (0.136)    \\[0.5ex]

\quad 40-49               &  0.00410   & -0.498**  & -0.0881    & 0.0855    & 0.124      & 0.395***   \\
                          &  (0.249)   & (0.244)   & (0.218)    & (0.117)   & (0.0934)   & (0.134)    \\[0.5ex]

\quad 50-59               &  -0.124    & -0.559**  & -0.0560    & 0.0737    & 0.152      & 0.420***   \\
                          &  (0.251)   & (0.245)   & (0.219)    & (0.107)   & (0.0956)   & (0.134)    \\[0.5ex]

\quad 60+                 &            & -0.829*** & -0.113     & -0.357    & 0.120      &            \\
                          &            & (0.298)   & (0.232)    & (0.299)   & (0.110)    &            \\[0.5ex]

\% empl. with tert. educ. &  0.109     & 0.662***  & 0.405***   & 0.455     & -0.198**   & 0.108      \\
                          &  (0.0793)  & (0.135)   & (0.0393)   & (0.347)   & (0.0991)   & (0.0759)   \\[0.5ex]

\% empl. with sec. educ.  &  0.137**   & 0.728***  & 0.159***   & 0.117     & -0.133     & 0.522***   \\
                          &  (0.0557)  & (0.0982)  & (0.0382)   & (0.237)   & (0.0945)   & (0.0699)   \\[0.5ex]

\% managers and profess.  &  -0.0209   & 0.0162    & -0.186***  & 0.206     & -0.00602   & -0.581***  \\
                          &  (0.0906)  & (0.110)   & (0.0604)   & (0.290)   & (0.0594)   & (0.0785)   \\[0.5ex]

\% part-time empl.        &  -0.152**  & 0.115*    & -0.0164    & 0.0864    & -0.0706    & 0.386***   \\
                          &  (0.0682)  & (0.0648)  & (0.0430)   & (0.366)   & (0.0560)   & (0.0558)   \\[0.5ex]

\% permanent contracts    &  0.510***  & 0.116     & -0.156***  & -0.279*   & 0.242**    & -0.262***  \\
                          &  (0.115)   & (0.127)   & (0.0385)   & (0.151)   & (0.0978)   & (0.0797)   \\[0.5ex]

Firm size:                \\[1ex]
\quad \textit{50-249 empl.}              &  0.704***  & 0.105***  & 0.624***   & 0.440***  & -0.184**   & 0.473***   \\
                          &  (0.0408)  & (0.0383)  & (0.0235)   & (0.0855)  & (0.0720)   & (0.0385)   \\[0.5ex]

\quad \textit{$\geq$ 250 empl.}         &  1.135***  & 0.179***  & 1.133***   & 1.164***  & -0.290***  & 0.624***   \\
                          &  (0.0425)  & (0.0375)  & (0.0233)   & (0.0916)  & (0.0511)   & (0.0387)   \\[0.5ex]

Public firm               &  -1.230*** & -0.921*** & 0.570***   & 0.202     & -1.346***  & 0.838***   \\
                          &  (0.0964)  & (0.0404)  & (0.0405)   & (0.139)   & (0.0403)   & (0.0428)   \\[0.5ex]

NACE Sector:              \\[1ex]
\quad D                   &  -0.413*   & -0.293**  & -0.380***  & 0.609***  & 0.201      & -1.373***  \\
                          &  (0.243)   & (0.137)   & (0.0667)   & (0.218)   & (0.311)    & (0.180)    \\[0.5ex]

\quad E                   &  -0.809*** & 0.407***  & 0.272***   & 1.075***  & 0.141      & 1.157***   \\
                          &  (0.304)   & (0.141)   & (0.0803)   & (0.305)   & (0.328)    & (0.155)    \\[0.5ex]

\quad F                   &  -1.140*** & -1.096*** & -0.882***  & -0.894*** & -1.032***  & -1.334***  \\
                          &  (0.253)   & (0.203)   & (0.0770)   & (0.225)   & (0.318)    & (0.278)    \\[0.5ex]

\quad G                   &  -0.853*** & -0.451*** & -0.493***  & 0.799***  & 0.405      & -1.270***  \\
                          &  (0.245)   & (0.156)   & (0.0715)   & (0.254)   & (0.311)    & (0.220)    \\[0.5ex]

\quad H                   &  -0.999*** & 0.275     & -0.808***  &           & -0.510     & -0.486***  \\
                          &  (0.268)   & (0.168)   & (0.0832)   &           & (0.334)    & (0.182)    \\[0.5ex]

\quad I                   &  -0.441*   & 1.229***  & -0.112     & -0.168    & -0.248     & 0.573***   \\
                          &  (0.248)   & (0.136)   & (0.0705)   & (0.228)   & (0.310)    & (0.152)    \\[0.5ex]

\quad J                   &  -0.0811   & -0.816*** & -1.105***  & 0.216     & 0.562*     & -0.00925   \\
                          &  (0.255)   & (0.161)   & (0.0818)   & (0.309)   & (0.315)    & (0.158)    \\[0.5ex]

\quad K                   &  -0.744*** & 0.117     & -0.509***  & 1.391***  & 0.0816     & 0.169      \\
                          &  (0.247)   & (0.139)   & (0.0728)   & (0.286)   & (0.314)    & (0.155)    \\[0.5ex]

\quad L                   &            &           & -0.101     & 2.351***  & 0.324      & 1.735***   \\
                          &            &           & (0.111)    & (0.364)   & (0.311)    & (0.159)    \\[0.5ex]

\quad M                   &  -0.966*** & -0.0856   & -0.953***  & 0.142     & -0.676**   & -0.510***  \\
                          &  (0.282)   & (0.157)   & (0.0920)   & (0.286)   & (0.311)    & (0.189)    \\[0.5ex]

\quad N                   &  -0.860*** & 0.973***  & -0.670***  &           & -1.004***  & -0.554***  \\
                          &  (0.248)   & (0.138)   & (0.0834)   &           & (0.312)    & (0.163)    \\[0.5ex]

\quad O                   &  -0.602**  & 0.394***  & 0.126*     & 1.703***  & -0.496     & 0.338**    \\
                          &  (0.254)   & (0.139)   & (0.0744)   & (0.357)   & (0.311)    & (0.155)    \\[0.5ex]

Regional GDP pps          &  0.00120   & 0.000710  & 0.00371    & 0.302***  & 0.00649*** & -0.00184   \\
                          &  (0.00157) & (0.00397) & (0.00231)  & (0.0712)  & (0.00181)  & (0.00168)  \\[0.5ex]

Regional unemp. rate      &  0.00426   & 0.0357*** & 0.00866*** &           & 0.0716***  & -0.0433*** \\
                          &  (0.00366) & (0.00585) & (0.00161)  &           & (0.00832)  & (0.0101)   \\[0.5ex]

Constant                  &  -1.770*** & -1.832*** & -1.679***  & -7.023*** & 0.544      & -2.022***  \\
                          &  (0.374)   & (0.329)   & (0.235)    & (1.437)   & (0.351)    & (0.243)    \\
\midrule
Observations              &  13,730    & 12,312    & 37,887     & 3,498     & 14,502     & 29,943     \\
Area under ROC curve      &  0.781     & 0.811     & 0.825      & 0.870     & 0.875      & 0.935      \\
\bottomrule
\end{tabular}
%
\begin{tablenotes}
\item Notes: Dependent variable is FLB dummy. Standard errors in parentheses; asterisks denote significance levels: $^{*}$ p$<$0.05, $^{**}$ p$<$0.01, $^{***}$ p$<$0.001
\end{tablenotes}

\end{threeparttable}

\end{table}




\end{document}









